\section{Durchführung}
\label{sec:durchfuehrung}

    Es sollen nun erst der Aufbau der Messapparatur und anschließend das Messverfahren beschrieben werden.

    Die Apparatur besteht aus einem Dewar-Gefäß,
    in dem sich ein Rezipient befindet,
    welcher die Kupfer-Probe der Masse \SI{342}{\gram} enthält.
    Die Probe hat keinen Kontakt zum Gefäß,
    sodass der Effekt der Wärmeleitung minimiert wird.
    Sowohl Probe als auch das Gefäß sind mit Heizspulen umwickelt,
    durch die ein Strom geleitet wird.
    \autoref{fig:durchfuehrung:aufbau} zeigt die Messapparatur.
    \begin{figure}
        \centering
        \includegraphics[width=\textwidth]{content/img/Abb_1.pdf}
        \caption{Aufbau der verwendeten Messapparatur. \cite{versuchsanleitung}}
        \label{fig:durchfuehrung:aufbau}
    \end{figure}

    Zu Beginn des Versuchs ist der Rezipient evakuiert.
    Es wird nun zuerst Helium in den Rezipienten gefüllt,
    welches dazu dient,
    die Probe schneller abzukühlen.
    Zusätzlich wird flüssiger Stickstoff in das Dewar-Gefäß gefüllt.
    Die Probe wird nun solange gekühlt,
    bis der Pt100-Widerstand der Probe einen Wert von etwa \SI{22}{\ohm} hat,
    was einer Temperatur von ungefähr \SI{80}{\kelvin} entspricht.
    Der Zusammenahng von Temperatur $T$ und Widerstand $R$ ist über
    \begin{equation}
        T = \num{0.00134} R^2 + \num{2.296} R - \num{243.02}
        \label{eqn:durchfuehrung:pt100}
    \end{equation}
    gegeben,
    wobei die Temperatur in \si{\celsius} berechnet wird.

    Wenn dieser Wert erreicht ist,
    wird das Helium wieder aus dem Rezipienten gepumpt,
    welcher anschließend evakuiert wird.
    Das Vakuum im Dewar-Gefäß dient dazu,
    weiteren Wärmetransport über Konvektion auszuschließen.
    Nun wird für die Heizspule der Probe ein Konstantstrom von \SI{150}{\milli\ampere} eingestellt.
    Der Widerstand,
    und damit die Temperatur der Probe und des Gefäßes,
    erhöht sich nun.
    Mithilfe der zweiten Heizspule um das Gefäß wird dafür gesorgt,
    dass Temperatur von Probe und Gefäß ungefähr gleich bleiben.

    Es wird nun zu Beginn in Schritten von \SI{120}{\second} der Strom beider Spulen,
    sowie der Wert beider Widerstände gemessen.
    Zwischendurch wird das Messintervall auf \SI{5}{\minute} erhöht.
    Wenn eine Temperatur von etwa \SI{170}{\kelvin} erreicht ist,
    wird der Strom an der Heizspule der Probe auf \SI{180}{\milli\ampere} erhöht.
    Diese Messung wird solange durchgeführt,
    bis eine Temperatur von \SI{300}{\kelvin},
    also ein Widerstand von etwa \SI{111}{\ohm},
    erreicht ist.

