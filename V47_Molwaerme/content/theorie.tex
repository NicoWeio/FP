\section{Theorie}
\label{sec:theorie}

    Die Wärmekapazität eines Festkörpers gibt die Wärmemenge an,
    die benötigt wird um die Temperatur des Festkörpers um \SI{1}{\kelvin} zu erhöhen.
    Allgemein kann diese Wärmekapazität mithilfe von
    \begin{equation*}
        C = \frac{\symup{\Delta Q}}{\symup{\Delta} T}
    \end{equation*}
    berechnet werden.
    Außerdem wird zwischen der Wärmekapazität bei konstantem Volumen
    \begin{equation}
        C_V = \frac{\partial Q}{\partial T} \Bigl|_V
    \end{equation}
    und der Wärmekapazität mit konstantem Druck
    \begin{equation}
        C_p = \frac{\partial Q}{\partial T} \Bigl|_p
    \end{equation}
    unterschieden.
