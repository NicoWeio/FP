\section{Theorie}
\label{sec:theorie}

    Die Wärmekapazität eines Festkörpers bezeichnet die Wärmemenge $\symup{\Delta}$,
    die bnötigt wird Temperatur $T$ dieses Festkörpers um \SI{1}{\kelvin} zu erhöhen.
    Es gilt
    \begin{equation*}
        C = \frac{\symup{\Delta}Q}{\symup{\Delta}T} \ .
    \end{equation*}
    Die Wärmekapazität ist dabei material-spezifisch und wird deshalb oft in Bezug auf eine Stoffmenge,
    ein Volumen oder eine Masse angegeben.
    Dies wird als \textit{Spezifische Wärmekapazität} bezeichnet.
    Diese ergibt sich nach \cite{grossmarx} entsprechend zu
    \begin{align}
        c^\text{mol} = \frac{C}{mol} && c^\text{V} = \frac{C}{V} && c^\text{mass} = \frac{C}{m} \ .
    \end{align}
    Mithilfe des ersten Hautptsatzes der Thermodynamik
    \begin{equation}
        \symup{d}Q = \symup{d}U - \symup{d}W = \symup{d}U + p\symup{d}V
        \label{eqn:theorie:hauptsatz}
    \end{equation}
    kann zwischen der Wärmekapazität bei konstantem Volumen $C_V$ und konstantem Druck $C_p$ unterschieden werden.
    Hierbei gilt
    \begin{align}
        C_V &= \frac{\partial Q}{\partial T} \Bigl|_V = \frac{\partial U}{\partial T} \Bigl|_V
        \label{eqn:theorie:c_vconst} \\
        C_p &= \frac{\partial Q}{\partial T} \Bigl|_p = \frac{\partial U}{\partial T} \Bigl|_p
        \label{eqn:theorie:c_pconst} \ .
    \end{align}
    Weiterhin ergibt sich die Differenz
    \begin{equation}
        C_p - C_V = T \alpha^2_V B V = 9 \alpha^2 \kappa V_0 T
        \label{eqn:theorie:c_differenz}
    \end{equation}
    mit dem Ausdehnungskoeffizenten $\alpha$,
    dem Bulkmodul $B$ und dem Kompressionsmodul $\kappa$.
    Diese Differenz ist im Allgemeinen $\geq 0$,
    da $C_p \geq C_V$ ist,
    was daran liegt,
    dass nach \autoref{eqn:theorie:hauptsatz} bei konstantem Druck mehr Energie auf die Ausdehnung des Volumens aufgewendet werden muss,
    sodass mehr Wärme benötigt wird,
    um die gleiche Temperatur zu erreichen.
    Aus diesem Grund ist die Differenz \autoref{eqn:theorie:c_differenz} im Falle eines Festkörpers geringer als bei einem idealen Gas,
    da sich der Festkörper weniger stark ausdehnt,
    wenn er erwärmt wird.

    Im Folgenden wird zwischen klassischer und quantenmechanischer Betrachtung unterschieden.
    Zusätzlich werden die Schwingungsmoden im Festkörper als harmonische Oszillatoren behandelt.

    In der klassischen Betrachtung kann ein System,
    welches in Kontakt mit einem Wärmebad ist,
    bei jeder Temperatur angeregt werden,
    da der klassische harmonische Oszillator ein kontinuierliches Energiespektrum besitzt.
    Jede Schwingungsmode hat dabei eine mittlere Energie von $\sfrac{1}{2} k_\text{B} T$.
    Ein dreidimensionales System mit $N$ Atomen,
    drei Ortsfreiheitsgraden und drei Impulsfreiheitsgraden hat demnach eine Energie von $3Nk_\text{B}T$ und demnach eine Wärmekapazität von $3Nk_\text{B}$.
    Dieser Zusammenhang wird als \textit{Dulong-Petit-Gesetz} bezeichnet und stellt den Hochtemperaturgrenzfall der Wärmekapazität dar.













































    %Die Wärmekapazität eines Festkörpers gibt die Wärmemenge an,
    %die benötigt wird um die Temperatur des Festkörpers um \SI{1}{\kelvin} zu erhöhen.
    %Allgemein kann diese Wärmekapazität mithilfe von
    %\begin{equation*}
    %    C = \frac{\symup{\Delta Q}}{\symup{\Delta} T}
    %\end{equation*}
    %berechnet werden.
    %Außerdem wird zwischen der Wärmekapazität bei konstantem Volumen $C_V$ und bei konstantem Druck $C_p$
    %%\begin{equation}
    %%    C_V = \frac{\partial Q}{\partial T} \Bigl|_V = \frac{\partial U}{\partial T} \Bigl|_V
    %%    \label{eqn:theorie:volumen}
    %%\end{equation}
    %%und der Wärmekapazität mit konstantem Druck
    %%\begin{equation}
    %%    C_p = \frac{\partial Q}{\partial T} \Bigl|_p = \frac{\partial U}{\partial T} \Bigl|_p
    %%\end{equation}
    %unterschieden,
    %wobei der 1. Hautptsatz der Thermodynamik $\symup{d}Q = \symup{d}U - \symup{d}W = \symup{d}U + p \symup{d}V$ verwendet wird.
    %Demnach gilt
    %\begin{align}
    %    C_V = \frac{\partial Q}{\partial T} \Bigl|_V = \frac{\partial U}{\partial T} \Bigl|_V & \
    %    C_p = \frac{\partial Q}{\partial T} \Bigl|_p = \frac{\partial U}{\partial T} \Bigl|_p \ .
    %\end{align}
    %Da die Wärmekapazität Stoff- und Mengenabhängig ist,
    %wird oft die spezielle Wärmekapazität verwendet
    %\begin{equation*}
    %    c = \frac{C}{m} \ .
    %\end{equation*}
    %Zwischen $C_V$ und $C_p$ gilt der Zusammenhang
    %\begin{equation}
    %    C_p - C_V = T \alpha^2_V B V  = 9 \alpha^2 \kappa V_0 T \ ,
    %    \label{eqn:theorie:diff_waermekapazitaet}
    %\end{equation}
    %wobei $\alpha_V$ den Ausdehnungskoeffizenten und $B$ den Bulk-Modul bezeichnet.
    %Die Differenz in \autoref{eqn:theorie} ist im Falle eines Festkörpers sehr viel geringer als bei einem idealen Gas,
    %da

    %Es wird nun zusätzlich zwischen der klassischen und quantenmechanischen Betrachtung eines Systems in Kontakt mit einem Wärembad unterschieden.
    %In der klassischen Betrachtung besitzen die Schwingungsmoden eine Energie von $\sfrac{1}{2} k_\text{B} T$.
    %Mit $N$ Atomen mit jeweils drei Orts- und drei Impulsfreiheitsgraden ergibt sich eine innere Energie $U = 3Nk_\text{B}T$.
    %Nach \autoref{eqn:theorie:volumen} gilt für die Wärmekapazität bei konstantem Volumen
    %\begin{equation}
    %    C_V = 3 N k_\text{B} \ .
    %\end{equation}
    %Diese Beziehung wird auch als \textit{Dulong-Petit-Gesetz} bezeichnet.
    %Aufgrund des Kontakts mit dem Wärmebad findet eine Anregung der Atome statt,
    %wobei hier der harmonische Oszillator angenommen wird.
    %Im klassischen System kann das System für jede Temperatur $T$ angeregt werden.
