\section{Auswertung}
\label{sec:auswertung}

\subsection{Temperaturabhängige Wärmekapazität}
Um für jedes Messintervall die Wärmekapazität $c_p$ zu bestimmen,
wird die über die Heizspule an die Probe übertragene Wärmeenergie
 gemäß TODO-REF
ins Verhältnis zu der gemessenen Erhöhung der Temperatur gesetzt.
Die Wärmeenergie ergibt sich dabei aus $\symup{\Delta}E = U I \symup{\Delta}t$.
Die Temperatur wird über die Messung des Pt100-Widerstand gemäß \autoref{eqn:durchfuehrung:pt100} ermittelt.
% , sodass die Temperaturdifferenz…
Aus der Probenmasse $m = \SI{342}{\gram}$ \cite{versuchsanleitung} und der molaren Masse $M = TODO$ von Kupfer
folgt für die Stoffmenge $n = \sfrac{m}{M} TODO$.

Mittels \autoref{eqn:theorie:c_differenz} kann schließlich von $c_p$ auf $c_V$ geschlossen werden.
Dazu werden
der Ausdehnungskoeffizent aus \cite[Tabelle 2]{versuchsanleitung},
der Kompressionsmodul $\kappa = TODO$ und
das Molvolumen $V_0 = TODO$ eingesetzt.

In \autoref{fig:auswertung:c_v} sind beide Wärmekapazitäten gegen $T$ aufgetragen.
Die zugrundeliegenden Messwerte sind in \autoref{tab:mess},
die daraus berechneten Wärmekapazitäten in \autoref{tab:warmekapazitaeten} aufgeführt.

\begin{figure}[H]
    \centering
    \includegraphics[width=\textwidth]{build/plt/c_v.pdf}
    \caption{Gemessene Wärmekapazitäten $C_p$ und daraus berechnete $C_V$ in Abhängigkeit der Temperatur.}
    \label{fig:auswertung:c_v}
\end{figure}

\begin{table}
    \centering
    \caption{Messwerte.}
    \label{tab:mess}
    \expandableinput{build/tab/mess.tex}
\end{table}

\begin{table}
    \centering
    \caption{Zeitdifferenzen und Wärmekapazitäten.}
    \label{tab:warmekapazitaeten}
    \expandableinput{build/tab/warmekapazitaeten.tex}
\end{table}


\subsection{Experimentelle Bestimmung der Debye-Temperatur}
Unter Zuhilfenahme der in \cite[Tabelle 1]{versuchsanleitung}
in der Form $f: \frac{\Theta_\text{D}}{T} \longrightarrow C_V$
angegebenen Debye-Funktion
kann die Debye-Temperatur bestimmt werden.
Lineare Interpolation ermöglicht es, näherungsweise eine Umkehrfunktion
$f^{-1}: C_V \longrightarrow \frac{\Theta_\text{D}}{T}$
anzugeben.
Somit ist
\[
    \Theta_\text{D} = f^{-1}(C_V) · T
\]
die Bestimmungsgleichung, um mit $T$ und $C_V$ aus \autoref{tab:warmekapazitaeten} die Debye-Temperatur zu bestimmen.


\subsection{TODO Integral Forderung Dings}
\begin{equation*}
    \omega_\text{D}
    = \sqrt[3]{
        \frac{18 \pi^2 N_\text{A}}{V_0} · \frac{1}{\frac{2}{v_\text{trans}} + \frac{1}{v_\text{long}}}
    }
\end{equation*}
