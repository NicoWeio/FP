\section{Auswertung}
\label{sec:auswertung}

\subsection{Temperaturabhängige Wärmekapazität}
Um für jedes Messintervall die molare Wärmekapazität $c_p$ zu bestimmen,
wird die über die Heizspule auf die Probe übertragene Wärmeenergie
gemäß \autoref{eqn:theorie:c_p}
ins Verhältnis zu der gemessenen Erhöhung der Temperatur gesetzt
und auf die Stoffmenge normiert.
Die Wärmeenergie ergibt sich dabei aus $\symup{\Delta}E = U I \symup{\Delta}t$.
Die Temperatur wird über die Messung des Pt-100-Widerstandes gemäß \autoref{eqn:durchfuehrung:pt100} ermittelt.
% , sodass die Temperaturdifferenz…
Aus der Probenmasse $m = \SI{342}{\gram}$ \cite{versuchsanleitung}
und der molaren Masse von Kupfer $M = \SI{63.546e-3}{\kilogram\per\mol}$ \cite{periodictable}
folgt für die Stoffmenge $n = \frac{m}{M} = \SI{5.38}{\mol}$.

Mittels \autoref{eqn:theorie:c_differenz} kann schließlich von $c_p$ auf $c_V$ geschlossen werden.
Dazu werden
der (linear interpolierte) Ausdehnungskoeffizent aus \cite[Tabelle 2]{versuchsanleitung},
der Kompressionsmodul $K = \SI{140}{\giga\pascal}$ \cite{periodictable} und
das Molvolumen $V_0 = \SI{7.0922e-6}{\cubic\meter\per\mol}$ \cite{periodictable} eingesetzt.

In \autoref{fig:auswertung:waermekapazitaeten} sind beide Wärmekapazitäten gegen $T$ aufgetragen.
Die zugrundeliegenden Messwerte sind in \autoref{tab:mess},
die daraus berechneten Wärmekapazitäten in \autoref{tab:warmekapazitaeten} aufgeführt.

\begin{figure}[H]
    \centering
    \includegraphics[width=\textwidth]{build/plt/waermekapazitaeten.pdf}
    \caption{Gemessene Wärmekapazitäten $c_p$ und daraus berechnete $c_V$ in Abhängigkeit der Temperatur.}
    \label{fig:auswertung:waermekapazitaeten}
\end{figure}

\begin{table}
    \centering
    \caption{Messwerte zur Heizspule der Probe sowie Widerstände und daraus berechnete Temperaturen. Unsicherheiten siehe \autoref{sec:auswertung}}
    \label{tab:mess}
    \expandableinput{build/tab/mess.tex}
\end{table}

\begin{table}
    \centering
    \caption{Aus \autoref{tab:mess} berechnete spezifische Wärmekapazitäten in Abhängigkeit der Temperatur(differenz).}
    \label{tab:warmekapazitaeten}
    \expandableinput{build/tab/warmekapazitaeten.tex}
\end{table}


\subsection{Experimentelle Bestimmung der Debye-Temperatur}
\label{sec:auswertung:debye_exp}
Unter Zuhilfenahme der in \cite[Tabelle 1]{versuchsanleitung}
in der Form $f: \frac{\Theta_\text{D}}{T} \longrightarrow c_V$
angegebenen Debye-Funktion
kann die Debye-Temperatur bestimmt werden.
Lineare Interpolation ermöglicht es, näherungsweise eine Umkehrfunktion
$f^{-1}: c_V \longrightarrow \frac{\Theta_\text{D}}{T}$
anzugeben.
Somit ist
\[
    \Theta_\text{D} = f^{-1}(c_V) · T
\]
die Bestimmungsgleichung, um mit $T$ und $c_V$ aus \autoref{tab:warmekapazitaeten} die Debye-Temperatur zu bestimmen.

Bei Berücksichtigung der Messwerte mit $T \leq \SI{170}{\kelvin}$ (gestrichelte Linie in \autoref{fig:auswertung:waermekapazitaeten})
ergibt sich
\[
    \Theta_\text{D} = \SI{317.81}{\kelvin} \ .
\]


\subsection{Bestimmung der Debye-Temperatur aus der Schallgeschwindigkeit}
\label{sec:auswertung:debye_vs}
Aus der Forderung in \autoref{eqn:theorie:debye_int}
kann bei bekannter Schallgeschwindigkeit $v_\text{s}$ die Debye-Temperatur $T_\text{D}$ bestimmt werden.

Zunächst wird die Debye-Frequenz berechnet,
indem die Schallgeschwindigkeit aus \autoref{eqn:theorie:schallgeschwindigkeit}
in \autoref{eqn:theorie:omega_D} eingesetzt wird:
\begin{align*}
    \omega_\text{D} &= \underbrace{\left( \frac{3}{\sum_{i=1}^3 \frac{1}{v^3_i}} \right)^\frac{1}{3}}_{v_\text{s}}
    \cdot \left(6 \pi^2 \frac{N_\text{A}}{V_0}\right)^{\frac{1}{3}}
    \\
    &= \left( \frac{18 \pi^2 \frac{N_\text{A}}{V_0}}{\frac{2}{v_\text{trans}} + \frac{1}{v_\text{long}}} \right)^\frac{1}{3} \ .
\end{align*}
Dabei ist $N_\text{A}$ die Avogadro-Konstante und $V_0 = \SI{7.0922e-6}{\cubic\meter\per\mol}$ \cite{periodictable} wie zuvor das Molvolumen.

Die Schallgeschwindigkeiten je Richtung sind gegeben \cite{versuchsanleitung} als
\begin{align*}
    v_\text{long} &= \SI{4.7}{\kilo\meter\per\second}
    % \intertext{und}
    &
    v_\text{trans} &= \SI{2.26}{\kilo\meter\per\second} \ .
\end{align*}

Schließlich folgt mit \autoref{eqn:theorie:Theta_D}
\[
    \Theta_\text{D} = \SI{332.49}{\kelvin} \ .
\]
