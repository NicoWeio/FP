\section{Diskussion}
\label{sec:diskussion}

\subsection{Abweichungen}

    Der Literaturwert für die Debye-Temperatur lautet $\Theta_\text{D} = \SI{343}{\kelvin}$ \cite[Abbildung 6.9]{grossmarx}.

    Der hier \hyperref[sec:auswertung:debye_exp]{experimentell bestimmte} Wert beträgt \SI{317.81(6)}{\kelvin}.
    Das entspricht einer relativen Abweichung von \SI{7.35}{\percent}.

    Wird die Debye-Temperatur stattdessen \hyperref[sec:auswertung:debye_vs]{aus der Schallgeschwindigkeit bestimmt},
    ergibt sich \SI{332.49}{\kelvin}.
    Sie ist damit um \SI{14.68}{\kelvin} (oder \SI{4.62}{\percent}) größer als die experimentell bestimmte.


\subsection{Mögliche Fehlerquellen}

    % ↓ Die Kurve sieht aber gut aus. Ich lasse das mal weg…
    % Zu Beginn der Messreihe musste schnell die Zeit gestoppt werden und gleichzeitig die Messwerte abgelesen werden.
    % Hier ergab sich deshalb eine Verzögerung der Zeit,
    % sodass die erste Messung wahrscheinlich nicht zwei Minuten lang war.

    Die wesentliche Voraussetzung für den Erfolg des Experiments,
    dass die Probe einzig durch die Spule erhitzt und nicht abgekühlt wird,
    ist in der Realität nur näherungsweise erreichbar.
    Gründe hierfür sind unter anderem
    das nicht perfekte Vakuum (→ Konvektion),
    die Aufhängung der Probe (→ Wärmeleitung), % vermutlich vernachlässigbar, aber trotzdem :P % Oxford-Komma
    und die Wärmestrahlung zwischen der Probe und dem umliegenden Kupfer-Gefäß,
    welche nicht optimal kompensiert ist,
    wenn nicht beide gleich warm sind.

    Mithilfe einer zweiten Spannungsquelle wurde der Strom der Heizspule um das Gefäß geregelt.
    Der zugehörige Pt-100-Widerstand reagierte jedoch sehr träge,
    sodass eine Änderung erst nach einigen Minuten eintrat.
    Auf diese Weise war es schwierig,
    den Widerstand von Gefäß und Probe auf dem gleichen Wert zu halten.
    Im Durchschnitt betrug die Temperaturdifferenz zwischen Probe und Zylinder \SI{2.34}{\kelvin},
    maximal lag sie bei \SI{12.51}{\kelvin}.

    Das Ablesen von vier Werten je Messintervall erfolgte naturgemäß nicht genau gleichzeitig.
    Zusätzlich führten kleine Fluktuationen in den digitalen Anzeigen
    zu ebenso kleinen zusätzlichen Unsicherheiten in den Messwerten.

    Eine weitere Unsicherheit liegt darin,
    dass sich der Wert des Konstantstroms mit der Zeit auch leicht veränderte,
    sodass der tatsächliche Heizstrom der Probe
    auch innerhalb eines Messintervalls nicht vollständig konstant war.
