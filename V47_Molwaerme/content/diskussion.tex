\section{Diskussion}
\label{sec:diskussion}

\subsection{Abweichungen}

    Der Literaturwert für die Debye-Temperatur lautet $\theta_\text{D} = \SI{343}{\kelvin}$ \cite[Abbildung 6.9]{grossmarx}.


\subsection{Mögliche Fehlerquellen}

    Zu Beginn der Messreihe musste schnell die Zeit gestoppt werden und gleichzeitig die Messwerte abgelesen werden.
    Hier ergab sich deshalb eine Verzögerung der Zeit,
    sodass die erste Messung wahrscheinlich nicht zwei Minuten lang war.

    Mithilfe einer zweiten Spannungsquelle wurde der Strom der Heizspule um das Gefäß geregelt.
    Diese war jedoch sehr träge,
    sodass eine Änderung erst nach einigen Minuten eintrat.
    Auf diese Weise war es schwierig,
    den Widerstand von Gefäß und Probe auf dem gleichen Wert zu halten.
    Ein weiterer Faktor liegt darin,
    dass sich der Wert des Konstantstroms mit der Zeit auch leicht veränderte,
    sodass tatsächlich der Heizstrom der Probe nicht vollständig konstant war.

    Bezüglich des Aufbaus ist es gut möglich,
    dass nicht alle weiteren Effekte des Wärmetransportes vollständig ausgeschlossen werden konnten,
    und so Wärmeverluste auftreten.
    Dies kann auch durch eine nicht vollständige Isolierung passieren.
