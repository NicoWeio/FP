\section{Theorie}
\label{sec:theorie}

    Im folgenden Abschnitt werden die theoretischen Grundlagen für den Versuch erläutert.

\subsection{Energieniveaus und Quantenzahlen eines Atoms}

    Bei der Betrachtung eines Atoms ist es sinnvoll,
    zwischen Kern und Atomhülle zu unterscheiden.

    Die Atomhülle besitzt einen Gesamtdrehimpuls $\vec{J}$,
    welcher sich aus dem Bahndrehimpuls $\vec{L}$ und dem Spin $\vec{S}$ der sich in der Hülle befindenden Elektronen zusammensetzt.
    Zu jedem dieser Drehimpulse gehört außerdem ein magnetisches Moment,
    welches durch
    \begin{gather}
        \vec{\mu}_J = -g_J \mu_\text{B} \vec{J} \\
        \vec{\mu}_L = -g_L \mu_\text{B} \vec{L} \\
        \vec{\mu}_S = -g_S \mu_\text{B} \vec{S}
    \end{gather}
    beschrieben werden kann,
    wobei $g_L = 1$ und $g_S \approx 2$ ist.
    Der Faktor $\mu_\text{B} = \frac{e \hbar}{2m_\text{e}}$ bezeichnet das Bohr'sche Magneton und $g_J$ stellt den Landé-Faktor der Elektronenhülle dar,
    welcher mit
    \begin{equation}
        g_J = 1 + \frac{J(J+1) + S(S+1) - L(L+1)}{2J(J+1)}
        \label{eqn:landeJ}
    \end{equation}
    berechnet werden kann.

    Die Elektronenkonfiguration innerhalb der Atomhülle kann über die Quantenzahlen $n, l, m$ beschrieben werden,
    wobei $n \in \symbb{N}$ die Energie- oder Hauptquantenzahl,
    $l \in \{0, 1, 2, ..., n-1\}$ die Bahndrehimpulsquantenzahl und $m \in \{-l, -l+1, ..., l-1, l\}$ die magnetische Quantenzahl ist.
    Das hier betrachtete Rubidium hat eine Elektronenkonfiguration von $1s^2 2s^2 2p^6 3s^2 3p^6 3d^{10} 4s^2 4p^6 5s$.
    Mithilfe der Hund'schen Regeln kann ein Gesamtspin von $S=\frac{1}{2}$,
    ein Gesamtbahndrehimpuls $L=0$ und ein Gesamtdrehimpuls von $J = \lvert L \pm S \rvert = \frac{1}{2}$ für die Atomhülle berechnet werden.

    Aufgrund der Wechselwirkung der magnetischen Momente von Spin $\mu_s$ und Bahndrehimpuls $\mu_l$ mit dem Magnetfeld des Kerns entsteht eine Aufspaltung der Energieniveaus,
    welche als \textit{Feinstrukturaufspaltung} oder \textit{LS-Kopplung} bezeichnet wird.

    Der Kern hat den Spin $\vec{I}$.
    Im Falle von $\ce{^{85}Rb}$ ergibt sich ein Spin von $I = \frac{5}{2}$ und für $\ce{^{87}Rb}$ gilt $I = \frac{3}{2}$.
    Der Spin lässt sich mithilfe der Spinkonfiguration von Protonen und Neutronen bestimmen.
    Mit Betrachtung des Kernspins tritt ein weiterer Effekt bei der Aufspaltung der Energieniveaus auf,
    die sogenannte \textit{Hyperfeinstruktur},
    welche aus der Wechselwirkung der magnetischen Momente des Kernspins $\mu_I$ und dem des Gesamtbahndrehimpuls der Atomhülle $\mu_J$ resultiert.

    Für das gesamte Atom ergibt sich ein Atomdrehimpuls $\vec{F} = \vec{I} + \vec{J}$ mit der magnetischen Quantenzahl $M = \{-\lvert I+J \rvert, ..., \lvert I+J \rvert\}$.
    Für das entsprechende magnetische Moment gilt
    \begin{equation}
        \vec{\mu_F} = - M g_F \mu_\text{B} \vec{F} \ ,
    \end{equation}
    wobei
    \begin{equation}
        g_F = g_J \cdot \frac{F(F+1) + J(J+1) - I(I+1)}{2F(F+1)}
        \label{eqn:landeF}
    \end{equation}
    der Landé-Faktor des Atoms ist,
    mit $g_J$ aus \autoref{eqn:landeJ}.

    Wird nun ein externes Magnetfeld angelegt,
    wird der \textit{Zeeman-Effekt} relevant.
    Es findet eine Aufspaltung der Energienievaus nach der magnetischen Quantenzahl $M$ statt,
    wodurch $2F+1$ Zustände entstehen.
    Die Energiedifferenz zwischen diesen Zuständen wird mithilfe von
    \begin{equation}
        \symup{\Delta} E_\text{Z} = g_F \mu_\text{B} B \symup{\Delta} m_J
        \label{eqn:zeemanaufspaltung}
    \end{equation}
    berechnet.
    Die \autoref{fig:aufspaltung} zeigt die Feinstrukturaufspaltung,
    die Hyperfeinstrukturaufspaltung sowie die Aufspaltung durch den Zeeman-Effekt im Falle des Grundzustands $^2S_{1/2}$ und angeregten Zustands $^2P_{1/2}$.
    %\begin{figure}
    %    \centering
    %    \includegraphics[width=\textwidth]{}
    %    \caption{Aufspaltung der Energieniveaus durch Kopplungen innerhalb des Atoms und externe Magnetfelder.}
    %    \label{fig:aufspaltung}
    %\end{figure}
    Zwischen den Energieniveaus können Übergänge stattfinden,
    wobei ein Elektron aus einem höhreren Niveau in ein niedrigeres zurückfällt und dabei ein Photon mit der entsprechenden Energiedifferenz emittiert.
    Es können nur Übergänge stattfinden,
    bei denen die Auswahlregeln für elektrische und magnetische Dipolübergänge erfüllt sind.
    Es gilt $\symup{\Delta} L = \pm 1$, $\symup{\Delta} S = 0$, $\symup{\Delta} J = 0, \pm 1$ und $\symup{\Delta} F = 0, \pm 1$.
    Entsprechend ergibt sich $\symup{\Delta} M = 0, \pm 1$.
    Abhängig davon,
    welchen Wert $\symup{\Delta} M$ bei einem Übergang annimmt,
    ist das entstehende Photon entweder linear ($\pi$-) polarisiert ($\symup{\Delta}M = 0$) oder zirkular ($\sigma^{\pm}$-) polarisiert ($\symup{\Delta}M = \pm 1$).
    Umgekehrt kann also nur ein $\sigma^{\pm}$-Übergang mit $\symup{\Delta} M = \pm 1$ stattfinden,
    wenn das Atom mit zirkular polarisiertem Licht bestrahlt wird.

    Es wird nun zusätzlich zwischen \textit{Spontaner} und \textit{Stimulierter Emission} unterschieden.
    Bei der Spontanen Emission wird ein Photon beim Rückfall eines Atoms aus einem angeregten Zustand in den Grundzustand erzeugt.
    Bei der Stimulierten Emission trifft ein Photon auf ein angeregtes Atom,
    welches daraufhin unter Aussendung von Photonen in den Grundzustan zurückfällt.
    Das einfallende und die emittierten Photonen sind dabei kohärent,
    haben also die gleiche Wellenlänge, Polarisation, Frequenz, Richtung und Phasenbeziehung.


