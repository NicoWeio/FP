\section{Durchführung}
\label{sec:durchfuehrung}

    Im Folgenden sollen nun der Aufbau des Experiments,
    sowie die Durchführung beschrieben werden.

\subsection{Versuchsaufbau}

    Der Gesamtaufbau besteht zum einen aus einer Rubidium-Spektrallampe,
    welche Licht auf einen Kolben mit Rubidium-Gas strahlt,
    wobei der Kolben von Helmholtzspulen umgeben ist,
    welche verschiedene Magnetfelder erzeugen.
    Das entstehende emittierte Licht wird von einer Photozelle detektiert.
    \autoref{fig:aufbau} zeigt den Aufbau der Apparatur.
    \begin{figure}
        \centering
        \includegraphics[width=\textwidth]{content/img/Abb_1.pdf}
        \caption{Versuchsaufbau mit Spektrallampe, optischen Elementen, Gaskolben und Photoelement. \cite{versuchsanleitung}}
        \label{fig:aufbau}
    \end{figure}

    Wie in \autoref{sec:optisches_pumpen} erläutert,
    soll nun rechts polarisiertes Licht mit der Wellenlänge des $D_1$-Übergangs verwendet werden.
    Dazu sind optische Elemente wie Linsen,
    Polarisationsfilter und beschichtete Plättchen gegeben.
    Mithilfe von plan-konkaven Sammellinsen soll der entstehende Lichtstrahl fokussiert werden.
    Da die Rubidium-Lampe nicht nur die $D_1$-Linie erzeugt,
    werden alle anderen mithilfe eines beschichteten Plättchens herausgefiltert.
    Anschließend wird der Lichtstrahl mithilfe eines Polarisationsfilters erst linear,
    dann mithilfe eines $\sfrac{\lambda}{4}$-Plättchens zirkular polarisiert.

    Mithilfe eines Ofens wird das Rubidium in der Dampfzelle auf einem optimalen Druck gehalten.

    Es wird zwischen verschiedenen Spulen unterschieden,
    welche sich um die Dampfzelle herum befinden;
    eine vertikale Spule und eine horizontale Spule,
    wobei letztere mit einer weiteren Sweep-Spule umwickelt ist.
    Mithilfe der Sweep-Spule können in einer festen Zeit periodisch verschiedene Magnetfeldstärken durchlaufen werden.
    Das Magnetfeld dieser Spulen ist über den Stromfluss regelbar.
    Zusätzlich ist eine Modulationsspule eingebaut,
    welche ein radio-frequentiges Magnetfeld erzeugt,
    dessen Stärke über die Frequenz verstellbar ist.

    Mithilfe eines Oszilloskops kann die registrierte Intensität dargestellt werden.

\subsection{Einjustierung und Kompensation des Erdmagnetfelds}

    Zu Beginn müssen die optischen Elemente so in den Strahlengang gesetzt werden,
    dass die gemessene Intensität maximal wird.
    Es werden dazu erst die beiden Linsen nacheinander in horizontaler und vertikaler Richtung ausgerichtet und anschließend die verschiedenen Filter eingebaut.

    Auf dem Oszilloskop ist nun ein breiter Peak zu sehen.
    Als nächstes muss die Vertikal-Komponente des Erdmagnetfelds kompensiert werden.
    Dazu wird der Stromfluss durch die Vertikalspule so erhöht,
    dass der Peak möglichst schmal wird.
    Zusätzlich wird der Tisch,
    auf dem die Apparatur steht,
    so gedreht,
    dass der Strahlengang in Nord-Süd-Richtung ausgerichtet ist.
    Auch hier soll der Peak möglichst schmal werden.

\subsection{Messung der Magnetfeldstärke der Rubidium-Isotope}

    Für diese Messung wird das erste Mal ein Strom durch die Modulationsspule gegeben,
    welcher sinusförmig ist und eine Frequenz $f$ hat.
    Die Frequenz wird in der folgenden Messung von \SI{100}{\kilo\hertz} an in Schritten von \SI{100}{\kilo\hertz} bis auf \SI{100}{\mega\hertz} erhöht.

    Beim Anlegen einer Frequenz von \SI{100}{\kilo\hertz} ergibt sich eine Darstellung wie in \autoref{fig:peaks},
    welche während der Messung aufgenommen wurde.
    %\begin{figure}
    %    \centering
    %    \includegraphics[width=\textwidth]{}
    %    \caption{}
    %    \label{fig:peaks}
    %\end{figure}

    Der größte Peak entspricht $B = 0$,
    dort beginnt der Sweep.
    Die beiden kleineren Peaks rechts davon stellen die Rubidium-Isotope dar,
    welche aufgrund des Sweeps symmetrisch um den Null-Peak sind.
    Es ist ausreichend nur auf einer Seite zu messen.

    Das Oszilloskop wird nun so eingestellt,
    dass mit Erhöhen der Stromstärke in der horizontalen Sweep-Spule der Verlauf der Kurve abgelaufen werden kann,
    wobei jeweils der Stromwert notiert wird,
    für den die Spitze eines der Isotopen-Peaks erreicht wird.
    Sollte die Feldstärke der Sweep-Spule nicht ausreichen,
    kann zur Darstellung der Peaks die Stromstärke in der horizontalen Spule erhöht werden.
    Der Gesamtwert der entsprechenden Stromstärke ergibt sich aus Summation der Werte aus beiden Spulen.
    Diese Messung wird für jede Frequenz aus dem oben genannten Bereich wiederholt.

    Zuletzt wird für eine Frequenz von \SI{100}{\kilo\hertz} die Amplitude der beiden Isotopen-Peaks auf dem Oszilloskop vermessen.
