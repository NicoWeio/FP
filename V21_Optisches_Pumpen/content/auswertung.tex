\section{Auswertung}
\label{sec:auswertung}

\subsection{Vertikales Erdmagnetfeld}

Die magnetische Flussdichte $B$ im Zentrum eines Helmholtz-Spulenpaars ist durch
\begin{equation}
    B = \frac{8}{\sqrt{125}} \frac{\mu_0 NI}{R}
    \label{eqn:helmholtz}
\end{equation}
gegeben,
wobei $N$ die Windungszahl, $I$ den Spulenstrom, und $R$ den Radius der beiden Spulen bezeichnet. % Oxford-Komma

\begin{table}
    \centering
    \caption{Daten zu den verwendeten Helmholtz-Spulen. \cite{versuchsanleitung}}
    \label{tab:spulen}
    \begin{tabular}{l S S}
        \toprule
        Spule &
        {$N$} &
        {$R \mathbin{/} \si{\centi\meter}$} \\
        \midrule
        Vertikal   &  20 & 11.735 \\
        Horizontal & 154 & 15.79  \\
        Sweep      &  11 & 16.39  \\
        % Die RF-Spule gibt es auch noch, aber wir haben und brauchen dazu keine Daten.
        \bottomrule
    \end{tabular}
\end{table}

Es wurde experimentell bestimmt,
dass der Effekt des Erdmagnetfelds minimal ist,
der Peak also am schmalsten ist,
wenn die Vertikal-Spule mit einem Strom von ${ I = \SI{0.23(1)}{\ampere} }$ betrieben wird.
Mit den Spulen-Parametern aus \autoref{tab:spulen} und \autoref{eqn:helmholtz}
folgt daraus für die vertikale Komponente des Erdmagnetfelds
\[ B_\text{vert} = \SI{35.25(15)}{\micro\tesla} \ . \]
% Vergleichswert: 44 µT [Wikipedia]


\subsection{Gyromagnetische Faktoren und horizontales Erdmagnetfeld}
\label{sec:auswertung:g_F}
\def\coloralpha{{\color{blue} \alpha}}
\def\colorbeta{{\color{blue} \beta}}

Aus den gemäß \autoref{sec:durchfuehrung:messung} gemessenen Resonanzpositionen
können die gyromagnetischen Faktoren von \Rb{87} und \Rb{85} berechnet werden.


Indem die magnetische Flussdichte aus \autoref{eqn:zeemanaufspaltung}
in Abhängigkeit der RF-Frequenz, also als $B(f)$, ausgedrückt wird,
wobei $\symup{\Delta} M = 1$, $\symup{\Delta} E_\text{Z} = hf$ und ein konstanter Summand $\colorbeta$ eingesetzt werden,
ergibt sich ein linearer Zusammenhang
\begin{align*}
    \symup{\Delta} E_\text{Z} &= g_F \mu_\text{B} B \symup{\Delta} M
    \tag{\ref{eqn:zeemanaufspaltung}} \\
    % \Leftrightarrow
    % B &= \frac{\symup{\Delta} E_\text{Z}}{g_F \mu_\text{B} \symup{\Delta} M}
    \Rightarrow
    B &= \underbrace{\frac{h}{g_F \mu_\text{B}}}_{\coloralpha} · f
    + \colorbeta \ ,
\end{align*}
welcher mithilfe einer Regressionsrechnung
(unter Verwendung von \texttt{scipy.stats.linregress})
quantifiziert wird.
Dazu werden die Messwerte aus \autoref{tab:mess} verwendet.

Es ergeben sich die Parameter
\begin{align*}
    {\color{blue} \alpha_{\Rb{87}}} &= \SI{0.1420 \pm 0.0006}{\micro\tesla\per\kilo\hertz} \\
    {\color{blue} \beta_{\Rb{87}}} &= \SI{25.7 \pm 0.4}{\micro\tesla} \\
    \intertext{und}
    {\color{blue} \alpha_{\Rb{85}}} &= \SI{0.206 \pm 0.004}{\micro\tesla\per\kilo\hertz} \\
    {\color{blue} \beta_{\Rb{85}}} &= \SI{28.2 \pm 2.3}{\micro\tesla} \ ;
\end{align*}
in \autoref{fig:plt:g_F} sind Messwerte und Regressionsgeraden visualisiert.

Aus
\[
    \coloralpha = \frac{h}{g_F \mu_B}
    \Leftrightarrow
    g_F = \frac{h}{\coloralpha \mu_B}
    = \frac{4 \pi m_e}{\coloralpha e}
\]
folgt für die gyromagnetischen Faktoren
\begin{align*}
    g_{F, \Rb{87}} &= \num{0.503 \pm 0.002} \\
    g_{F, \Rb{85}} &= \num{0.347 \pm 0.006} \ .
\end{align*}


Der eingeführte Parameter $\color{blue} \beta$ steht
(unter Vernachlässigung möglicher systematischer Fehler)
für die horizontale Komponente des Erdmagnetfelds.
Im arithmetischen Mittel ist
\[
    B_\text{erd, hor}
    = \frac{{\color{blue} \beta_{\Rb{87}}} + \color{blue} \beta_{\Rb{85}}}{2}
    = \SI{26.94 \pm 1.16}{\micro\tesla} \ .
\]

\begin{figure}
    \centering
    \includegraphics[width=\textwidth]{build/plt/g_F.pdf}
    \caption{Messwerte und Regressionsgeraden zur Abhängigkeit der Peak-Positionen von der RF-Frequenz $f$.}
    \label{fig:plt:g_F}
\end{figure}

\begin{table}
    \centering
    \caption{Spulenströme und magnetische Flussdichten zu den Peaks beider Rubidium-Isotope für verschiedene Frequenzen $f$.}
    \label{tab:mess}
    \expandableinput{build/tab/messwerte.tex}
\end{table}

% \subsubsection{horizontales Erdmagnetfeld}


\subsection{Kernspin}
% Code sagt: I = J * (2/g_F - 1)
% MagischeMiesmuschel: I = g_J / (2*g_F - ½)
% elliekayl: I = ½((g_J/g_F) - 1)

% F = I + J

\autoref{eqn:landeF} umstellen nach $I$ ergibt:
\[
    I = \frac{1}{2} \left( \frac{g_J}{g_F} - 1 \right)
\]
mit $g_J = \num{2.0023}$ und $g_F$ aus dem vorherigen Abschnitt.


\subsection{Isotopenverhältnis}
Anhand eines Oszilloskop-Bilds wie in \autoref{fig:peaks} kann das Isotopenverhältnis
als Verhältnis der Höhe ihrer jeweiligen Peaks ermittelt weden.
…


\subsection{Quadratischer Zeeman-Effekt}
Schließlich wird die Energieaufspaltung des quadratischen Zeeman-Effekts abgeschätzt.
Die Hyperfeinstrukturaufspaltung des Grundzustands wird dazu aus \cite{versuchsanleitung} übernommen:
\begin{align*}
    \symup{\Delta}E_\text{HFS, \Rb{87}} &= \SI{4.53E-24}{\joule} \\
    \symup{\Delta}E_\text{HFS, \Rb{85}} &= \SI{2.01E-24}{\joule} \ .
\end{align*}
Für die maximalen Magnetfelder je Isotop
% (siehe \autoref{tab:mess})
\begin{align*}
    \max(B_{\Rb{87}}) &= \SI{168.19}{\micro\tesla} \\
    \max(B_{\Rb{85}}) &= \SI{235.75}{\micro\tesla} \\
\end{align*}
wird dann anhand von \autoref{eqn:quadr_zeeman} die Aufspaltung berechnet.

So ergibt sich
\begin{align*}
    % \symup{\Delta}E_\text{QZ, \Rb{87}} &= \SI{1.360(11)E-31}{\joule} \\
    % \symup{\Delta}E_\text{QZ, \Rb{85}} &= \SI{2.86(10)E-31}{\joule} \ .
    \symup{\Delta}E_\text{QZ, \Rb{87}} &=
        \underbrace{\SI{7.85(03)E-28}{\joule}}_{\propto B} +
        \underbrace{\SI{-4.08(03)E-31}{\joule}}_{\propto B^2} \\
        &= \SI{7.84(03)E-28}{\joule}
    \\
    \\
    \symup{\Delta}E_\text{QZ, \Rb{85}} &=
        \underbrace{\SI{7.58(14)E-28}{\joule}}_{\propto B} +
        \underbrace{\SI{-1.43(05)E-30}{\joule}}_{\propto B^2} \\
        &= \SI{7.57(14)E-28}{\joule}
\end{align*}
