\section{Auswertung}
\label{sec:auswertung}

% \subsection{Magnetfeld der Helmholtz-Spulenpaare}
\subsection{Erdmagnetfeld}

Die magnetische Flussdichte $B$ im Zentrum eines Helmholtz-Spulenpaars ist durch
\begin{equation}
    B = \frac{8}{\sqrt{125}} \frac{\mu_0 NI}{R}
    \label{eqn:helmholtz}
\end{equation}
gegeben,
wobei $N$ die Windungszahl, $I$ den Spulenstrom, und $R$ den Radius der beiden Spulen bezeichnet. %Oxford-Komma

\begin{table}
    \centering
    \caption{Daten zu den verwendeten Helmholtz-Spulen.}
    \label{tab:spulen}
    \begin{tabular}{l S S}
        \toprule
        Spule &
        {$N$} &
        {$R \mathbin{/} \si{\centi\meter}$} \\
        \midrule
        Sweep      &  11 & 16.39  \\
        Horizontal & 154 & 15.79  \\
        Vertikal   &  20 & 11.735 \\
        \bottomrule
    \end{tabular}
\end{table}

Es wurde experimentell bestimmt,
dass der Effekt des Erdmagnetfelds minimal ist,
wenn die Vertikal-Spule mit einem Strom von $I = \SI{123}{\ampere}$ betrieben wird.
Mit \autoref{tab:spulen} und \autoref{eqn:helmholtz} folgt daraus für die vertikale Komponente des Erdmagnetfelds
\[ B_\text{vert} = \SI{1234}{\micro\tesla} \ . \]
% Vergleichswert: 44 µT [Wikipedia]


\subsection{Gyromagnetische Faktoren}
Aus den gemäß \autoref{sec:durchfuehrung:todo} gemessenen Resonanzpositionen können die…

Indem die magnetische Flussdichte in Abhängigkeit der RF-Frequenz, also als $B(f)$, ausgedrückt wird,
ergibt sich ein linearer Zusammenhang ($B(f) = m·f+b$),
welcher mithilfe einer Regressionsrechnung
(unter Verwendung von \texttt{scipy.stats.linregress})
quantifiziert wird.
Es ergeben sich die Parameter
\begin{align*}
    m_{\Rb{87}} &= \SI{0}{\micro\tesla\per\hertz} \\
    b_{\Rb{87}} &= \SI{0}{\micro\tesla} \\
    \intertext{und}
    m_{\Rb{85}} &= \SI{0}{\micro\tesla\per\hertz} \\
    b_{\Rb{85}} &= \SI{0}{\micro\tesla} \ .
\end{align*}

Durch Einsetzen in … folgt für die gyromagnetischen Faktoren
\begin{align*}
    m_{\Rb{87}} &= \SI{0}{} \\
    m_{\Rb{85}} &= \SI{0}{} \ .
\end{align*}

% TODO: Mit Theorie angleichen; \omega oder f gemessen?
Durch Umstellen von \autoref{eqn:todo} ergibt sich
\begin{equation*}
    B(f) = \underbrace{\frac{\hbar}{g_F µ_B}}_m · \underbrace{2\pi f}_\omega \ .
\end{equation*}

\begin{figure}
    \centering
    \includegraphics[width=\textwidth]{build/plt/g_F.pdf}
    \caption{Messwerte und Regressionsgeraden zu TODO.}
    \label{fig:plt:g_F}
\end{figure}


\subsection{Kernspin}


\subsection{Isotopenverhältnis}


\subsection{Quadratischer Zeeman-Effekt}
