\section{Diskussion}
\label{sec:diskussion}

\subsection{Abweichungen}

…


\subsection{Mögliche Fehlerquellen}

    Zu Beginn der Messung musste eine Einjustierung von optischen Elementen,
    sowie der Ausrichtung des Tisches vorgenommen werden.
    Diese geschah alleinig nach Augenmaß,
    sodass es durchaus eine noch bessere Position der jeweiligen Elemente,
    oder eine bessere Einstellung der Stromstärke der Vertikalspule geben könnte,
    für die der beobachtete Peak noch schmaler würde.

    Eine weitere Schwierigkeit bestand darin,
    die entstehenden Peaks der Rubidium-Isotope auf dem Oszilloskop zu vermessen,
    da sich aufgrund des Sweeps und des äußeren Lichts,
    - welches trotz der Abdeckung mit zusätzlichen Jacken nicht ganz ausgeschlossen werden konnte -
    das Bild auf dem Oszilloskop in vertikaler Richtung verschoben hat.
    Es ist gut möglich,
    dass aufgrund dessen nicht die exakte Spitze des Peaks getroffen wurde.
    Während dieser Messung wurde der Strom durch die horizontale Spule mittels eines externen Generators erzeugt und auf einem Amperemeter abgelesen,
    wobei dieser Wert zwischenzeitig stark schwankte und teilweise ein Mittelwert gewählt wurde.
    Auch die Skala der Potentiometer zur Einstellung des Vertikal- und horizontalen Sweep-Feldes konnte nicht eindeutig abgelesen werden.
    Beim Ablesen der Amplitude auf dem Oszilloskop können zusätzlich Fehler enstanden sein,
    da sich das Bild bewegte.
%\begin{itemize}
%    \item Schrägstellung → kleineres B_vert
%    \item Abschirmung durch Gebäude → kleineres B_vert
%    \item Ungenaue Ausrichtung entlang des horizontalen Erdmagnetfelds → kleineres B_hor (?)
%    \item …
%\end{itemize}
