\section{Diskussion}
\label{sec:diskussion}

\subsection{Abweichungen}

Die \hyperref[sec:auswertung:B_erd_vert]{Vertikalkomponente des Erdmagnetfeldes}
wurde zu \SI{35.25 \pm 0.15}{\micro\tesla} bestimmt.
Dieser Messwert weicht vom gängigen Literaturwert \SI{44}{\micro\tesla} \cite{erdmagnetfeld}
um \SI{19.89}{\percent} ab.
Möglicherweise wurde das Erdmagnetfeld vom Gebäude teilweise abgeschirmt.


In \autoref{sec:auswertung:g_F} wurden die gyromagnetischen Faktoren beider Rubidium-Isotope bestimmt.
Für \Rb{87} ergab sich \num{0.503 \pm 0.002}.
Somit weicht die Messung um \SI{0.63}{\percent} vom Theoriewert \num{0.500} ab.
Für \Rb{85} ergab sich \num{0.347 \pm 0.006}.
Die Messung weicht somit um \SI{4.06}{\percent} vom Theoriewert \num{0.333} ab.


Vergleichbare Abweichungen zeigen sich bei der \hyperref[sec:auswertung:kernspin]{Berechnung der Kernspins}.
Die Theoriewerte lauten hier
\begin{align*}
    I_\text{lit, \Rb{87}} &= \frac{3}{2} = \num{1.50} \\
    I_\text{lit, \Rb{85}} &= \frac{5}{2} = \num{2.50} \ .
\end{align*}
Somit betrtägt die relative Abweichung
\SI{0.68}{\percent} für \Rb{87} und
\SI{4.55}{\percent} für \Rb{85}.


Die Horizontalkomponente des Erdmagnetfeldes
als $y$-Achsenabschnitt der Regressionsgerade
wurde zu \SI{26.94 \pm 1.16}{\micro\tesla} bestimmt.
Vom Literaturwert \SI{20}{\micro\tesla} \cite{erdmagnetfeld}
weicht die Messung entsprechend um \SI{34.72}{\percent} ab;
hier wird das Magnetfeld jedoch überschätzt statt unterschätzt.


Das \hyperref[sec:auswertung:isotopenverhaeltnis]{Isotopenverhältnis}
wurde aus einer einzelnen Messung zu
\def\verh#1#2{#1 : #2}
$\verh{\Rb{87}}{\Rb{85}} = \verh{1}{2}$
ermittelt.
Das natürliche Verhältnis liegt bei
$\verh{\num{27.83(2)}}{\num{72.17(2)}}$ \cite{isotopenverhaeltnis},
was einer Abweichung von \SI{29.66}{\percent} entspricht.
Jedoch ist bekannt, dass das Isotopengemisch in der Dampfzelle mit \Rb{87} angereichert ist.
% ↑ Auf https://www.teachspin.com/optical-pumping wird nur in Bezug auf die Rb-Lampe von Anreicherung gesprochen,
% uns wurde aber Gegenteiliges gesagt.


% \hyperref[sec:auswertung:quad_zeeman]{…}
% Was soll man dazu sagen?
% Ist bei den hier erreichten B noch vernachlässigbar.


\subsection{Mögliche Fehlerquellen}

    Zu Beginn der Messung mussten eine
    \hyperref[sec:durchfuehrung:einjustierung]{Einjustierung} von optischen Elementen vorgenommen
    und der Tisch ausgerichtet werden.
    Beides geschah alleinig nach Augenmaß unter Zuhilfenahme des Oszilloskops,
    sodass es durchaus eine noch bessere Position der jeweiligen Elemente,
    oder eine bessere Einstellung der Stromstärke der Vertikalspule geben könnte,
    für die die Vertikalkomponente des Erdmagnetfeldes besser kompensiert würde.

    Eine weitere Schwierigkeit bestand darin,
    die auf dem Oszilloskop entstehenden \hyperref[sec:durchfuehrung:messung]{Peaks der Rubidium-Isotope} zu vermessen,
    da sich aufgrund des Sweeps und des äußeren Lichts
    – welches trotz zusätzlicher Abdeckung mit Jacken nicht ganz ausgeschlossen werden konnte –
    das Bild auf dem Oszilloskop in vertikaler Richtung verschob.
    Es ist gut möglich,
    dass aufgrund dessen nicht die exakte Spitze des Peaks getroffen wurde.
    Während dieser Messung wurde der Strom durch die horizontale Spule mittels eines externen Generators erzeugt und auf einem Amperemeter abgelesen,
    dessen angezeigter Wert zwischenzeitig stark schwankte,
    sodass teilweise ein Mittelwert gewählt werden musste.
    %
    Auch die Skala der Potentiometer zur Einstellung des Vertikal- und horizontalen Sweep-Feldes konnte nicht eindeutig abgelesen werden.
    Beim Ablesen der Amplituden auf dem Oszilloskop können zusätzlich Fehler enstanden sein,
    da der dargestellte Graph eine nicht unwesentliche Dicke aufwies und mit der Zeit fluktuierte.

    Für die Vermessung des Erdmagnetfelds war die Größe des angelegten Magnetfelds relevant,
    während zur Bestimmung von $g_F$ und den darauf aufbauenden Werten nur die Steigung der Regressionsgerade vonnöten war.
    Dementsprechend könnten erstere Ergebnisse durch systematische Fehler wie unzureichend geeichte Amperemeter verfälscht sein.\\
    Eine weitere mögliche Fehlerquelle liegt in der Ausrichtung des Aufbaus:
    Indem dieser entlang der horizontalen Feldlinien gekippt wird,
    wird die Horizontalkomponente zunehmend überschätzt,
    da die tatsächliche Vertikalkomponente größer als die tatsächliche Horizontalkomponente ist.
    Eine Verdrehung des Aufbaus
    als Folge von ungenauer Ausrichtung entlang der Horizontalkomponente
    würde hingegen zur Unterschätzung derselben führen.
