\section{Diskussion}
\label{sec:diskussion}

\subsection{Abweichungen}

Das bei der Eichung des Elektromagneten genutzte Polynom dritten Grades
beschreibt den Verlauf der gemessenen Kurve sehr gut.


Der Landé-Faktor zur roten Linie weist eine relative Abweichung von lediglich \SI{0.72}{\percent} auf;
der theoretische Wert liegt bei $g_\text{rot, theo} = 1$,
der gemessene Wert ist $g_\text{rot} = \num{0.99 \pm 0.06}$.
Hierbei ist jedoch anzumerken, dass die Auswahl der zu berücksichtigen Linien einen wesentlichen Einfluss
auf das Ergebnis hat.
% Bei einer anderen Wahl hätte sich beispielsweise XY ergeben.

Der Landé-Faktor zur blauen $\sigma$-Spektrallinie wurde nur ungenau
– mit einer relativen Abweichung von \SI{25.56}{\percent} –
ermittelt.
Möglicherweise war das Bild zu unscharf.

Der Landé-Faktor zur blauen $\pi$-Spektrallinie konnte nicht sinnvoll ermittelt werden,
da mit dem verwendeten Versuchsaufbau keine ausreichenden magnetischen Flussdichten erreicht werden konnten,
um die Interferenzlinien klar zu trennen.
Der in \autoref{sec:vorb_magnetfeldstaerke} berechnete Idealwert beträgt mehr als das Doppelte vom tatsächlich erreichten Magnetfeld.

\autoref{tab:diskussion} zeigt eine Übersicht der vorhergesagten und gemessenen $g$ sowie der zugehörigen Abweichungen.

\begin{table}
    \centering
    \caption{Übersicht der Landé-Faktoren und zugehörigen Abweichungen.}
    \label{tab:diskussion}
    \begin{tabular}{l S S S}
        \toprule
        {Spektrallinie} &
        {$g_\text{theo}$} &
        {$g_\text{mess}$} &
        {$\text{rel. Abweichung} \mathbin{/} \si{\percent}$} \\
        \midrule
        rot            & 1.00 & 0.99 \pm 0.06 & - 0.72 \\
        blau, $\sigma$ & 1.75 & 1.30 \pm 0.04 & -25.56 \\
        blau, $\pi$    & 0.50 & 0.99 \pm 0.06 &  97.74 \\
        \bottomrule
    \end{tabular}
\end{table}


\subsection{Mögliche Fehlerquellen}

    Ein möglicher Grund für die Abweichung ist das Einjustieren der Linsen und Spalte vor Beginn der eigentlichen Messung,
    da dies alleinig nach Augenmaß geschah.
    Es ist möglich,
    dass ein besseres Ergebnis bei leicht veränderten Positionen der Linsen und Spalte erreicht werden kann,
    wenn das Licht beispielsweise noch schärfer dargestellt wird.
    Eine andere Fehlerquelle kann das Einstellen des Stromes zur Erzeugung des Magnetfelds bei der Messung sein,
    da nach dem Eichen des Magnetfelds in Schritten von $\SI{0.2}{\ampere}$ der Stromwert gewählt wurde,
    bei dem die Stärke des Magnetfelds am nächsten am theoretisch berechneten Wert lag.
    Der exakte Wert wurde nicht eingestellt
    und konnte im Falle der roten Linie und der $\pi$-polarisierten, blauen Linie nicht erreicht werden,
    da der Strom gar nicht so hoch eingestellt werden konnte.
    Gerade im Falle der $\pi$-polarisierten, blauen Linie konnte nur etwa die Hälfte der benötigten Magnetfeldstärke erreicht werden.
    Auch die Belichtungszeit der Kamera hatte Einfluss auf die Qualität der Bilder.
    % Link? ↓
    Aufgrund der Ausrichtung des Magnetfelds im Aufbau war es auch nicht möglich,
    eine Aufspaltung in drei Linien zu beobachten.
    % TODO: Erdmagnetfeld?
