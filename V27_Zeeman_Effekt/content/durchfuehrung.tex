\section{Durchführung}
\label{sec:durchfuehrung}
Im Folgenden wird der Aufbau und die Durchführung des Versuches beschrieben.


\subsection{Aufbau des Versuches}

Für die Durchführung des Versuches ist ein Aufbau nach \autoref{fig:aufbau} gegeben.
\begin{figure}[H]
    \centering
    \includegraphics[width=\textwidth]{content/img/Abb_1.pdf}
    \caption{Schematischer Aufbau der Messapparatur. \cite{versuchsanleitung}}
    \label{fig:aufbau}
\end{figure}

Das Licht wird von einer Cadmium-Spektrallampe erzeugt,
welche sich in einem Magnetfeld befindet,
das von stromdurchflossenen Spulen erzeugt wird.
Mithilfe verschiedener Linsen und einem Spalt wird der Lichtstrahl gebündelt und fällt auf ein Geradsichtprisma,
welches das Licht in seine verschiedenen Wellenlängen aufspaltet.
Zudem befindet sich ein Polarisationsfilter im Aufbau,
der bei der tatsächlichen Durchführung allerdings vor dem Geradsichtprisma positioniert wurde.
Dieser dient dazu,
die jeweiligen Polarisationen ($\sigma$ und $\pi$) zu unterscheiden.
Nachdem das Licht durch das Prisma aufgespalten wurde,
wird es durch eine zusätzliche Linse und einen Spalt weiter gebündelt und trifft schließlich auf das Eingangsprisma der Lummer-Gehrcke-Platte.
Der Strahlengang des Lichts innerhalb der Lummer-Gehrcke-Platte ist in \autoref{fig:lg_platte} gezeigt.
\begin{figure}[H]
    \centering
    \includegraphics[scale=1]{content/img/Abb_2.pdf}
    \caption{Strahlengang des Lichts innerhalb der Lummer-Gehrcke-Platte. \cite{versuchsanleitung}}
    \label{fig:lg_platte}
\end{figure}
Das resultierende Bild der Lummer-Gehrcke-Platte wird mit einer Digitalkamera aufgenommen,
die so ausgerichtet wird,
dass das Bild die größtmöglichen Abstände zwischen den einzelnen Linien zeigt.
Vom optischen Zoom der Kamera wurde maximaler Gebrauch gemacht.


\subsection{Messung der Aufspaltung der Spektrallinien in einem externen Magnetfeld}

Zu Beginn der Messung wird das Magnetfeld geeicht,
indem anstelle der Cadmium-Lampe eine Hall-Sonde genau mittig zwischen den beiden Spulen positioniert wird,
mithilfe der das Magnetfeld gemessen werden kann.
Nun wird der Strom an einem Generator langsam erhöht und in einem festen Abstand werden Strom und resultierendes Magnetfeld gemessen.
Die Eichung soll die Messung erleichtern,
da nun nur ein Stromwert eingestellt werden muss,
aber das Magnetfeld nicht zusätzlich gemessen wird.

Anschließend werden die Linsen so eingestellt,
dass ein möglichst scharfes Bild auf die Spalte, das Prisma und die Lummer-Gehrcke-Platte trifft.
Mithilfe des Spalts $\mathrm{S_2}$ kann entweder die rote oder die blaue Linie auf den Eingang der Lummer-Gehrcke-Platte gerichtet werden.

Zuerst wird die Aufspaltung der roten Linie gemessen.
Dazu wird der Polarisationsfilter auf $\SI{0}{\degree}$ eingestellt und ein Foto mit ausgeschaltetem Magnetfeld gemacht.
Anschließend wird der Strom maximal eingestellt,
da die in \autoref{sec:vorb_magnetfeldstaerke} abgeschätzte Magnetfeldstärke nicht erreicht werden kann.
Nun wird für $\SI{0}{\degree}$- und $\SI{90}{\degree}$- polarisiertes Licht ein Foto mit der Digitalkamera aufgenommen.
Abschließend wird der Strom wieder heruntergeregelt,
um die Spulen nicht zu beschädigen.
Für die blaue Linie wird analog vorgegangen.
Hier ist allerdings zu beachten,
dass für die verschiedenen Polarisationen entsprechend verschiedene Magnetfeldstärken (TODO Link) eingestellt werden müssen.
Auch hier werden wieder jeweils Fotos ohne Magnetfeld und mit verschiedener Einstellung des Polarisationsfilters aufgenommen.
