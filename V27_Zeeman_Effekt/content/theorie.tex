\section{Theorie}
\label{sec:theorie}

    Im folgenden Abschnitt werden kurz die theoretischen Grundlagen des Versuches erläutert.

\subsection{Energiequantelung eines Atoms}

    In einem Atom,
    wie zum Beispiel dem Wasserstoffatom,
    existieren diskrete, gequantelte Energieniveaus.
    Aufgrund der Quantisierung sind die einzelnen Niveaus entartet.
    Die Niveaus können durch die Haupt- oder Energiequantenzahl $n \in \symbb{N}$,
    die Bahndrehimpulsquantenzahl $l \in \{0, 1, 2, ... , n-1\}$ und die magnetische Quantenzahl $m \in \{-l, -l+1, ..., l-1, l\}$,
    welche $2l + 1$ Werte annehmen kann,
    beschrieben werden.
    Zudem kann jedem Teilchen ein Spin zugeordnet werden.
    Ein Elektron besitzt den Spin $s = \frac{1}{2}$.
    Der Gesamtdrehimpuls der Atomhülle wird mit $j$ beschrieben.
    Die Entartung der Energien kann durch verschiedene Prozesse aufgehoben werden,
    welche eine Aufspaltung der Energieniveaus zufolge haben und entweder mit oder ohne externe Felder stattfinden.
    Eine Möglichkeit ist die sogenannte Feinstrukturaufspaltung.
    Grundlage dieses Effekts sind die magnetischen Momente $\vec{\mu_l}$ des Bahndrehimpulses und $\vec{\mu_s}$ des Spins des Elektrons,
    die untereinander und mit einem magnetischen Feld wechselwirken.
    Bei der Feinstruktur wird die Wechselwirkung der Hüllenelektronen mit dem geringen Magnetfeld des Kerns betrachtet.
    Die magnetischen Momente sind durch folgende Gleichungen beschrieben
    \begin{gather}
        \lvert \vec{\mu_l} \rvert = \mu_\text{B} \sqrt{l(l+1)} = g_l \mu_\text{B} \frac{l}{\hbar}
        \label{eqn:magn_mom_l} \\
        \lvert \vec{\mu_s} \rvert = \mu_\text{B} \sqrt{s(s+1)} = g_s \mu_\text{B} \frac{s}{\hbar} \ .
        \label{eqn:magn_mom_s}
    \end{gather}
    Der Faktor $\mu_\text{B} = \frac{e \hbar}{2 m_{e}}$ ist das Bohr'sche Magneton und die Faktoren $g_{i}$ mit $i \in \{j,l,s\}$ werden als Landé-Faktoren bezeichnet.
    Sie beschreiben das Verhältnis der Messung der jeweiligen Größe zur klassischen Erwartung.
    Der Faktor lässt sich mithilfe von
    \begin{equation}
        g_{j} = 1 + \frac{j(j+1) + s(s+1) - l(l+1)}{2j(j+1)}
        \label{eqn:lande}
    \end{equation}
    berechnen.
    Es gilt $g_l = 1$ im Falle von $s = 0$ und $g_s \approx 2$ im Falle von $l = 0$.\\
    Bei der \textbf{LS-Kopplung} ist die Kopplung der einzelnen Momente an das Magnetfeld stärker als die Kopplung untereinander.
    In Atomen mit mehreren Elektronen setzen sich Bahndrehimpuls und Spin aus den Werten der einzelnen Atome zusammen
    \begin{align}
        L = \sum_k l_k && S = \sum_k s_k \ .
    \end{align}
    Analog zu \autoref{eqn:magn_mom_l} und \autoref{eqn:magn_mom_s} können die magnetischen Momente berechnet werden.
    Bei sehr schweren Atomen,
    die eine höhere Ordnungszahl besitzen,
    wird die \textbf{jj-Kopplung} relevant.
    In diesem Fall ist die Kopplung der magnetischen Momente $\mu_l$ und $\mu_s$ untereinander stärker als einzeln zum Magnetfeld,
    sodass ein Gesamtdrehimpuls $j_k = l_k + s_k$ und analog $J = \sum_k j_k$ definiert werden kann.

\subsection{Der Zeeman-Effekt}

    Der Zeeman-Effekt beschreibt die Aufspaltung der Energieniveaus in einem äußeren Magnetfeld.
    Das magnetische Moment des Gesamtdrehimpuls kann in einem Magnetfeld $\vec{B} = \lvert \vec{B} \rvert \vec{e_z}$ hat die Form
    \begin{equation}
        \vec{\mu_J} = - m g_j \mu_\text{B} \vec{e_z} \ .
    \end{equation}
    Zwischen den aufgespaltenen Energieniveaus besteht die Energiedifferenz
    \begin{equation}
        \symup{\Delta} E = g_j \symup{\Delta} m \mu_\text{B} \lvert \vec{B} \rvert \ .
        \label{eqn:energiedifferenz}
    \end{equation}
    Aufgrund der Drehimpulserhaltung bei Übergängen zwischen den einzelnen Energieniveaus,
    bei denen Photonen mit der entsprechenden Wellenlänge der Energiedifferenz emittiert werden,
    kann $\symup{\Delta} l$ die Werte $\pm 1$ annehmen und entsprechend gilt $\symup{\Delta} m = \pm 1, 0$.
    Diese Einschränkungen der Übergänge werden als Auswahlregeln bezeichnet.
    Abhängig vom tatsächlichen Wert von $\symup{\Delta} m$ ist das emittierte Licht unterschiedlich polarisiert.
    Für $\symup{\Delta} m = \pm 1$ ist das Licht zirkular in Richtung des Magnetfelds polarisiert.
    Dies wird als $\sigma^+$ und $\sigma^-$ Polarisation bezeichnet.
    Für $\symup{\Delta} m = 0$ ist das Licht linear in Richtung des Magnetfelds polarisiert.
    Dies wird als $\pi$ Polarisation bezeichnet.\\
    Nun muss noch zwischen dem normalen und annormalen Zeeman-Effekt unterschieden werden.

\subsubsection{Normaler Zeeman-Effekt}

    Beim normalen Zeeman-Effekt werden ausschließlich spinlose Teilchen betrachtet,
    es gilt also $S = 0$ und demnach $g_j = g_l = 1$ nach \autoref{eqn:lande}.
    %\begin{figure}
    %   \centering
    %    \includegraphics[width=\textwidth]{}
    %    \caption{Schematische Darstellung der Aufspaltung von Energieniveaus außerhalb und innerhalb eines Magnetfelds beim Normalen Zeeman-Effekt.\cite{haken_wolf}}
    %    \label{fig:normal_zeeman}
    %\end{figure}
    Die \autoref{fig:normal_zeeman} zeigt einen möglichen Übergang des Cadmium-Atoms.
    Außerhalb des Magnetfelds unterscheiden sich die Energieniveaus alleinig durch den Bahndrehimpuls,
    während im Magnetfeld ein Aufspaltung in drei Linien auftritt.
    Je nachdem,
    welchen Wert $\symup{\Delta} m$ annimmt,
    ist das bei Übergang entstehende Licht $\sigma-$ oder $\pi-$ polarisiert.
    Abhängig davon,
    ob in longitudinaler oder transversaler Richtung zum Magnetfeld beobachtet wird,
    ist nicht jede Polarisation sichtbar.
    \autoref{fig:aufspaltung_richtung} zeigt dies.
    %\begin{figure}
    %    \centering
    %    \includegraphics[width=\textwidth]{}
    %    \caption{Sichtbare Aufspaltung, abhängig von der Beobachtungsrichtung.\cite{haken_wolf}}
    %    \label{fig:aufspaltung_richtung}
    %\end{figure}
    In longitudinaler Richtung,
    also entlang des Magnetfelds,
    sind nur $\sigma-$ polarisierte Linien zu erkennen,
    in transversaler Richtung,
    also senkrecht zum Magnetfeld,
    sind sowohl $\sigma-$ als auch $\pi-$ Linien zu erkennen.

\subsubsection{Annormalen Zeeman-Effekt}

    Der annormale Zeeman-Effekt tritt bei Teilchen mit Spin $S \neq 0$ auf,
    sodass $g_j \neq g_l \neq g_s$ gilt.
    Aus diesem Grund sind die Übergänge zusätzlich zu Bahndrehimpuls auch vom Spin der Atomhülle abhängig.
    Die \autoref{} zeigt die möglichen Übergänge des annormalen Zeeman-Effekts im Cadmium-Atom.
    %\begin{figure}
    %   \centering
    %    \includegraphics[width=\textwidth]{}
    %    \caption{Schematische Darstellung der Aufspaltung von Energieniveaus außerhalb und innerhalb eines Magnetfelds beim Annormalen Zeeman-Effekt.}
    %    \label{fig:annormal_zeeman}
    %\end{figure}

\subsection{Vorbereitung}

    Die Vorbereitung dient als Grundlage der Durchführung,
    um zu wissen,
    welche Magnetfeldstärken und Übergänge zu erwarten sind.

\subsubsection{Übergang der Roten und Blauen Cd-Linie}
\label{sec:vorb_uebergaenge}

    Es sollen nun die Übergänge der roten und der blauen Cd-Linie betrachtet werden.
    Die Energieniveaus sind nach dem Schema $^{2S+1}L_J$ dargestellt.
    Für den Übergang der roten Linie $^1P_1 \leftrightarrow ^1D_2$ ergibt sich das Schema,
    wie in \autoref{fig:normal_zeeman} dargestellt.
    Es sind sechs $\sigma-$ und drei $\pi-$ Übergänge möglich.
    Der Landé-Faktor kann mit \autoref{eqn:lande} gerechnet werden.
    Für beide Niveaus ergibt sich $g_j = g_l = 1$.
    Es handelt sich um den normalen Zeeman-Effekt,
    da $S = 0$ gilt.
    Die Energiedifferenz kann nach \autoref{eqn:energiedifferenz} zu
    \begin{equation*}
        \symup{\Delta} E =
        \begin{cases}
            - \mu_\text{B} \lvert \vec{B} \rvert & , \symup{\Delta} m = -1 \\
            0 & , \symup{\Delta} m = 0 \\
            \mu_\text{B} \lvert \vec{B} \rvert & , \symup{\Delta} m = +1
        \end{cases}
    \end{equation*}
    berechnet werden.
    Die blaue Linie entsteht bei dem Übergang $^3S_1 \leftrightarrow ^3P_1$,
    der Landé-Faktor ergibt sich mit \autoref{eqn:lande} zu
    \begin{gather*}
        g_j(^3S_1) = 1 + \frac{1(1+1)+1(1+1)}{2 \cdot 1(1+1)} = 2 \\
        g_j(^3P_1) = 1 + \frac{1(1+1)+1(1+1)-1(1+1)}{2 \cdot 1(1+1)} = \frac{3}{2} \ .
    \end{gather*}
    Es gilt $S \neq 0$ und damit der annormale Zeeman-Effekt.
    Die möglichen Übergänge sind in \autoref{fig:annormal_zeeman} darstellt.
    Anders,
    als beim normalen Zeeman-Effekt,
    sind hier vier $\sigma-$ und drei $\pi-$ Übergänge möglich.
    Die Energiedifferenz kann mithilfe von
    \begin{gather}
        \symup{\Delta} E = g_{ab} \mu_\text{B} \lvert \vec{B} \rvert \ ,
        \label{eqn:energiedifferenz_annormal}\\
        g_{ab} = m_a g_a - m_b g_b
        \label{eqn:lande_annormal}
    \end{gather}
    berechnet werden,
    wobei $a$ das höhere und $b$ das niedrigere Niveau beschreibt.
    \begin{table}
        \centering
        \caption{Lande-Faktoren $g_{ab}$ zur Bestimmung der Energiedifferenz zwischen den Niveaus nach \autoref{eqn:lande_annormal}.}
        \label{tab:lande_ab}
        \begin{tabular}{c c c c}
            \toprule
            & \multicolumn{2}{c}{$m_b$}  \\
            \cmidrule(lr){2-4}
            {$m_a$} & {$-1$} & {$0$} & {$+1$} \\
            \midrule
            $-1$ & $- \frac{1}{2}$ & $- \frac{3}{2}$ & $-$ \\
            $0$ &  $2$ & $0$ & $-2$ \\
            $+1$ & $-$ & $\frac{3}{2}$ & $\frac{1}{2}$ \\
            \bottomrule
        \end{tabular}
    \end{table}
    \autoref{tab:lande_ab} zeigt die verschiedenen Werte von $g_{ab}$ mit denen sich die verschiedenen Energiedifferenzen zwischen den Niveaus ergeben.

\subsubsection{Berechnung des Dispersionsgebiets und der spektralen Auflösung}

    Damit sich die Aufspaltungslinien nicht überschneiden,
    wird ein Dispersionsgebiet
    \begin{equation}
        \symup{\Delta}\lambda_\text{D} = \frac{\lambda^2}{2d} \frac{1}{\sqrt{n^2 - 1}}
        \label{eqn:dispersionsgebiet}
    \end{equation}
    definiert.
    Die Variablen $d$ und $L$ stellen die Abmessungen der Lummer-Gehrcke-Platte dar und $n$ bezeichnet den Brechungsindex.
    Zudem hat die Lummer-Gehrcke-Platte eine Auflösung,
    die durch
    \begin{equation}
        A = \frac{L}{\lambda} (n^2 -1)
        \label{eqn:aufloesung}
    \end{equation}
    beschrieben wird.
    In \autoref{tab:vorb_sechs} sind die Wellenlängen der roten und blauen Cd-Linie,
    sowie das nach \autoref{eqn:dispersionsgebiet} berechnete $\symup{\Delta}\lambda_\text{D}$ und $A$ nach \autoref{eqn:aufloesung} gegeben.
    Für die Lummer-Gehrcke-Platte gilt $d = \SI{4}{\milli\meter}$ und $L = \SI{120}{\milli\meter}$.

    \begin{table}
        \centering
        \caption{Dispersionsgebiet und Auflösung der Lummer-Gehrcke-Platte für die rote und blaue Linie.}
        \label{tab:vorb_sechs}
        \begin{tabular}{c c c c}
            \toprule
            {$\lambda \mathbin{/} \si{\nano\meter}$} & {$n$} & {$\symup{\Delta}\lambda_\text{D} \mathbin{/} \si{\pico\meter}$} & {$A$} \\
            \midrule
            643.8 & 1.4567 & 48.91 & 209128.5 \\ %Bin mir hier mit dem A-Wert nicht sicher, bitte überprüfen!
            480.0 & 1.4635 & 26.95 & 285458.0 \\
            \bottomrule
        \end{tabular}
    \end{table}

\subsubsection{Abschätzung der optimalen Magnetfeldstärke}
\label{sec:vorb_magnetfeldstaerke}

    Um zwischen den verschiedenen Polarisationen unterscheiden zu können,
    muss die optimale Magnetfeldstärke der jeweiligen Aufspaltung bestimmt werden.
    Diese ergibt sich aus \autoref{eqn:energiedifferenz},
    wobei sich $\symup{\Delta} E$ über die Wellenlänge des emittierten Lichts bestimmen lässt.
    Ohne Aufspaltung ergibt sich ein Abstand $\symup{\Delta} s$ zwischen den Linien,
    welcher der Wellenlänge $\symup{\Delta} \lambda_\text{D}$ entspricht.
    Mit Aufspaltung ändert sich der Abstand zu $\symup{\delta} s$,
    was $2 \symup{\delta} \lambda$ entspricht.
    Es gilt
    \begin{gather*}
        \frac{\symup{\delta} s}{\symup{\Delta} s} = \frac{1}{2} = \frac{2 \symup{\delta} \lambda}{\symup{\Delta} \lambda_\text{D}} \\
        \intertext{und dementsprechend} \\
        \symup{\delta} \lambda = \frac{1}{4} \symup{\Delta} \lambda_\text{D} \ . \\
        \intertext{Mit der Änderung von $E$} \\
        \symup{\Delta} E = - h \frac{c}{\lambda^2} \symup{\delta} \lambda \\
        \intertext{ergibt sich die Magnetfeldstärke mit \autoref{eqn:energiedifferenz_annormal} zu} \\
        B = \frac{h c}{4 \lambda^2} \frac{\symup{\Delta} \lambda_\text{D}}{\mu_\text{B} g} \ .
    \end{gather*}
%Etwas unelegant vielleicht mit dem vielen Intertext, aber einfacher, als immer eine neue Gleichung zu machen
    Für die rote Linie lässt sich der Wert $B = \SI{0.632}{\tesla}$ berechnen.
    Für die blaue Linie ergibt sich bei $\pi-$ polarisiertem Licht $B = \SI{1.253}{\tesla}$,
    bei $\sigma-$ polarisiertem Licht $B = \SI{0.313}{\tesla}$ bei $g=2$ und $B = \SI{0.418}{\tesla}$ bei $g=\frac{3}{2}$.



