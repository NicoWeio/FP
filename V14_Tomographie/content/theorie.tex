\section{Theorie}
\label{sec:theorie}

    Im folgenden Abschnitt werden die theoretischen Grundlagen für diesen Versuch erläutert werden.
    Dabei wird zum einen die Wechselwirkung von Gamma-Strahlung mit Materie,
    sowie die Detektion der Strahlung beschrieben.

\subsection{Wechselwirkung von Gamma-Strahlung mit Materie}

    Gamma-Strahlung besteht aus hochenergetischen Photonen die bei Zerfallsprozessen erzeugt werden.
    In diesem Fall wird der Zerfall des radioaktiven Zerfalls $^{137}\ce{Cs}$ betrachtet,
    welches über einen $\beta$-Zerfall und anschließenden $\gamma$-Zerfall in das stabile $^{137}\ce{Ba}$ zerfällt.
    Dabei wird Strahlung der Energie $E_{\symup{\gamma}} \approx \SI{662}{\kilo\eV}$ \cite{caesium} frei.
    Dieser Prozess ist in \autoref{} dargestellt.
    %\begin{figure}
    %    \centering
    %    \includegraphics[width=\textwidth]{}
    %    \caption{Zerfallsprozess von $^{137}\ce{Cs}$ in das stabile $^{137}\ce{Ba}$. \cite{caesium}}
    %    \label{fig:theorie:zerfallsprozess}
    %\end{figure}

    Wenn die $\gamma$-Strahlung nun auf Materie trifft,
    muss abhängig von der Energie der Strahlung zwischen folgenden Effekten unterschieden werden \cite{radioaktivitaet}.
    Für niedrigere Energien (\%Bereich aus KET) findet größtenteils \textit{Photoeffekt} statt.
    Dabei muss die Energie des Photons $E_{\symup{\gamma}} = hf$ der Bindungsenergie eines der Hüllenelektronen entsprechen.
    Das Photon gibt in diesem Fall seine gesamte Energie an das Elektron ab,
    welches aus dem Atom herausgelöst wird.
    Das Elektron behält dabei eine Energie $E_{e} = hf - E_\text{Bind}$.
    Der Wirkungsquerschnitt unterscheidet sich dabei,
    ob die Energie des Photons kleiner oder größer der relativstischen Ruheenergie des Elektrons ist.
    Es gilt
    \begin{equation*}
        \sigma_\text{Photo} \propto
        \begin{cases}
            \sfrac{Z^5}{E^{\sfrac{7}{2}}_{\symup{\gamma}}} & , E_{\symup{\gamma}} \ll m_{e} c^2 \\
            \sfrac{Z^5}{E_{\symup{\gamma}}} & , E_{\symup{\gamma}} \gg m_{e} c^2 \\
        \end{cases} \ .
    \end{equation*}
    Für zunehmende Energien dominiert der \textit{Compton-Effekt}.
    Das Photon gibt in diesem Fall nicht mehr seine gesamte Energie ab,
    sondern stößt elastisch mit lose gebundenen oder freien Elektronen.
    Dabei gilt Energie- und Impulserhaltung,
    sodass das Photon einen Teil seiner Energie an das Elektron abgibt und beide eine Richtungsänderung erfahren.
    Für den Wirkungsquerschnitt gilt
    \begin{equation*}
        \sigma_\text{Compton} \propto \sfrac{Z}{E_{\symup{\gamma}}} \ .
    \end{equation*}
    Für eine Energie $E_{\symup{\gamma}} \geq 2m_{e}c^2 = \SI{1.022}{\mega\eV}$ kann \textit{Paarerzeugung} stattfinden.
    Dabei wandelt sich das Photon im Coulombfeld des Atoms in ein Elektron-Positron-Paar um.
    Dieser Prozess kann aufgrund der Erhaltungssätze nur mit dem Atomkern als weiteren Stoßpartner stattfinden.
    Die entstehenden Teilchen haben eine Energie von $E_{e^{+}} + E_{e^{-}} = E_{\symup{\gamma}} - 2m_{e}c^2$.
    Weiterhin gilt
    \begin{equation*}
        \sigma_\text{Paar} = Z^2 \ln(E_{\symup{\gamma}}) \ .
    \end{equation*}

    In jedem der Fälle wird die Gamma-Strahlung vom Material absorbiert.

