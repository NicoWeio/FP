\section{Theorie}
\label{sec:theorie}

    Im folgenden Abschnitt werden die theoretischen Grundlagen für diesen Versuch erläutert.
    Dabei werden die Wechselwirkung von Gamma-Strahlung mit Materie
    sowie die Detektion der Strahlung beschrieben.

    Bei dem verwendeten Verfahren der Tomographie wird die zu untersuchende Probe mithilfe von Projektionen schichtweise betrachtet.

\subsection{Wechselwirkung von Gamma-Strahlung mit Materie}

    Gamma-Strahlung besteht aus hochenergetischen Photonen, die bei Zerfallsprozessen erzeugt werden.
    % NOTE: „Die Energiebereiche natürlicher Gamma- und Röntgenstrahlung überlappen sich, was eine gewisse Unschärfe dieser Begriffe zur Folge hat.“
    In diesem Fall wird der Zerfall des radioaktiven Isotops \ce{^137Cs} betrachtet,
    welches über einen $\beta$-Zerfall und anschließenden $\gamma$-Zerfall in das stabile \ce{^137Ba} übergeht.
    Dabei wird Strahlung der Energie $E_{\symup{\gamma}} \approx \SI{662}{\kilo\eV}$ \cite{caesium} frei.
    Dieser Prozess ist in \autoref{fig:theorie:zerfallsprozess} dargestellt.

    Wenn die $\gamma$-Strahlung nun auf Materie trifft,
    muss abhängig von der Energie der Strahlung zwischen folgenden Effekten unterschieden werden \cite{radioaktivitaet}.

    \begin{figure}
       \centering
       \includegraphics[width=0.5\textwidth]{content/img/zerfallsschema_Leifi.pdf}
       \caption{Zerfallsprozess von \ce{^137Cs} in das stabile \ce{^137Ba}. \cite{caesium}}
       \label{fig:theorie:zerfallsprozess}
    \end{figure}

    \subsubsection*{Photoeffekt} \label{sec:theorie:photoeffekt}
    Für niedrigere Energien ($\lessapprox \SI{100}{\kilo\electronvolt}$) tritt größtenteils der \textbf{Photoeffekt} auf.
    Dabei muss die Energie des Photons $E_{\symup{\gamma}} = hf$ der Bindungsenergie eines der Hüllenelektronen entsprechen.
    Das Photon gibt in diesem Fall seine gesamte Energie an das Elektron ab,
    welches aus dem Atom herausgelöst wird.
    Das Elektron behält dabei eine Energie $E_{e} = hf - E_\text{Bind}$.
    Der Wirkungsquerschnitt unterscheidet sich dabei,
    ob die Energie des Photons kleiner oder größer der relativistischen Ruheenergie des Elektrons ist.
    Es gilt
    \begin{equation*}
        \sigma_\text{Photo} \propto
        \begin{cases}
            \sfrac{Z^5}{E^{\sfrac{7}{2}}_{\symup{\gamma}}} & , E_{\symup{\gamma}} \ll m_{e} c^2 \\
            \sfrac{Z^5}{E_{\symup{\gamma}}} & , E_{\symup{\gamma}} \gg m_{e} c^2 \\
        \end{cases}
    \end{equation*}
    mit der Ordnungszahl $Z$.

    \subsubsection*{Compton-Effekt} \label{sec:theorie:comptoneffekt}
    Für zunehmende Energien dominiert der \textbf{Compton-Effekt}.
    Das Photon gibt in diesem Fall nicht mehr seine gesamte Energie ab,
    sondern stößt elastisch mit lose gebundenen oder freien Elektronen.
    Dabei gilt Energie- und Impulserhaltung,
    sodass das Photon einen Teil seiner Energie an das Elektron abgibt und beide eine Richtungsänderung erfahren.
    Die Energieabgabe des Photons ist abhängig vom Streuwinkel.
    Das Elektron kann eine maximale Energie erreichen,
    wenn für den Streuwinkel des Photons $\vartheta = \SI{0}{\degree}$ gilt.
    Im Energiespektrum findet sich an diesem Punkt die sogenannte \textit{Compton-Kante}.
    Für den Wirkungsquerschnitt gilt
    \begin{equation*}
        \sigma_\text{Compton} \propto \sfrac{Z}{E_{\symup{\gamma}}} \ .
    \end{equation*}


    \subsubsection*{Paarerzeugung} \label{sec:theorie:paarerzeugung}
    Für eine Energie $E_{\symup{\gamma}} \geq 2m_{e}c^2 = \SI{1.022}{\mega\eV}$ kann \textbf{Paarerzeugung} stattfinden.
    Dabei wandelt sich das Photon im Coulombfeld des Atoms in ein Elektron-Positron-Paar um.
    Dieser Prozess kann aufgrund der Erhaltungssätze nur mit dem Atomkern als weiterem Stoßpartner stattfinden.
    Die entstehenden Teilchen haben eine Energie von $E_{e^{+}} + E_{e^{-}} = E_{\symup{\gamma}} - 2m_{e}c^2$.
    Weiterhin gilt
    \begin{equation*}
        \sigma_\text{Paar} = Z^2 \ln(E_{\symup{\gamma}}) \ .
    \end{equation*}

    % ---

    Das sich ergebende Spektrum ist in \autoref{fig:theorie:spektrum} dargestellt.
    Es ist die in \autoref{sec:theorie:comptoneffekt} erwähnte Compton-Kante,
    sowie der Photopeak gezeigt.
    Der Photopeak entsteht durch den Photoeffekt (\autoref{sec:theorie:photoeffekt}),
    wenn die gesamte Energie der einfallenden Strahlung auf die Elektronen übertragen wird.
    Der Rückstrahlpeak entsteht durch den Compton-Effekt mit dem Material in der Umgebung.

    \begin{figure}
       \centering
       \includegraphics[width=\textwidth]{content/img/cs137-spektrum_Leifi.pdf}
       \caption{Spektrum der verwendeten $\ce{^{137}Cs}-Quelle$. \cite{caesium}}
       \label{fig:theorie:spektrum}
    \end{figure}

    In jedem der Fälle wird die Gamma-Strahlung vom Material absorbiert.
    Dabei ist die Absorption materialabhängig.
    Dies kann über den Absorptionskoeffizienten
    \begin{equation}
        \mu = \sigma \cdot \rho
        \label{eqn:theorie:absorptionskoeffizient}
    \end{equation}
    beschrieben werden,
    wobei $\rho$ die Dichte des Materials und $\sigma = \sigma_\text{Photo} + \sigma_\text{Compton}$ die Wirkungsquerschnitte darstellt.
    Paarerzeugung kann mit der verwendeten \ce{^137Cs}-Strahlungsquelle nicht stattfinden,
    da die entsprechende Energie zu gering ist.
    \autoref{tab:theorie:literaturwerte} zeigt die Wirkungsquerschnitte,
    Dichten und Absorptionskoeffizienten von verschiedenen Stoffen.
    \begin{table}[H]
        \centering
        \caption{Absorptionskoeffizient, Dichte und Wirkungsquerschnitte verschiedener Stoffe.
        Der Absorptionskoeffizient wird über \autoref{eqn:theorie:absorptionskoeffizient} berechnet.
        Die Dichten der Stoffe wurden \cite{dichten} entnommen,
        die Wirkungsquerschnitte aus \cite{crosssections}.
        Die Daten zu Holz stammen aus \cite{holz}.}
        \label{tab:theorie:literaturwerte}
        \expandableinput{build/tab/mu.tex}
    \end{table}
    Abhängig vom Material des Stoffes wird die Gammastrahlung bei der Absorption abgeschwächt.
    Dabei fällt die Intensität exponentiell mit der Weglänge ab.
    Für eine Aneinanderreihung von Wegstücken $x_i$
    mit unterschiedlichen Absorptionskoeffizienten $\mu_i$
    ergibt sich für die Intensität
    \begin{align}
        I &= I_0 \exp\left(-\sum_{i=1}^N \mu_i x_i \right) \\
        \iff \underbrace{ \ln\left(\frac{I_0}{I}\right) }_{J_k} &= \sum_{i=1}^N \mu_i x_i \ .
        \label{eqn:theorie:intensitaet2}
    \end{align}
    Wird die Vermessung derselben $\mu_i$ mit anderen Schnittgeraden und somit anderen $x_i$ wiederholt,
    kann aus den Gleichungen der Form \eqref{eqn:theorie:intensitaet2} ein lineares Gleichungssystem
    \begin{equation}
        \vec{J} = \symbf{A} \cdot \vec{\mu}
    \end{equation}
    aufgestellt werden,
    wobei $\vec{J}$ aus den Inhomogenitäten
    und $\symbf{A}$ aus den $x_{i,k}$
    zusammengesetzt ist.

    Für die gewählten Projektionen durch die mittlere $3\times3$-Schnittebene des Würfels,
    die in \autoref{fig:theorie:projektionen} gezeigt sind,
    ergibt sich mit \autoref{eqn:theorie:intensitaet2} das Gleichungssystem
    \begin{equation}
        \label{eqn:theorie:intensitaet3}
        \begin{pmatrix}
            J_{1|4|7} \\
            J_{2|5|8} \\
            J_{3|6|9} \\
            J_{7|8|9} \\
            J_{4|5|6} \\
            J_{1|2|3} \\
            J_{2/4}  \\
            J_{3/5/7} \\
            J_{6/8}  \\
            J_{4/8}  \\
            J_{1/5/9} \\
            J_{2/6}
        \end{pmatrix}
        = d \cdot
        \begin{pmatrix}
            1        & 0        & 0        & 1        & 0        & 0        & 1        & 0        & 0        \\
            0        & 1        & 0        & 0        & 1        & 0        & 0        & 1        & 0        \\
            0        & 0        & 1        & 0        & 0        & 1        & 0        & 0        & 1        \\
            0        & 0        & 0        & 0        & 0        & 0        & 1        & 1        & 1        \\
            0        & 0        & 0        & 1        & 1        & 1        & 0        & 0        & 0        \\
            1        & 1        & 1        & 0        & 0        & 0        & 0        & 0        & 0        \\
            0        & \sqrt{2} & 0        & \sqrt{2} & 0        & 0        & 0        & 0        & 0        \\
            0        & 0        & \sqrt{2} & 0        & \sqrt{2} & 0        & \sqrt{2} & 0        & 0        \\
            0        & 0        & 0        & 0        & 0        & \sqrt{2} & 0        & \sqrt{2} & 0        \\
            0        & 0        & 0        & \sqrt{2} & 0        & 0        & 0        & \sqrt{2} & 0        \\
            \sqrt{2} & 0        & 0        & 0        & \sqrt{2} & 0        & 0        & 0        & \sqrt{2} \\
            0        & \sqrt{2} & 0        & 0        & 0        & \sqrt{2} & 0        & 0        & 0
        \end{pmatrix}
        \cdot
        \begin{pmatrix}
            \mu_1 \\
            \mu_2 \\
            \mu_3 \\
            \mu_4 \\
            \mu_5 \\
            \mu_6 \\
            \mu_7 \\
            \mu_8 \\
            \mu_9
        \end{pmatrix}
        \ ,
    \end{equation}
    wobei $d = \SI{1}{\centi\meter}$ die Kantenlänge eines Elementarwürfels ist.

    \begin{figure}
        \centering
        \expandableinput{content/img/würfel.tex}
        \caption{Projektionen durch die mittlere $3\times3$-Schnittebene.}
        \label{fig:theorie:projektionen}
    \end{figure}


\subsection{Detektion der Gammastrahlung}

    Zur Detektion der Gamma-Strahlung wird ein \textit{Szintillator} genutzt.
    In diesem werden durch die auftreffende Strahlung Atome angeregt,
    die Licht emittieren.
    Dieses Licht wird mithilfe eines Photomultipliers verstärkt und in Photostrom umgewandelt,
    sodass Lichtimpulse gemessen werden können.
    Da das entstehende Licht proportional zur Energie der einfallenden Strahlung ist,
    kann mithilfe des Szintillators ein Energiespektrum gemessen werden.
    Zudem wird ein \textit{Diskriminator} verwendet.
    Dieser dient dazu,
    Hintergrundsignal auszusortieren,
    welches durch den Photomultiplier oder andere Detektoren entsteht.
    Der Diskriminator legt dazu eine Energieschwelle fest.
    Gemessene Signale,
    deren Energie unter dieser Schwelle liegt,
    werden nicht registriert.
    Mithilfe eines \textit{Multichannel analyzers} werden die gemessenen Impulse ihrer Energie nach geordnet.
