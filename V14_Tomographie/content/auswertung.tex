\section{Auswertung}
\label{sec:auswertung}

\subsection{Nullmessung}
Um in den folgenden Untersuchungen auf eine Nullrate $I_0$ zurückgreifen zu können,
wird diese anfangs anhand vom leeren \textbf{Würfel 1} vermessen.
Damit soll der Effekt der homogenen Aluminiumwand aller Würfel ausreichend berücksichtigt sein.

\autoref{tab:auswertung:wuerfel1} listet die Nullraten für die drei relevanten Kategorien von Projektionen
(Hauptdiagonale, Nebendiagonale und Parallele)
auf.
Zusätzlich ist das nach einer Messzeit von $T = \SI{240}{\second}$ resultierende Energiespektrum in \autoref{plt:auswertung:spektrum} dargestellt.

\begin{table}[H]
    \centering
    \caption{TODO.}
    \label{tab:auswertung:wuerfel1}
    \expandableinput{build/tab/wuerfel1.tex}
\end{table}

\begin{figure}
    \centering
    \includegraphics[width=0.8\textwidth]{build/plt/spektrum.pdf}
    \caption{Spektrum TODO.}
    \label{plt:auswertung:spektrum}
\end{figure}


\subsection{Untersuchung homogener Würfel}
Anhand von \autoref{eqn:auswertung:foo}…
\begin{equation}
    \mu = \frac{1}{d} \ln \left( \frac{I_0}{I} \right)
    \label{eqn:auswertung:foo}
\end{equation}


\subsubsection{\textbf{Würfel 2}}
\begin{table}[H]
    \centering
    \caption{TODO.}
    \label{tab:auswertung:wuerfel2}
    \expandableinput{build/tab/wuerfel2.tex}
\end{table}

\subsubsection{\textbf{Würfel 3}}
\begin{table}[H]
    \centering
    \caption{TODO.}
    \label{tab:auswertung:wuerfel3}
    \expandableinput{build/tab/wuerfel3.tex}
\end{table}


\subsection{Untersuchung eines inhomogenen Würfels}

\begin{table}[H]
    \centering
    \caption{TODO.}
    \label{tab:auswertung:wuerfel4}
    \expandableinput{build/tab/wuerfel4.tex}
\end{table}

\begin{figure}
    \centering
    \includegraphics[width=0.7\textwidth]{build/plt/wuerfel4.pdf}
    \caption{Visualisierung Würfel4 TODO.}
    \label{plt:auswertung:wuerfel4}
\end{figure}
