\section{Auswertung}
\label{sec:auswertung}

\subsection{Nullmessung}
Um in den folgenden Untersuchungen auf eine Nullrate $I_0$ zurückgreifen zu können,
wird diese anfangs anhand von \textbf{Würfel 1} vermessen.
Damit soll der Effekt der homogenen Aluminiumwand des Würfels ausreichend berücksichtigt sein.

Das nach einer Messzeit von $T = \SI{240}{\second}$ resultierende Energiespektrum ist in \autoref{plt:auswertung:spektrum} dargestellt.

% TODO: Messwerte-Tabelle

\begin{figure}
    \centering
    \includegraphics[width=0.5\textwidth]{build/plt/spektrum.pdf}
    \caption{Spektrum TODO.}
    \label{plt:auswertung:spektrum}
\end{figure}


\subsection{Untersuchung homogener Würfel}
\subsubsection{\textbf{Würfel 2}}
\begin{table}[H]
    \centering
    \caption{TODO.}
    \label{tab:auswertung:wuerfel2}
    \expandableinput{build/tab/wuerfel2.tex}
\end{table}

\subsubsection{\textbf{Würfel 3}}
\begin{table}[H]
    \centering
    \caption{TODO.}
    \label{tab:auswertung:wuerfel3}
    \expandableinput{build/tab/wuerfel3.tex}
\end{table}


\subsection{Untersuchung eines inhomogenen Würfels}

\begin{figure}
    \centering
    \includegraphics[width=0.7\textwidth]{build/plt/wuerfel4.pdf}
    \caption{Visualisierung Würfel4 TODO.}
    \label{plt:auswertung:wuerfel4}
\end{figure}
