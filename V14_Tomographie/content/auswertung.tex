\section{Auswertung}
\label{sec:auswertung}
% TODO: Poisson-Fehler nochmal erwähnen?

\subsection{Nullmessung und Spektrum}
% Eigentlich sollte das Spektrum wohl ohne Würfel im Strahlengang aufgenommen werden, der Unterschied dürfte aber marginal sein.

Um in den folgenden Untersuchungen auf eine Nullrate $I_0$ zurückgreifen zu können,
wird diese anfangs anhand vom leeren \textbf{Würfel 1} vermessen.
Damit soll der Effekt der homogenen Aluminiumwand aller Würfel ausreichend berücksichtigt sein.

\autoref{tab:auswertung:wuerfel1} listet die Nullraten für die drei relevanten Kategorien von Projektionen
(Hauptdiagonale, Nebendiagonale und Parallele)
auf.
Beispielhaft ist ein nach einer Messzeit von $T = \SI{240}{\second}$ resultierende Energiespektrum in \autoref{plt:auswertung:spektrum} dargestellt.
Im Sinne einer aussagekräftigen Abbildung sind die energiereichsten \SI{1}{\percent} aller Zählungen ausgelassen.

\begin{table}[H]
    \centering
    \caption{Zählraten für verschiedene Projektionen durch \textbf{Würfel 1}.}
    \label{tab:auswertung:wuerfel1}
    \expandableinput{build/tab/wuerfel1.tex}
\end{table}

\begin{figure}
    \centering
    \includegraphics[width=0.8\textwidth]{build/plt/spektrum.pdf}
    \caption{Energiespektrum der Projektion \enquote{4|5|6}.}
    \label{plt:auswertung:spektrum}
\end{figure}


\subsection{Untersuchung homogener Würfel}
Aus \autoref{eqn:theorie:intensitaet2} ist ersichtlich,
dass für den Absorptionskoeffizienten homogener Materialien bei Weglänge $x$ folgende Bestimmungsgleichung gilt:
\begin{equation*}
    \mu = \frac{1}{x} \ln \left( \frac{I_0}{I} \right) \ .
\end{equation*}
Über die so je Projektion ermittelten Absorptionskoeffizienten kann dann ein Mittelwert gebildet werden.


\subsubsection{\textbf{Würfel 2}}
\begin{table}[H]
    \centering
    \caption{Zählraten und Absorptionskoeffizienten für verschiedene Projektionen durch \textbf{Würfel 2}.}
    \label{tab:auswertung:wuerfel2}
    \expandableinput{build/tab/wuerfel2.tex}
\end{table}
Im Mittel ergibt sich $\mu = \SI{0.049 \pm 0.001}{\per\centi\meter}$
bei einer relativen Unsicherheit von \SI{2.4}{\percent}.
Die geringste Abweichung besteht damit zum Literaturwert von Delrin (siehe \autoref{tab:theorie:literaturwerte});
sie beträgt \SI{58.07}{\percent}.


\subsubsection{\textbf{Würfel 3}}
\begin{table}[H]
    \centering
    \caption{Zählraten und Absorptionskoeffizienten für verschiedene Projektionen durch \textbf{Würfel 3}.}
    \label{tab:auswertung:wuerfel3}
    \expandableinput{build/tab/wuerfel3.tex}
\end{table}
Im Mittel ergibt sich $\mu = \SI{1.033 \pm 0.003}{\per\centi\meter}$
bei einer relativen Unsicherheit von \SI{1.17}{\percent}.
Die geringste Abweichung besteht damit zum Literaturwert von Blei (siehe \autoref{tab:theorie:literaturwerte});
sie beträgt \SI{12.02}{\percent}.


\subsection{Untersuchung eines inhomogenen Würfels}

\begin{table}[H]
    \centering
    \caption{Zählraten für verschiedene Projektionen durch die mittlere Ebene von \textbf{Würfel 4}.}
    \label{tab:auswertung:wuerfel4}
    \expandableinput{build/tab/wuerfel4.tex}
\end{table}

\begin{table}[H]
    \centering
    \caption{Durch Lösen von XY bestimmte Werte für $\mu_i$ zu \textbf{Würfel 4}.}
    \label{tab:auswertung:wuerfel4_mu}
    \expandableinput{build/tab/wuerfel4_mu.tex}
\end{table}

\begin{figure}
    \centering
    \includegraphics[width=0.7\textwidth]{build/plt/wuerfel4.pdf}
    \caption{Visualisierung der Absorptionskoeffizienten in der mittleren Ebene von \textbf{Würfel 4}.}
    \label{plt:auswertung:wuerfel4}
\end{figure}
