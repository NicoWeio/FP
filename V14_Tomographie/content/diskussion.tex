\section{Diskussion}
\label{sec:diskussion}

\subsection{Abweichungen}
    Da die tatsächliche Zusammensetzung der Würfel nicht bekannt ist,
    können lediglich über Abweichungen innerhalb der Messungen gesicherte Aussagen getroffen werden.

    Durch ausreichende Messzeiten wurde
    eine relative Unsicherheit der Impulsrate von $< \SI{3}{\percent}$ sichergestellt,
    welche jedoch nicht genügt,
    um die Abweichungen zwischen äquivalenten Messungen der homogenen Würfel zu erklären,
    welche das gleiche theoretische Ergebnis liefern,
    wie beispielsweise \enquote{1|2|3} und \enquote{4|5|6}.

    Mit Abweichungen von
    \SI{19.43}{\percent} (Holz) bei \textbf{Würfel 2} beziehungsweise
    \SI{12.04}{\percent} (Blei) bei \textbf{Würfel 3}
    sind die ermittelten Materialien der homogenen Würfel zwar nicht einwandfrei bestätigt,
    aber zumindest plausibel,
    zumal die nächst-größeren Abweichungen
    \SI{75.4}{\percent} (Aluminium) beziehungsweise
    \SI{69.6}{\percent} (Messing)
    betragen.
    Es sei darauf hingewiesen, dass zur Bestimmung des Materials die \emph{absoluten} Abweichungen minimiert wurden,
    hier jedoch im Sinne der Vergleichbarkeit die \emph{relativen} Abweichungen angegeben wurden.

    Die Absorptionskoeffizienten der Elementarwürfel von \textbf{Würfel 4} folgen einem klaren Muster,
    wonach die mittlere vertikale Reihe ($\mu_2, \mu_5, \mu_8$) aus einem stark absorbierenden Material
    und die übrigen Elementarwürfel aus einem oder mehreren eher durchsichtigen Materialien bestehen,
    wie auch \autoref{plt:auswertung:wuerfel4} zu erkennen ist.


\subsection{Mögliche Fehlerquellen}

    Ein möglicher Grund für die Abweichungen liegt in der Einjustierung der Projektionen.
    Der Strahl war nicht sichtbar,
    deshalb diente als Anhaltspunkt nur ein Metallstab von nicht unerheblicher Dicke,
    der sich parallel zur Strahlrichtung über dem Würfel befand.
    Zur Einjustierung musste also immer von oben auf den Stab geschaut werden.
    Insbesondere bei der Vermessung der Diagonalen,
    bei denen eine laterale Verschiebung zu einer Veränderung der Weglänge im Material führt,
    können dadurch Abweichungen verursacht worden sein.

    Zudem ist der Versuch als Zählexperiment stets mit einer statistischen Unsicherheit (Poisson) belegt,
    welche durch längere Messzeiten zwar kleiner wird,
    aber aufgrund der begrenzten Zeit zur Durchführung
    noch von Bedeutung ist.

    Ähnlich zur Impulszahl ist auch die Anzahl an vermessenen Projektionen
    ausschlaggebend für die statistische Unsicherheit.
    Hierbei ist jedoch zusätzlich eine Kompensation von systematischen Abweichungen denkbar.
    Insofern hätte das Hinzunehmen weiterer Projektionen,
    beispielsweise durch einzelne Elementarwürfel an den Ecken,
    genauere Ergebnisse ermöglicht.
