\section{Durchführung}
\label{sec:durchfuehrung}

    Im Folgenden werden der Aufbau und die Durchführung der Messung beschrieben.

    Der Messaufbau ist in \autoref{fig:durchfuehrung:aufbau} dargestellt.
    \begin{figure}
       \centering
       \includegraphics[width=0.7\textwidth]{content/img/Abb_1.pdf}
       \caption{Verwendeter Aufbau.
       Zu sehen sind die Strahlungsquelle,
       welche kollimiert auf den Probenwürfel trifft.
       Hinter dem Würfel befindet sich der Natriumiodid-Detektor. \cite{versuchsanleitung}}
       \label{fig:durchfuehrung:aufbau}
    \end{figure}
    Der gesamte Aufbau ist zur Abschirmung mit Bleiblöcken umgeben.
    Als Strahlungsquelle wird,
    wie in \autoref{sec:theorie} beschrieben,
    eine $\ce{^{137}Cs}$-Quelle verwendet,
    die zu Beginn eingesetzt wird.
    Die Strahlung wird auf den Probenwürfel gerichtet.
    % NOTE: I = Iod; früher J, so auch in der Versuchsanleitung
    Hinter diesem befindet sich ein $\ce{NaI}$-Szintillator,
    der mit einem Photomultiplier verbunden ist.
    Am Photomultiplier ist eine Spannung von \SI{500}{\volt} angelegt.
    Dieser ist wiederum an den Computer angeschlossen.
    Für die Messung wird das Programm \enquote{MAESTRO} verwendet.
    Bei der Messung muss darauf geachtet werden,
    dass genügend Impulse gemessen werden,
    um eine relative Unsicherheit von \SI{3}{\percent} nicht zu überschreiten.
    Es gilt
    \begin{equation*}
        \frac{\sqrt{N}}{N} \stackrel{!}{<} \SI{3}{\percent}
        \iff N \stackrel{!}{>} 1111.\bar{1} \approx 1112
    \end{equation*}
    mit der absoluten Unsicherheit $\sqrt{N}$ einer Poisson-verteilten Zählrate.

    Für die Messung wird nun zuerst \textbf{Würfel 1} in den Strahlengang eingebaut.
    Dieser besteht im Inneren nur aus Luft.
    Es wird eine Nullmessung der $I_0$ für alle betrachteten Kategorien von Projektionen,
    konkret \enquote{4|5|6}, \enquote{1/5/9} und \enquote{2/6},
    über ein Zeitintervall von \SI{240}{\second} durchgeführt.
    Dabei werden die Anzahl der Impulse innerhalb des Photopeaks (\autoref{fig:theorie:spektrum}) gezählt und das Spektrum gespeichert.
    Anschließend wird der \textbf{Würfel 2} in den Strahlengang gebracht.
    Es werden nun nacheinander alle der in \autoref{fig:theorie:projektionen} gezeigten Projektionen eingestellt und jeweils die Anzahl der Impulse gemessen.
    % NOTE: Man hätte sich bei den homogenen Würfeln auch mit weniger Projektionen zufrieden geben können.
    Dies wird für \textbf{Würfel 3} und \textbf{Würfel 4} wiederholt.
    Während der Einjustierung der Projektionen und dem Auswechseln der Würfel muss darauf geachtet werden,
    dass nicht in den Strahlengang gefasst wird.
    Außerdem gelten die Regeln des Strahlenschutzes.

