\section{Diskussion}
\label{sec:diskussion}

\subsection{Abweichungen}
Die \hyperref[sec:auswertung:energiekalibrierung]{Energiekalibrierung},
    welche essenziell für die weitere Auswertung ist,
zeigt nur geringfügige Unsicherheiten in der Regressionsgeraden (siehe \autoref{fig:plt:energy_calibration}).

Zur \hyperref[sec:auswertung:effizienz]{Vollenergienachweiswahrscheinlichkeit} liegt kein Vergleichswert vor,
der Fit in \autoref{fig:plt:effizienz} zeigt jedoch eine gute Übereinstimmung mit dem Modell eines polynomiellen Abfalls mit moderaten Abweichungen.

Im Rahmen der \hyperref[sec:auswertung:Cs_137]{Untersuchung des Gammaspektrums von \ce{^137Cs}}
wurden die Lagen von Compton-Kante und Photopeak bestimmt.
Die geschätzte Lage der Compton-Kante weicht um \SI{1.54}{\percent} vom theoretischen Wert ab,
weist jedoch eine relativ große Unsicherheit auf.
Der Photopeak konnte hingegen mit einer geringen Abweichung von \SI{0.03}{\percent} bestimmt werden.
Das Verhältnis von Zehntelwertsbreite zur Halbwertsbreite des Photopeaks
weicht um \SI{4.24}{\percent} vom für eine ideale Gaußverteilung erwarteten Wert ab.

Für die \hyperref[sec:auswertung:Ba_133]{Aktivität des \ce{^133Ba}-Strahlers} ist ebenfalls kein Vergleichswert vorhanden.
Der Strahler konnte jedoch in Abgrenzung zu \ce{^125Sb} eindeutig identifiziert werden.

Das \hyperref[sec:auswertung:uranstein]{Energiespektrum des Uran-Steins} weist eine Reihe von Peaks auf,
die gut mit \ce{^214Bi} übereinstimmen.
Aufgrund der Vielzahl an Zerfallsprodukten von Uran ist die Zuordnung weniger intensiver Peaks jedoch weniger eindeutig.
% Auch hier ist kein Vergleichswert vorhanden.


\subsection{Mögliche Fehlerquellen}
An mehreren Stellen wurden nur die deutlichsten Peaks verwendet,
beispielsweise bei der \hyperref[sec:auswertung:energiekalibrierung]{Energiekalibrierung}.
Dies ermöglicht eine problemlose Zuordnung anhand der Intensitäten,
schmälert jedoch zugleich die Datengrundlage für die Regression beziehungsweise die Bestimmung der Effizienz oder der Aktivität.

Durch eine Vielzahl an Störeffekten wird die Qualität der gemessenen Spektren beeinträchtigt.
Dazu zählen
    die statistische Schwankung der Elektron-Loch-Paarzahl,
    das Rauschen des Leckstromes,
        % der infolge der Eigenleitung und Restverunreinigungen im Kristall
        % und der hohen Saugspannung entsteht,
    das Rauschen des an den Detektor angeschlossenen Verstärkers
    und weitere.
Bei der Wahl der Saugspannung müssen zudem Kompromisse eingegangen werden.

Da es sich im Kern um ein Zählexperiment handelt,
kann eine längere Messdauer zu einer besseren Statistik führen.

% Schließlich muss geprüft werden, ob der Untergrund effektiv eliminiert wurde.

% - Abstand Probe-Detektor?
