\section{Auswertung}
\label{sec:auswertung}
Die Berechnung der Unsicherheiten wird im Folgenden vom Python-Paket \texttt{uncertainties} übernommen,
welches auf der \emph{Gaußschen Fehlerfortpflanzung} basiert.

Der Untergrund wird implizit von den Messwerten subtrahiert,
wobei das Verhältnis der Messdauern berücksichtigt wird:
\begin{align*}
    I' &= I - I_\text{bg} \\
    \leftrightarrow
    \frac{N'}{T} &= \frac{N}{T} - \frac{N_\text{bg}}{T_\text{bg}} \\
    \rightarrow
    N' &= N - \frac{T_{\phantom{\text{bg}}}}{T_\text{bg}} N_\text{bg}
\end{align*}


\subsection{Energiekalibrierung} \label{sec:auswertung:energiekalibrierung}
Der Vielkanalanalysator wird anhand des Spektrums einer \ce{^152Eu}-Quelle kalibriert.
Dazu werden die Positionen der Peaks im
    über eine Zeit von $\SI{56.5}{\minute}$ gemessenen
Spektrum (siehe \autoref{fig:plt:spektrum_Eu_152})
mit den bekannten Energien der \ce{^152Eu}-Quelle abgeglichen.
Jeder Peak wird durch den Fit einer Gaußfunktion angenähert:
\begin{equation}
    N(x) = a \exp\left(-\frac{(x - x_0)^2}{2 \sigma^2}\right) + N_0 \, .
    \label{eqn:gauss}
\end{equation}
Dabei entsprechen
    $a$ einem Skalierungsfaktor, % es ist ja nicht direkt die Amplitude…
    $x_0$ der Position (als Kanalnummer)
    und $\sigma$ der Breite
des Peaks.
Der Parameter $N_0$ beschreibt den Untergrund.

\autoref{tab:1_energiekalibrierung} zeigt für jeden Peak
    die Fit-Parameter sowie die zugeordnete Energie aus der Literatur \cite{lara}.

\begin{figure}
    \centering
    \includegraphics[width=\textwidth]{build/plt/spektrum_152-Eu.pdf}
    \caption{
        Energiespektrum der \ce{^152Eu}-Quelle mit eingezeichneten Peaks.
        Die grünen Linien markieren die Position der Theoriewerte nach der Kalibrierung
        und sind entsprechend ihrer erwarteten Intensitäten unterschiedlich opak.
    }
    \label{fig:plt:spektrum_Eu_152}
\end{figure}

\begin{table}
    \centering
    \caption{Fit-Parameter sowie zugeordnete Energie aus der Literatur \cite{lara} je Peak.}
    \label{tab:1_energiekalibrierung}
    \expandableinput{build/tab/1_energiekalibrierung.tex}
\end{table}

Durch lineare Regression kann schließlich die Abhängigkeit der Energie von der Kanalnummer bestimmt werden.
Für die Funktion
\begin{equation*}
    E(x) = m x + n
\end{equation*}
ergeben sich die Parameter
\begin{align*}
    m &= \SI{0.20679 \pm 0.00001}{\kilo\electronvolt} \\
    n &= \SI{-1.34 \pm 0.04}{\kilo\electronvolt} \, .
\end{align*}
Ein Plot dieser Regressionsrechnung ist in \autoref{fig:plt:energy_calibration} einsehbar.

\begin{figure}
    \centering
    \includegraphics[width=\textwidth]{build/plt/energy_calibration.pdf}
    \caption{Zugeordnete Energien in Abhängigkeit der Kanalnummern mit Regressionsgerade.}
    \label{fig:plt:energy_calibration}
\end{figure}


\FloatBarrier % temporär
\subsection{Bestimmung der Vollenergienachweiswahrscheinlichkeit} \label{sec:auswertung:effizienz}
Um die Vollenergienachweiswahrscheinlichkeit anhand des zuvor betrachteten \ce{^152Eu}-Spektrums zu bestimmen,
wird zunächst die Aktivität der Probe berechnet.
Am Tag der Herstellung (01.10.2000) betrug diese $A_0 = \SI{4130(60)}{\becquerel}$.
Bei einer Halbwertszeit von $t_{1/2} = \SI{13.522(16)}{a}$ \cite{lara} entspricht dies einer Aktivität von
\begin{equation*}
    A = A_0 \cdot \exp\left(-\frac{\ln(2)}{t_{1/2}} t\right)
    = \SI{1326.40(1935)}{\becquerel}
\end{equation*}
am Tag der Messung,
also nach $t = \SI{22.16}{a}$.

Der Raumwinkel $\Omega$ wird anhand der Formel
\begin{equation*}
    \Omega = 2\pi \left( 1 - \frac{d}{\sqrt{d^2 + r^2}} \right)
\end{equation*}
bestimmt.
Mit
    einem Abstand von $d = \SI{8.5}{\centi\meter}$ zwischen Probe und Detektor
        (
            $\SI{7}{\centi\meter}$ Probe–Schutzhaube
            und
            $\SI{1.5}{\centi\meter}$ Schutzhaube–Detektor
        )
    sowie einem Detektorradius von $r = \SI{2.25}{\centi\meter}$
ergibt sich ein Raumwinkel von
$\Omega = 4\pi \cdot \num{0.0166} = \num{0.2092}$.

Die Inhalte $Z$ der Peaks werden anhand der Parameter des Gauß-Fits aus \autoref{tab:1_energiekalibrierung} bestimmt:
\begin{equation}
    Z = a \sqrt{2\pi \sigma^2} \, .
    \label{eqn:peakinhalt}
\end{equation}

\phantomsection \label{sec:auswertung:effizienz:effizienz}
Die Effizienz $Q$ wird dann über \autoref{eqn:effizienz} berechnet.
Um die Energieabhängigkeit selbiger zu beschreiben,
wird eine Potenzfunktion der Form
\begin{equation*}
    Q(E) = p \cdot \left(\frac{E}{\SI{1}{\kilo\electronvolt}}\right)^q
\end{equation*}
mithilfe von \texttt{scipy.optimize.curve\_fit} an die in \autoref{tab:2_effizienz} aufgeführten Daten angepasst,
wie in \autoref{fig:plt:effizienz} gezeigt.
Die resultierenden Parameter sind
\begin{align*}
    p &= \num{49.31 \pm 10.02} \\
    q &= \num{-0.95 \pm 0.03} \, .
\end{align*}
Die Unsicherheit dieser konnte durch \texttt{curve\_fit} nicht geschätzt werden.

\begin{figure}
    \centering
    \includegraphics[width=\textwidth]{build/plt/effizienz.pdf}
    \caption{Messwerte und Fit der Effizienz $Q$ in Abhängigkeit der Energie $E$.}
    \label{fig:plt:effizienz}
\end{figure}

\begin{table}
    \centering
    \caption{
        Berechnete Effizienzen $Q$ der zuvor betrachteten Peaks.
        Zusätzlich sind die der Berechnung zugrundeliegenden
            Inhalte $Z$ und
            Intensitäten (/ Emissionswahrscheinlichkeiten) $W$ \cite{lara}
        angegeben.
    }
    \label{tab:2_effizienz}
    \expandableinput{build/tab/2_effizienz.tex}
\end{table}


\FloatBarrier % temporär
\subsection{Untersuchung des monochromatischen Gammaspektrums von \ce{^137Cs}} \label{sec:auswertung:Cs_137}
Als Nächstes wird das
    in \autoref{fig:plt:spektrum_Cs_137} dargestellte
Spektrum einer \ce{^137Cs}-Probe untersucht.
Rückstreu- und Photopeak werden erneut mit Gauß-Kurven approximiert
    (siehe \autoref{eqn:gauss}).
Die resultierenden Parameter
    sowie die mithilfe von \autoref{eqn:peakinhalt} berechneten Inhalte $Z$
sind in \autoref{tab:3_Cs_137} aufgeführt.

\begin{table}
    \centering
    \caption{
        Fit-Parameter der Gauß-Approximationen von Rückstreu- und Photopeak
        sowie die daraus berechneten Energien und Inhalte.
    }
    \label{tab:3_Cs_137}
    \expandableinput{build/tab/3_Cs-137.tex}
\end{table}

Die Energie des Strahlers entspricht der Lage des Photopeaks (Vollenergielinie)
und beträgt \SI{661.43 \pm 0.06}{\kilo\electronvolt}.
Zur Bestimmung der Halbwertsbreite und Zehntelwertsbreite desselben wird auf Regressionsgeraden ($N(E) = m·E+n$) zurückgegriffen.
Deren Parameter lauten
\begin{align*}
m_\text{FWHM, links} &= \SI{6.0(5)e+02}{\per\kilo\electronvolt} \\
n_\text{FWHM, links} &= \num{-3.93(33)e+05} \\
m_\text{FWHM, rechts} &= \SI{-6.9(4)e+02}{\per\kilo\electronvolt} \\
n_\text{FWHM, rechts} &= \num{4.59(25)e+05} \\
 \\
m_\text{FWTM, links} &= \SI{550(32)}{\per\kilo\electronvolt} \\
n_\text{FWTM, links} &= \num{-3.63(21)e+05} \\
m_\text{FWTM, rechts} &= \SI{-5.4(5)e+02}{\per\kilo\electronvolt} \\
n_\text{FWTM, rechts} &= \num{3.61(31)e+05} \, .
\end{align*}
Daraus ergeben sich
\begin{align}
    \text{FWHM} = \SI{2.13 \pm 92.47}{\kilo\electronvolt} \\
    \text{FWTM} = \SI{4.05 \pm 98.52}{\kilo\electronvolt}
\end{align}
sowie das Verhältnis
\begin{equation}
    \frac{\text{FWTM}}{\text{FWHM}} = \num{1.90 \pm 94.42} \, .
\end{equation}

% NOTE: Vergleich mit dem optimalen Verhältnis in der Diskussion

% \begin{equation*}
%     \text{FWHM} = 2 \sigma \sqrt{2 \ln(2)}
%     = \SI{0.74 \pm 0.05}{\kilo\electronvolt}
% \end{equation*}
% ebenso wie die Zehntelwertsbreite
% \begin{equation*}
%     \text{FWTM} = 2 \sigma \sqrt{2 \ln(10)}
%     = \SI{2.44 \pm 0.02}{\kilo\electronvolt} \, .
% \end{equation*}

Die Lage der Compton-Kante im gemessenen Spektrum wird anhand des Schnittpunkts zweier Regressionsgeraden zu beiden Seiten der ungefähren Position der Kante bestimmt,
wie in \autoref{fig:plt:compton_kante} gezeigt.
Die resultierenden Parameter sind
\begin{align*}
    m_\text{links} &= \SI{0.13 \pm 0.01}{\per\kilo\electronvolt} \\
    n_\text{links} &= \num{-28.90 \pm 3.46} \\
    m_\text{rechts} &= \SI{-0.90 \pm 0.09}{\per\kilo\electronvolt} \\
    n_\text{rechts} &= \num{457.29 \pm 40.84} \, .
\end{align*}

Um den Inhalt des Compton-Kontinuums abzuschätzen,
% wird die Formel für den […] differentiellen Wirkungsquerschnitt
wird der aus \autoref{eqn:klein_nishina} abgeleitete differentielle Wirkungsquerschnitt
% COULDDO: Formel angeben 😅
um einen Skalierungsfaktor ergänzt und
an die Messwerte angepasst,
    wie in \autoref{fig:plt:klein_nishina} gezeigt.
Hierbei wird nur der weitgehend unverfälschte Bereich zwischen \SI{350}{\kilo\electronvolt} und der Compton-Kante berücksichtigt.
Somit ergibt sich für den geschätzten Inhalt des Compton-Kontinuums
\begin{equation*}
    Z_\text{Compton} = \SI{46319}{\kilo\electronvolt} \, .
\end{equation*}

% Vergleichen Sie die gemessenen Werte für die Compton-Kante und die
% Rückstreulinie mit den aus der Energie berechneten Werten.

% Bestimmen
%  Sie aus der Länge des Detektorkristalls und dem
% Extinktionskoeffizienten
%  für Photo- und Compton-Effekt die
% Absorptionswahrscheinlichkeit der 137Cs-Quanten für den hier vorliegenden
% Detektor. Vergleichen Sie Ihre Ergebnisse mit den Inhalten der Photolinie und des
% Compton-Kontinuums. Welchen Schluss muss man aus dem Vergleich ziehen? Wie
% kommt vermutlich die „Photolinie“ zustande?

\begin{figure}
    \centering
    \includegraphics[width=\textwidth]{build/plt/spektrum_137-Cs.pdf}
    \caption{Energiespektrum der \ce{^137Cs}-Quelle mit eingezeichnetem Rückstreu- und Photopeak.}
    \label{fig:plt:spektrum_Cs_137}
\end{figure}

\begin{figure}
    \centering
    \includegraphics[width=\textwidth]{build/plt/compton-kante.pdf}
    \caption{Approximation der Compton-Kante durch Bestimmung des Schnittpunkts zweier Regressionsgeraden.}
    \label{fig:plt:compton_kante}
\end{figure}

\begin{figure}
    \centering
    \includegraphics[width=\textwidth]{build/plt/photopeak_fwhm.pdf}
    \caption{Approximation der Halbwertsbreite durch Bestimmung des Schnittpunkts zweier Regressionsgeraden.}
    \label{fig:plt:photopeak_fwhm}
\end{figure}

\begin{figure}
    \centering
    \includegraphics[width=\textwidth]{build/plt/photopeak_fwtm.pdf}
    \caption{Approximation der Zehntelwertsbreite durch Bestimmung des Schnittpunkts zweier Regressionsgeraden.}
    \label{fig:plt:photopeak_fwtm}
\end{figure}

\begin{figure}
    \centering
    \includegraphics[width=\textwidth]{build/plt/klein-nishina.pdf}
    \caption{Fit des differentiellen Klein-Nishina-Wirkungsquerschnitts an die Messwerte.}
    \label{fig:plt:klein_nishina}
\end{figure}


\FloatBarrier
\subsection{Bestimmung der Aktivität von \ce{^133Ba}} \label{sec:auswertung:Ba_133}
Um zu entscheiden,
ob es sich bei der vorliegenden Messung um eine \ce{^133Ba}- oder \ce{^125Sb}-Quelle handelt,
kann das Energiespektrum mit den jeweiligen Literaturwerten verglichen werden.
Wie in \autoref{fig:plt:spektrum_Ba_133} zu sehen ist,
können die fünf größten Peaks eindeutig mit den Emissionslinien von \ce{^133Ba} identifiziert werden.

\begin{figure}
    \centering
    \includegraphics[width=\textwidth]{build/plt/spektrum_133-Ba.pdf}
    \caption{
        Energiespektrum der \ce{^133Ba}-Quelle mit eingezeichneten Emissionslinien von \ce{^133Ba} und \ce{^125Sb}.
        Werte über \SI{750}{\kilo\electronvolt} werden nicht dargestellt.
    }
    \label{fig:plt:spektrum_Ba_133}
\end{figure}

Zur Bestimmung der Aktivität der \ce{^133Ba}-Quelle
werden die Peaks wieder durch Gauß-Kurven approximiert
    (siehe \autoref{tab:4_Ba_133}).
Mit
    dem bereits bekannten Raumwinkel $\Omega$,
    der Messdauer $t_\text{m} = \SI{3465}{\second}$,
    der anhand des polynomiellen Fits aus \autoref{sec:auswertung:effizienz:effizienz} bestimmbaren Effizienz $Q$
    und den wie zuvor berechneten Peak-Inhalten $Z$ (siehe \autoref{eqn:peakinhalt})
lässt sich die Aktivität der \ce{^133Ba}-Quelle durch Umstellen von \autoref{eqn:effizienz} berechnen:
\begin{equation}
    A = \frac{4 \symup{\pi} Z}{\Omega Q W t_\text{m}}
\end{equation}
Die Ergebnisse sind in \autoref{tab:4_Ba_133_activities} zusammengefasst.

Als Aktivität der gesamten Quelle wird der Mittelwert der Aktivitäten aller Peaks mit Ausnahme des ersten gewählt,
    da dieser auffällig geringer ist als die anderen Peaks.
So ergibt sich eine Aktivität von $\bar A = \SI{1019.97 \pm 3.59}{\becquerel}$.

\begin{table}
    \centering
    \caption{
        Fit-Parameter der Gauß-Approximationen der Emissionslinien der \ce{^133Ba}-Quelle
        sowie die daraus berechneten Energien.
    }
    \label{tab:4_Ba_133}
    \expandableinput{build/tab/4_Ba-133.tex}
\end{table}

\begin{table}
    \centering
    \caption{
        Inhalte, Effizienzen und Aktivitäten der Emissionslinien der \ce{^133Ba}-Quelle
        sowie Literaturwerte der Energien.
    }
    \label{tab:4_Ba_133_activities}
    \expandableinput{build/tab/4_Ba-133_activities.tex}
\end{table}


\FloatBarrier
\subsection{Nuklididentifikation und Aktivitätsbestimmung} \label{sec:auswertung:uranstein}
Das Energiespektrum des Uran-Steins ist in \autoref{fig:plt:spektrum_uranstein} dargestellt.
Durch Vergleich der Peaks mit Literaturwerten lassen sich
    (ähnlich wie zuvor)
die aktiven Isotope identifizieren.
Über Gauß-Fits (siehe \autoref{tab:5_uranstein})
wird jedem Peak eine Energie und ein Peakinhalt zugeordnet.

(siehe \autoref{tab:5_uranstein_peaks}).
Aus den Energien lässt sich auf das jeweilige Isotop schließen,
während Peakinhalt, Emissionswahrscheinlichkeit und Effizienz die Berechnung der Aktivität ermöglichen
    (siehe \autoref{tab:5_uranstein_peaks}).
Durch Summation werden schließlich die Aktivitäten je Isotop zusammengefasst.
Diese sind in \autoref{tab:5_uranstein_activities_mean} angegeben.

\begin{figure}
    \centering
    \includegraphics[width=\textwidth]{build/plt/spektrum_uranstein.pdf}
    \caption{Energiespektrum des Uran-Steins mit eingezeichneten Emissionslinien der Zerfallsprodukte mit den höchsten Intensitäten.}
    \label{fig:plt:spektrum_uranstein}
\end{figure}

\begin{table}
    \centering
    \caption{
        Fit-Parameter der Gauß-Approximationen der Emissionslinien des Uran-Steins
        sowie die daraus berechneten Energien und Inhalte.
    }
    \label{tab:5_uranstein}
    \expandableinput{build/tab/5_uranstein.tex}
\end{table}

\begin{table}
    \centering
    \caption{
        Literaturwerte der Energien und Emissionswahrscheinlichkeiten für das jeweilige Isotop,
        sowie Effizienzen und Aktivitäten.
    }
    \label{tab:5_uranstein_activities}
    \expandableinput{build/tab/5_uranstein_activities.tex}
\end{table}

\begin{table}
    \centering
    \caption{Zusammengefasste Aktivitäten der Isotope des Uran-Steins.}
    \label{tab:5_uranstein_activities_mean}
    \expandableinput{build/tab/5_uranstein_activities_mean.tex}
\end{table}
