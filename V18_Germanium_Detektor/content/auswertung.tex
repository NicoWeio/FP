\section{Auswertung}
\label{sec:auswertung}
Fehlerrechnung mit \texttt{uncertainties}.
Bla bla. % TODO

\subsection{Energiekalibrierung}
Der Vielkanalanalysator wird anhand des Spektrums einer \ce{^152Eu}-Quelle kalibriert.
Dazu werden die Positionen der Peaks im
    über eine Zeit von $\SI{56.5}{\minute}$ gemessenen
Spektrum (siehe \autoref{fig:spektrum_Eu_152})
mit den bekannten Energien der \ce{^152Eu}-Quelle abgeglichen.
Jeder Peak wird durch den Fit einer Gaußfunktion angenähert:
\begin{equation*}
    N(x) = a \exp\left(-\frac{(x - x_0)^2}{2 \sigma^2}\right) + N_0 \, .
\end{equation*}
Dabei entsprechen
    $a$ der Amplitude, % TODO: Nicht direkt der Amplitude…
    $x_0$ der Position (als Kanalnummer)
    und $\sigma$ der Breite
des Peaks.
Der Parameter $N_0$ beschreibt den Untergrund.

\autoref{tab:1_energiekalibrierung} zeigt für jeden Peak
    die Fit-Parameter sowie die zugeordnete Energie aus der Literatur \cite{lara}.

\begin{figure}
    \centering
    \includegraphics[width=\textwidth]{build/plt/spektrum_152-Eu.pdf}
    \caption{
        Energiespektrum der \ce{^152Eu}-Quelle mit eingezeichneten Peaks.
        Die grünen Linien markieren die Position der Theoriewerte nach der Kalibrierung
        und sind entsprechend ihrer erwarteten Intensität unterschiedlich opak.
    }
    \label{fig:spektrum_Eu_152}
\end{figure}

\begin{table}
    \centering
    \caption{Fit-Parameter sowie zugeordnete Energie aus der Literatur \cite{lara} je Peak.}
    \label{tab:1_energiekalibrierung}
    \expandableinput{build/tab/1_energiekalibrierung.tex}
\end{table}

Durch lineare Regression kann schließlich die Abhängigkeit der Energie von der Kanalnummer bestimmt werden.
Für die Funktion
\begin{equation*}
    E(x) = m x + n
\end{equation*}
ergeben sich die Parameter
\begin{align*}
    m &= \SI{0.20679 \pm 0.00001}{\kilo\electronvolt} \\
    n &= \SI{-1.34 \pm 0.04}{\kilo\electronvolt} \, .
\end{align*}
Ein Plot dieser Regressionsrechnung ist in \autoref{fig:energy_calibration} gegeben.

\begin{figure}
    \centering
    \includegraphics[width=\textwidth]{build/plt/energy_calibration.pdf}
    \caption{Zugeordnete Energien in Abhängigkeit der Kanalnummern mit Regressionsgerade.}
    \label{fig:energy_calibration}
\end{figure}


\FloatBarrier % temporär
\subsection{Bestimmung der Effizienz bzw. Vollenergienachweiswahrscheinlichkeit anhand des Spektrums von \ce{^152Eu}}
Der Raumwinkel $\Omega$ wird anhand der Formel
\begin{equation*}
    \Omega = 2\pi \left( 1 - \frac{d}{\sqrt{d^2 + r^2}} \right)
\end{equation*}
bestimmt.
Mit
% TODO: Quelle(n)!
    einem Abstand von $d = \SI{8.31}{\centi\meter}$ zwischen Probe und Detektor
    sowie einem Detektorradius von $r = \SI{2.25}{\centi\meter}$
ergibt sich ein Raumwinkel von
$\Omega = 4\pi \cdot \num{0.0135} = \num{0.1691}$.

Die Effizienz wird anhand \autoref{eqn:effizienz} bestimmt… % TODO

\begin{figure}
    \centering
    \includegraphics[width=\textwidth]{build/plt/effizienz.pdf}
    \caption{TODO.}
    \label{fig:effizienz}
\end{figure}

\begin{table}
    \centering
    \caption{TODO.}
    \label{tab:2_effizienz}
    \expandableinput{build/tab/2_effizienz.tex}
\end{table}


\FloatBarrier % temporär
\subsection{Untersuchung des monochromatischen Gammaspektrums von \ce{^137Cs}}

\begin{figure}
    \centering
    \includegraphics[width=\textwidth]{build/plt/spektrum_137-Cs.pdf}
    \caption{TODO.}
    \label{fig:TODO}
\end{figure}
