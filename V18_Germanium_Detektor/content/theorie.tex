\section{Theorie}
\label{sec:theorie}

Im folgenden Abschnitt sollen die theoretischen Grundlagen der Wechselwirkung von hochenergetischer Strahlung mit Materie,
sowie die Funktionsweise des Germaniumdetektors erläutert werden.

\subsection{Wechselwirkung von Strahlung mit Materie}

Wenn hochenergetische $\gamma$-Strahlung auf Materie trifft,
treten,
abhängig von der Energie der Strahlung,
unterschiedliche Effekte auf.
Die Zahl der Photonen $N(d)$ ist durch den Zusammenhang
\begin{equation}
    N(d) = N_0 \cdot \symup{e}^{-\mu \cdot d}
\end{equation}
gegeben.
Der Faktor $\mu = n \cdot \sigma$ beschreibt den Extinktionskoeffizienten,
mit dem Wirkungsquerschnitt $\sigma$ des jeweiligen Prozess und $n$ der Anzahl Elektronen pro Probenvolumen.

Für Photon-Energien von $E_{\gamma} \approx \SI{100}{\kilo\eV}$ dominiert der \textbf{Photoeffekt}.
Dabei wird das Photon von einem Atom vollständig absorbiert und kann ein Elektron herauslösen,
wenn die Energie des Photons größer als die Bindungsenergie $E_\text{Bind}$ des Elektrons ist.
Das herausgelöste Elektron behält die Energie $E_\text{e} = E_{\gamma} - E_\text{Bind}$ zurück.
Im Atom entsteht eine Lücke in der entsprechenden Schale,
welche durch ein Elektron von einer anderen Schale gefüllt werden kann.
Dabei entsteht ein weiteres Photon mit der entsprechenden Energiedifferenz.
Für den Wirkungsquerschnitt des Photoeffekts gilt
\begin{equation*}
    \sigma_\text{Photo} \propto \frac{Z^n}{E^{\sfrac{7}{2}}_{\gamma}}
\end{equation*}
mit der Ordnungszahl $Z$ des Atoms und $4 \leq n \leq \num{5}$.

Für höhere Photon-Energien dominiert der \textbf{Compton-Effekt}.
Dabei stößt das Elektron mit einem Elektron und überträgt einen Teil der Energie auf das Elektron,
welcher abhängig vom Streuwinkel $\theta$ ist.
Die Energie des Photons nach dem Stoß kann mit
\begin{equation*}
    E^{'}_{\gamma} = \frac{E_{\gamma}}{1 + \epsilon(1 - \cos{\theta})}
\end{equation*}
berechnet werden,
wobei $\epsilon = \sfrac{E_{\gamma}}{m_\text{e} c^2}$ ist,
mit der Ruhemasse $m_\text{e}$ des Elektrons.
Da $\theta$ kontinuierliche Werte annehmen kann,
ist auch der Energieübertrag zwischen Photon und Elektron kontinuierlich.
Bei einem Streuwinkel von $\theta = \pi$ findet ein maximaler Energieübertrag statt,
das Elektron hat nach dem Stoß eine Energie von
\begin{equation*}
    E_\text{e} = \frac{2 \epsilon}{1 + 2 \epsilon} \ .
\end{equation*}
Der Wirkungsquerschnitt des Compton-Effekts ergibt sich über eine Winkelintegration der \emph{Klein-Nishina-Gleichung} \cite{TODO}.
Es ergibt sich
\begin{equation}
    \sigma_\text{Compton} =
    2 \pi r^2_\text{e} \biggl(\frac{1+\epsilon}{\epsilon} \biggl(\frac{1+2\epsilon}{2\epsilon} - \frac{1}{\epsilon}\ln(1+2\epsilon)\biggr)\biggr)
    + \frac{1}{2\epsilon} \ln(1+2\epsilon) - \frac{1+3\epsilon}{(1+2\epsilon)^2} \ .
\end{equation}

Ein weiterer möglicher Prozess ist durch \textbf{Paarbildung} gegeben.
Diese tritt erst für Photonenergien $E_{\gamma} \geq 2 m_\text{e} c^2 \approx \SI{1.022}{\mega\eV}$ auf.
Im Coulombfeld eines Atomkerns wandelt sich das Photon in ein Elektron-Positron-Paar um,
wobei diese die übrige Energie des Photons als kinetische Energie zurückbehalten.
Der Wirkungsquerschnitt für diesen Prozess verhält sich wie folgt
%TODO: Wirkungsquerschnitt Paarerzeugung von V14 übernehmen:
\begin{equation*}
    \sigma_\text{Paar} \propto Z^2 f(Z, E_{\gamma}) \ .
\end{equation*}
