\section{Discussion}
\label{sec:diskussion}

% \subsection{Deviations} % → N/A
% \subsection{Possible causes for deviations}

Overall,
the experiment was successful,
as the laser was sufficiently adjusted to ensure coverage of the demanded spectrum,
which could then be analyzed and displayed in a way that was comparable to the \hyperref[fig:measurement_lit]{expected results}.
%
However,
certain disturbances were observed,
which could have been avoided with more careful calibration and preparation.

One source of disturbance was the room light,
which adds an oscillating signal at a frequency of \SI{50}{\hertz} to the spectrum.
As such,
its impact varies with the frequency set at the \component{ramp generator}
and is hardly noticeable in \autoref{fig:measurement}.

Disturbance is also caused by the photodiodes,
which by mistake were not actively powered during the experiment,
therefore requiring higher gain
and ultimately leading to a less precise measurement.

Ultimately,
the manual adjustment of the laser using an Allen wrench was not very precise,
so that the longitudinal mode could not be found exactly,
resulting in suboptimal intensity.
The adjustment of the \componentlabel{SIDE} knob was similarly difficult,
but could be compensated for by the \component{piezo stack}.

% COULDDO: Remaining mode hops?
