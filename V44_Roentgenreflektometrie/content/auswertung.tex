\section{Auswertung}
\label{sec:auswertung}

% TODO: Fehlerrechnung

\subsection{Detektorscan}
Um die Strahlbreite zu quantifizieren,
wird der Detektorscan durch eine Gauß-Funktion approximiert,
welche zusätzliche Parameter für Untergrund ($I_0$) und Verschiebung ($\alpha_0$) enthält:

\begin{equation*}
    % I_max * np.exp(-((α - α_0)**2) / (2 * σ**2)) + I_0
    I(\alpha) = I_{\max} \exp\left(-\frac{( \alpha - \alpha_0)^2}{2 \sigma^2}\right) + I_0 \; .
\end{equation*}

Die Parameter werden mithilfe der Funktion \texttt{curve\_fit} aus dem Paket \texttt{scipy.optimize} bestimmt.
Sie ergeben sich zu:
\begin{align*}
    I_{\max} &= \SI{7.49(9)e5}{\per\second} \\
    I_0 &= \SI{7.5(22)e3}{\per\second} \\
    σ &= \SI{0.0431 \pm 0.0006}{\degree} \\
    α_0 &= \SI{-0.0017 \pm 0.0006}{\degree} \; .
\end{align*}

Die Halbwertsbreite beträgt somit:
\[
    \alpha_\text{FWHM}
    = 2 \sqrt{2 \ln 2} \cdot \sigma
    = \SI{0.1014 \pm 0.0014}{\degree} \; .
\]

In \autoref{fig:plt:1_detektor} sind die Messwerte sowie Fit und Halbwertsbreite dargestellt.

\begin{figure}
    \centering
    \includegraphics[width=0.75\textwidth]{build/plt/1_detektor.pdf}
    \caption{Messwerte und Fit des Detektorscans.}
    \label{fig:plt:1_detektor}
\end{figure}


\FloatBarrier
\subsection{Z-Scan}
\autoref{fig:plt:2_z1} zeigt die Messwerte des Z-Scans.
Die hervorgehobene Flanke von hoher zu geringer Intensität kann verwendet werden,
um die Breite des Röntgenstrahls zu abzuschätzen.
Sie beträgt etwa \SI{0.24}{\milli\meter}.
Die maximale Intensität liegt hier bei \SI{7.964(9)e5}{\per\second}
und somit oberhalb des gemessenen sowie gefitteten Maximums des Detektorscans.
% NOTE: größter Messwert dort: (7.659+/-0.009)e+05 / second

\begin{figure}
    \centering
    \includegraphics[width=0.75\textwidth]{build/plt/2_z1.pdf}
    \caption{Messwerte des Z-Scans mit hervorgehobener Strahlbreite.}
    \label{fig:plt:2_z1}
\end{figure}


\FloatBarrier
\subsection{Rockingscan}
Lorem ipsum dolor sit amet, consectetur adipiscing elit.

\begin{align*}
    % α_g_1 = -1.00 degree
    % α_g_2 = 0.92 degree
    % α_g_mean = 0.96 degree
    % α_g_alt = 0.8021671187643814 degree
    \alpha_{g, 1} &= \SI{-0.48}{\degree} \\
    \alpha_{g, 2} &= \phantom{-} \SI{0.48}{\degree} \\
    \alpha_g &= \frac{|\alpha_{g, 1}| + |\alpha_{g, 2}|}{2}
        = \SI{0.48}{\degree}
\end{align*}

\begin{figure}
    \centering
    \includegraphics[width=0.75\textwidth]{build/plt/4_rocking1_linear.pdf}
    \caption{Messwerte des Rockingscans.}
    \label{fig:plt:4_rocking1_linear}
\end{figure}

\begin{figure}
    \centering
    \includegraphics[width=0.75\textwidth]{build/plt/4_rocking1_log.pdf}
    \caption{Messwerte des Rockingscans.}
    \label{fig:plt:4_rocking1_log}
\end{figure}


\FloatBarrier
\subsection{Reflektivitätsscan}

% Plot-Aufteilung:
% 1.
% → Messdaten
% → Messdaten korrigiert um diffuse Strahlung
% → Messdaten korrigiert um Geometriefaktor
% → Messdaten korrigiert um diffuse Strahlung und Geometriefaktor
\begin{figure}
    \centering
    \includegraphics[width=\textwidth]{build/plt/schichtdicke_a.pdf}
    \caption{
        TODO.
        Zur besseren Lesbarkeit wird auf Fehlerbalken verzichtet.
    }
    \label{fig:plt:schichtdicke_a}
\end{figure}
% 2.
% → Messdaten (nur korrigiert)
% → Theorie-Kurve / Fit
%   → Fresnel-Reflektivität
%   → glatt
%   → rau
% → kritische Winkel
\begin{figure}
    \centering
    \includegraphics[width=\textwidth]{build/plt/schichtdicke_b.pdf}
    \caption{TODO.}
    \label{fig:plt:schichtdicke_b}
\end{figure}
