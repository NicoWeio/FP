\section{Diskussion}
\label{sec:diskussion}

\subsection{Abweichungen}
Die beiden gemessenen Geometriewinkel sind zwar identisch,
jedoch \SI{40.16}{\percent} kleiner als mit Strahlbreite und Probendicke ermittelte Wert.

Der Parratt-Algorithmus kann bei Berücksichtigung der Rauigkeit
die Kiessig-Oszillationen zwar reproduzieren,
die dafür notwendigen Parameter weichen jedoch merklich von den Literaturwerten ab.
Für $\delta_\text{PS}$ und $\delta_\text{Si}$ beträgt die relative Abweichung zu \cite{versuchsanleitung}
\SI{80.00}{\percent} beziehungsweise \SI{10.53}{\percent}.
Die kritischen Winkel $\alpha_\text{c, PS}$ und $\alpha_\text{c, Si}$
fallen um \SI{25.69}{\percent} beziehungsweise \SI{60.36}{\percent} größer als die Literaturwerte \cite{versuchsanleitung} aus.
Die Schichtdicke des Polystyrols wird
    sowohl aus den Abständen der Intensitätsmaxima
    als auch aus der Anpassung der Theoriekurve
mit guter Genauigkeit bestimmt.
Hierzu liegt jedoch kein Literaturwert vor.


\subsection{Mögliche Fehlerquellen}
Die Diskrepanz in den Geometriewinkeln ist durch die Uneindeutigkeit beim Ablesen zu begründen,
da der Übergang vom Untergrund zum Röntgenstrahl nicht eindeutig ist.

Aufgrund der Vielzahl an Parametern zur theoretischen Beschreibung der Reflektivität
wurde keine numerische Optimierung selbiger durchgeführt.
Stattdessen wurden der Einfluss der einzelnen Parameter studiert
und manuelle Anpassungen vorgenommen.
Erschwert wurde die Anpassung dadurch,
dass die Kiessig-Oszillationen nur wenig ausgeprägt waren.
