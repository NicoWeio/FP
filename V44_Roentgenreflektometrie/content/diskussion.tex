\section{Diskussion}
\label{sec:diskussion}

\subsection{Abweichungen}
Die beiden \hyperref[sec:auswertung:rockingscan]{gemessenen Geometriewinkel} sind zwar identisch,
jedoch \SI{30.19}{\percent} kleiner als der mit Strahlbreite und Probendurchmesser ermittelte Wert.

Der \hyperref[sec:auswertung:schichtdicke]{Parratt-Algorithmus} kann bei Berücksichtigung der Rauigkeit
die Kiessig-Oszillationen zwar weitgehend reproduzieren,
die dafür notwendigen Parameter weichen jedoch merklich von den Literaturwerten ab
und das Verhalten für große Winkel konnte nicht gleichzeitig reproduziert werden.
Für $\delta_\text{PS}$ und $\delta_\text{Si}$ beträgt die relative Abweichung zu \cite{versuchsanleitung}
\SI{68.48}{\percent} beziehungsweise \SI{5.42}{\percent}.
Die kritischen Winkel $\alpha_\text{c, PS}$ und $\alpha_\text{c, Si}$
fallen um \SI{44.37}{\percent} beziehungsweise \SI{24.85}{\percent} kleiner/größer als die Literaturwerte \cite{versuchsanleitung} aus.
% TODO: \beta erwähnen?
Die Schichtdicke des Polystyrols wird
    sowohl aus den Abständen der Intensitätsmaxima
    als auch aus der Anpassung der Theoriekurve
mit guter Genauigkeit bestimmt.
Die relative Abweichung zwischen den beiden Werten beträgt \SI{5.06}{\percent}
    (bezogen auf letztere Methode).
Es liegt jedoch kein Literaturwert vor.


\subsection{Mögliche Fehlerquellen}
Die für die Qualität der späteren Messwerte entscheidende Einjustierung
wurde manuell durchgeführt;
einige Justage-Parameter konnten daher nicht genau übernommen werden,
was signifikante Qualitätseinbußen verursacht haben könnte.

Die Diskrepanz in den Geometriewinkeln ist unter anderem durch die Uneindeutigkeit beim Ablesen zu begründen,
da der Übergang vom Untergrund zum Röntgenstrahl nicht eindeutig ist.

Aufgrund der Vielzahl an Parametern zur theoretischen Beschreibung der Reflektivität
wurde keine numerische Optimierung selbiger durchgeführt.
Stattdessen wurden der Einfluss der einzelnen Parameter studiert
und manuelle Anpassungen vorgenommen.
Erschwert wurde die Anpassung dadurch,
dass die Kiessig-Oszillationen nur wenig ausgeprägt waren.

% Zu Fehlerquellen vllt auch noch, dass wir die Einjustierung ja manuell gemacht haben, und dass das die Qualität der Messung beeinflusst hat
