\section{Theorie}
\label{sec:theorie}

In diesem Abschnitt sollen die theoretischen Grundlagen der Röntgenreflektometrie erläutert werden.
Dabei wird die Entstehung von Röntgenstrahlung erklärt,
sowie Besonderheiten bei der Reflektion der Strahlung an Oberflächen.

Röntgenstrahlung entsteht in einer Röntgenröhre,
welche aus einem Glaskolben besteht,
in dem sich eine Glühkathode,
sowie eine Anode befinden.
Mithilfe des glühelektrischen Effekts werden Elektronen erzeugt,
die in einem elektrischen Feld,
welches durch eine Beschleunigungsspannung zwischen Glühkathode und Anode vorliegt,
zur Anode hin beschleunigt werden.
Treffen die Elektronen auf die Anode,
werden sie stark abgebremst,
sodass Bremsstrahlung entsteht.
Des Weiteren übertragen sie ihre Energie an Anodenatome,
welche angeregt werden und schließlich Photonen mit einer charakteristischen Wellenlänge emittieren.
Diese bilden das charakteristische Spektrum der Röntgenstrahlung.

\subsection{Reflexion von Röntgenstrahlung an Oberflächen}

