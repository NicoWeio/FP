% \PassOptionsToPackage{sorting=none}{biblatex}
\input{../header_common.tex}

\usepackage[nolist,nohyperlinks]{acronym}
\acrodef{PMT}{Photomultiplier Tube}
\acrodef{TAC}{Time-Amplitude-Converter}
\acrodef{MCA}{Multichannel-Analyser}

\usepackage{longtable}

\sisetup{per-mode=reciprocal}

\subject{V01}
\title{Lebensdauer kosmischer Myonen}
\date{
    Durchführung: 24.10.2022
     \hspace{3em}
    Abgabe: 02.12.2022
}
\begin{document}

\pagenumbering{gobble}
\maketitle
\pagenumbering{arabic}
% \thispagestyle{empty}
% \tableofcontents
% \newpage
% \setcounter{page}{1}

\section{Zielsetzung}

    Im folgenden Versuch soll mithilfe eines Szintillationsdetektors die Lebensdauer kosmischer Myonen aus der Messung der Individuallebensdauern bestimmt werden.
    Dabei wird besonderer Fokus auf den Aufbau und die Funktionsweise der Messapparatur gelegt.

\section{Theorie}
\label{sec:theorie}

Im Folgenden werden die theoretischen Grundlagen für diesen Versuch erläutert.
Diese beinhalten die Eigenschaften des Myons
sowie die Methoden zur Messung der Lebensdauer.


\subsection{Eigenschaften des Myons}

Das Myon $\mu$ ist ein geladenes Lepton aus der zweiten Generation des Standardmodells der Teilchenphysik.
Es hat eine Masse von $m = \SI{105.658}{\mega\eV}$ und eine mittlere Lebensdauer von etwa $\SI{2,196e-6}{\second}$ \cite{pdg}.

Die hier untersuchten Myonen stammen aus kosmischer Strahlung,
wobei zwischen primärer und sekundärer komischer Strahlung unterschieden wird \cite{grupen}.
Primäre kosmische Strahlung beinhaltet unter anderem Protonen,
Photonen und Neutrinos,
welche mit den Atomen in der Atmosphäre wechselwirken.
Dadurch entstehen weitere Teilchen,
welche als sekundäre kosmische Strahlung bezeichnet werden.
Myonen sind ein Teil der sekundären kosmischen Strahlung und werden hauptsächlich in den Prozessen
\begin{align*}
    \pi^{+} &\to \mu^{+}     \nu_{\mu} &
    \pi^{-} &\to \mu^{-} \bar\nu_{\mu}
\end{align*}
erzeugt,
in einer Höhe von etwa \SIrange{10}{15}{\kilo\meter}.
Da das Myon in leichtere Teilchen zerfallen kann,
hat es eine endliche Lebensdauer.
Die häufigsten Prozesse sind
\begin{align*}
    \mu^{+} &\to e^{+}     \nu_{e} \bar\nu_{\mu} &
    \mu^{-} &\to e^{-} \bar\nu_{e}     \nu_{\mu} \ ,
\end{align*}
wobei das Myon in das leichtere Elektron und die zugehörigen Neutrinos zerfällt.
Die Lebensdauer der Myonen ist dabei über ihre Wahrscheinlichkeit zu zerfallen definiert.
% NOTE: Wie kann W hier eine Wahrscheinlichkeit sein? Für beliebig große (endliche) dt wäre dW doch >1?
Dabei gilt
\begin{equation*}
    \symup{d}W = \lambda \symup{d}t \ ,
\end{equation*}
wobei sich die gesamte Teilchenzahl um
\begin{equation*}
    \symup{d}N = -N \symup{d}W
\end{equation*}
ändert.
Daraus lässt sich das Zerfallsgesetz
\begin{equation*}
    N(t) = N_0 \text{exp}(- \lambda t)
\end{equation*}
mit der teilchenspezifischen mittleren Lebensdauer $\tau = \sfrac{1}{\lambda}$ ableiten.


\subsection{Messung der Lebensdauer}

Die mittlere Lebensdauer wird in diesem Versuch über den Mittelwert der Individuallebensdauern der Myonen bestimmt.
Um diese zu messen,
wird ein Szintillationsdetektor aus organischem Material verwendet.
Wenn ein Myon auf das Detektormaterial trifft,
wird es absorbiert,
wobei ein Lichtblitz mit einem charakteristischen Spektrum entsteht.
Dieser Lichtblitz wird in ein elektrisches Signal umgewandelt,
welches einen Zähler startet.
Um die Lebensdauer der Myonen zu bestimmen,
wird die Zeit zwischen zwei Signalen gemessen,
wobei das erste Signal durch das Eindringen des Myons in den Detektor entsteht,
während das zweite Signal auf das Elektron zurückzuführen ist,
welches beim Zerfall des Myons entsteht.
Die Signatur der Signale ist dabei für Elektronen und Myonen ununterscheidbar.
Aus diesem Grund kann es passieren,
dass die Messung zu früh gestoppt wird,
wenn ein weiteres Myon während der Suchzeit $T_\text{S}$ in den Detektor eindringt und ein Signal auslöst.
Die so entstehende Zahl von Untergrundereignissen wird über die Gleichung
\begin{equation}
    U = I_\text{Start} \cdot T_\text{S} \cdot \exp(I_\text{Start} \cdot T_\text{S}) \cdot N_\text{Start}
    \label{eqn:untergrund}
\end{equation}
mit $I_\text{Start} = N_\text{Start} / t_\text{Messung}$ beschrieben,
wobei die Wahrscheinlichkeit,
dass eine gewisse Anzahl von Myonen während der Suchzeit in den Detektor tritt,
poissonverteilt ist.
Die Größen $N_\text{Start}$ und $t_\text{Messung}$ beschreiben die Anzahl der Start-Signale und den gesamten Zeitraum der Messung.

\clearpage

\section{Durchführung}
\label{sec:durchfuehrung}

In diesem Kapitel wird der verwendete Aufbau beschrieben,
sowie die Einjustierung der jeweiligen Bauteile und die eigentliche Messung.

\subsection{Versuchsaufbau}

Der hier verwendete Aufbau ist in \autoref{fig:versuchsaufbau} gezeigt.
Der Detektor besteht aus einem zylinderförmigen Szintillatortank aus organischem Material \cite{kolanoskiwermes},
welcher ein Volumen von etwa \SI{50}{\liter} hat.
An beiden Enden des Detektors sind Photomultiplier (PMT) \cite{wrleo} angebracht,
welche an eine Hochspannungsleitung angeschlossen sind.
Mit Hilfe dieser wird das elektrische Signal,
welches im PMT durch den Photoeffekt erzeugt wird,
verstärkt.
Das jeweilige Signal der PMT wird über eine Verzögerungsleitung geleitet,
welche eventuelle Zeitverzögerungen zwischen den PMTs ausgleichen soll,
da die folgende Konzidenzschaltung nur für gleichzeitige Signale beider PMTs durchlässig ist.
Um weitere Untergrundquellen zu unterdrücken,
werden Diskriminatoren genutzt,
die eine Energieschwelle festlegen und so Signale,
die unterhalb dieser Schwelle liegen,
herausfiltern. 
Passiert ein Signal die Koinzidenzschaltung,
wird über ein AND-Gatter ein Impulszähler und die Zeitmessung gestartet.
Über ein Verzögerungsleitung mit \SI{30}{\nano\second} wird außerdem ein Univibrator erreicht. 
An diesem wird eine Suchzeit $T_\text{S}$ eingestellt,
in der das Signal des Zerfalls erwartet wird.
Wenn während dieser Suchzeit ein Signal gemessen wird,
wird ein zweites AND-Gatter aktiviert,
welches einen weiteren Impulszähler startet und gleichzeitig die Zeitmessung stoppt.
Diese Messung wird über einen Time-Amplitude-Converter (TAC) \cite{wrleo} in eine Amplitude umgewandelt und mit Hilfe eines Multichannel-Analysers (MCA) \cite{wrleo} gespeichert,
wobei die Signale anhand dieser Amplitude,
beziehungsweise Messzeit in verschiedenen Kanälen gespeichert werden.
Für den Fall,
dass während $T_\text{S}$ kein weiteres Signal gemessen wird,
wird die Messung zurückgesetzt.

\begin{figure}
    \centering
    \includegraphics[width=0.8\textwidth]{content/img/Abb_1.pdf}
    \caption{Schematische Abbildung des verwendeten Versuchaufbaus.
    Die im Szintillationsdetektor gemessenenen Signale werden in elektrische Impulse umgewandelt,
    welche über logische Bauelemente eine Zeitmessung aktivieren \cite{versuchsanleitung}.}
    \label{fig:versuchsaufbau}
\end{figure}

\subsection{Kalibrierung der Schaltung}

Vor Beginn der eigentlichen Messung müssen die Bauteile kalibriert werden.
Dies umschließt die Verzögerungsleitungen hinter den Photomultipliern,
sowie die Energieschwelle der Diskriminatoren und den Multichannel-Analyser.
Um sicher zu stellen,
dass beide Photomultiplier Signale senden,
werden sie an ein Oszilloskop angeschlossen,
wobei jeweils ein Peak erkennbar sein sollte. 

\subsubsection{Schwellspannung der Diskriminatoren}

Für die Einjustierung der Schwellspannung der Diskriminatoren werden die Photomultiplier an jeweils einen Diskriminator angeschlossen,
welcher wiederum mit zwei Impulszählern verbunden sind.
Es wird unter Variation der Schwellspannung die Anzahl der registrierten Impulse gemessen,
wobei diese im Bereich von \SI[per-mode=reciprocal]{30}{\per\second} liegen sollte.
Außerdem sollte das gemessene Signal eine Dauer von etwa $\symup{\Delta}t = \SI{10}{\nano\second}$ haben,
was über das Oszilloskop kontrolliert wird.

\subsubsection{Justierung der Verzögerungsleitungen}

Um eventuelle Zeitverzögerungen der PMTs auszugleichen und sicher zu stellen,
dass die Signale die Koinzidenzschaltung gleichzeitig erreichen,
wird die Impulsrate in Variation der Verzögerungszeit der beiden Verzögerungsleitungen gemessen.
Für eine bestmögliche Einstellung sollte diese einen Wert von etwa \SI[per-mode=reciprocal]{20}{\per\second} aufweisen.
Für die Justierung wird das über die Verzögerungsleitungen und Diskriminatoren geleitete Signal an die Koinzidenzschaltung gegeben,
welche mit einem Impulszähler verbunden ist.
Über einen Zeitraum von $\symup{\Delta}t = \SI{20}{\second}$ wird nun die Impulsrate unter Variation der Verzögerungszeit gemessen.
Anschließend wird die Verzögerungszeit, 
für die die gemessene Impulsrate der bestmöglichen am nächsten ist,
fest eingestellt.

\subsubsection{Kalibrierung der Multichannel-Analysers}

Für den letzten Teil der Justage wird der restliche Teil der Schaltung aufgebaut,
wobei am Univibrator eine Suchzeit von $T_\text{S} = \SI{10}{\micro\second}$ eingestellt wird.
Anschließend wird ein Doppelimpulsgenerator angeschlossen.
Dieser erzeugt mit einer Frequenz von \SI{1}{\kilo\hertz} zwei Impulse,
welche das Start- und Stopp-Signal simulieren.
In einem Zeitintervall von \SI{0.3}{\micro\second} bis \SI{9.9}{\micro\second} wird nun schrittweise die Zeit zwischen den Impulsen variiert.
Der Multichannel-Analyser speichert die jeweilige Zeit zwischen den gemessenen Impulse in verschiedenen Kanälen,
sodass für jede Zeitdauer der entsprechende Kanal notiert wird.

Abschließend wird der Doppelimpulsgenerator aus der Schaltung entfernt und die Photomultiplier wieder an die Impulszähler angeschlossen,
um die eigentliche Messung der Lebensdauer der Myonen zu starten.
Diese sollte eine Dauer von mindestens \SI{20}{\hour} umfassen.

\clearpage

\section{Auswertung}
\label{sec:auswertung}

\subsection{Stabilitätsbedingung}
% Aufgabe 2 in der Versuchsanleitung
\lipsum[1]

\begin{figure}
  \centering
   \includegraphics[width=\textwidth]{build/plt/2_stabilitaetsbedingung_theorie.pdf}
   \caption{Stabilitätsparameter $g_1 \cdot g_2$ in Abhängigkeit der Resonatorlänge $L$ für verschiedene Spiegelkonfigurationen.}
   \label{fig:plt:stabilitaetsbedingung_theorie}
\end{figure}

Durch Lösen von \autoref{eqn:TODO} können die theoretisch größtmöglichen Resonatorlängen $L$ bestimmt werden.
Hierbei ist zu beachten, dass bei Gleichheit nur noch ein metastabiler Zustand vorliegt.
\begin{align*}
    L_\text{max, kk, theo} &= r_1 + r_2 \\
    L_\text{max, pk, theo} &= r_2 \\
\end{align*}


\subsection{TEM-Moden}
% Aufgabe 2 in der Versuchsanleitung
\lipsum[1]

\begin{figure}
  \centering
   \includegraphics[width=\textwidth]{build/plt/3_tem_00.pdf}
   \caption{Lichtintensität in Abhängigkeit der Distanz zur optischen Achse für die $\text{TEM}_{00}$-Mode.}
   \label{fig:plt:tem_00}
\end{figure}

\begin{figure}
  \centering
   \includegraphics[width=\textwidth]{build/plt/3_tem_01.pdf}
   \caption{Lichtintensität in Abhängigkeit der Distanz zur optischen Achse für die $\text{TEM}_{01}$-Mode.}
   \label{fig:plt:tem_01}
\end{figure}


\subsection{Polarisation}
\lipsum[1]

\begin{figure}
  \centering
   \includegraphics[width=\textwidth]{build/plt/4_polarisation.pdf}
   \caption{Lichtintensität in Abhängigkeit der Winkeleinstellung $\alpha$ des Polarisationsfilters.}
   \label{fig:plt:polarisation}
\end{figure}


\subsection{Multimodenbetrieb und Frequenzspektrum des Lasers}
% \subsection{Longitudinale Moden}
% 5. Multimodenbetrieb und Frequenzspektrum des Lasers


\subsection{Wellenlänge}
\lipsum[1]


\clearpage

\section{Discussion}
\label{sec:diskussion}

\subsection{Deviations}

…


\subsection{Possible causes for deviations}

\begin{itemize}
    \item Photodiodes were not powered on
    \item More careful calibration
    \item Ambient light (50Hz)
    \item …
\end{itemize}
…

\clearpage

\printbibliography
\clearpage

\appendix
\section{Anhang}
\subsection{Weitere Messwerte}

\begin{longtable}{S S}
\caption{Messwerte zur Impulszahl je Kanal.}
\label{tab:3_lebensdauer} \\
\expandableinput{build/tab/3_lebensdauer.tex}
\end{longtable}


\end{document}
