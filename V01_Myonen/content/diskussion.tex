\section{Diskussion}
\label{sec:diskussion}

\subsection{Abweichungen}
Bei der \hyperref[sec:auswertung:koinzidenz]{Justage der Koinzidenzapparatur} zeigte sich
ein Maximum der Zählrate bei einer kleinen Zeitverschiebung
und davon ausgehend ein symmetrischer, weitestgehend linearer Abfall zu beiden Seiten.
Da die Randwerte relativ nah an Null sind,
kann geschlossen werden,
dass der Untergrund effektiv herausgefiltert wurde.
% KORR: Und Anhand der Breite könnt ihr auch sehen,dass das Koinzidenz Zeitfenster klein gegenüber den 10ns ist,
% da macht es kein Einfluss ob es 10 9,9 oder 10,1 ns sind.

% COULDDO: Halbwertsbreite einbauen
% Beide Beobachtungen decken sich mit den Erwartungen.

Die \hyperref[sec:auswertung:mca]{Kalibrierung des \acs{MCA}} ergab einen linearen Zusammenhang zwischen Kanalnummer und Zeitdifferenz.
% NOTE: Der Zusammenhang war sogar *perfekt*, aber das darf man ja nicht schreiben…
Die Abweichungen von der Regressionsgeraden liegen im Bereich der Rechengenauigkeit.

Die \hyperref[sec:auswertung:lebensdauer]{Bestimmung der mittleren Lebensdauer} durch einen Fit war in Anbetracht der Unsicherheiten zufriedenstellend.
Der anhand von \autoref{eqn:untergrund} vorhergesagte Untergrund von
$U_1 = \num{1.47 \pm 0.15}$ pro Kanal
trägt dazu bei, dass der beim Fit bestimmte übrige Untergrund von
$U_2 = \num{-0.61 \pm 0.19}$
sogar negativ ist.
Mögliche Ursachen sind die hohe Messungenauigkeit sowie das ersatzweise Vorgehen zur Bestimmung von $N_\text{Start}$,
wovon $U_1$ direkt abhängt.
%
Aufgrund der begrenzten Einstellmöglichkeiten des Binnings des \ac{TAC} wurden keine Ereignisse für besonders kleine und große Kanäle registriert.
Die hier bestimmte mittlere Lebensdauer $\tau = \SI{1.91 \pm 0.04}{\micro\second}$
weicht um \SI{-0.28}{\micro\second} beziehungsweise \SI{-12.9}{\percent} vom Literaturwert $\tau_\text{lit} = \SI{2.20}{\micro\second}$ \cite{pdg} ab,
wobei die Abweichung über die angegebene Unsicherheit hinausgeht.


\subsection{Mögliche Fehlerquellen}

Die Abweichungen bei der Messung der Lebensdauer der Myonen können auf verschiedene Unsicherheiten bei der Messung zurückgeführt werden.

Einerseits wurden die Schwellspannungen der Diskriminatoren nur über das händische Drehen einer Schraube ohne Skala variiert.
Wären zudem mehr Messungen der Impulse über einen längeren Zeitraum vorgenommen worden,
hätte die Schwellspannung weiter optimiert werden können,
sodass der Diskriminator effektiver Untergrundsignale hätte herausfiltern können.
%
Auch die Dauer der Impulse von \SI{10}{\nano\second} konnte auf dem Oszilloskop nur bedingt eingestellt werden.

Des Weiteren wurden bei der Messung der Impulsraten zur Bestimmung der bestmöglichen Verzögerungszeit an den Photomultipliern
manuell die Zeit sowie die Impulszähler gestoppt,
was zusätzliche Unsicherheiten zur Folge hat.
%
Es konnten außerdem nur diskrete Werte eingestellt werden,
sodass die bestmögliche Verzögerungszeit tatsächlich bei einem leicht anderen Wert liegen könnte.
Dies verursacht leicht geringere Zählraten und somit etwas größere relative Unsicherheiten,
die aber durch längeres Messen kompensiert werden könnten.

Eine weitere Unsicherheit ergab sich bei der Berechnung der Untergrundrate $U_1$,
da die Gesamtzahl der Start-Ereignisse $N_\text{Start}$ nicht erfasst wurde.
Aus diesem Grund wurde dieser Wert über die bei der Kalibrierung gemessenen Impulsraten und die Gesamtdauer der Messung abgeschätzt.
Es ist anzunehmen,
dass eine direkte Messung der Start-Ereignisse genauere Ergebnisse liefern würde.

% TODO: Gewichtung nach Unsicherheit präferiert kleinere Messwerte?

% KORR: Größten Einfluss hat der Untergrund, euer Wert sollte noch besser werden wenn ihr den berechneten Untergrund [→ U_1] benutzt.
% NOTE: Festsetzen von U=U_1 im Fit gibt eine größere Abweichung von 17.76%.
