\section{Diskussion}
\label{sec:diskussion}

\subsection{Abweichungen}
Bei der \hyperref[sec:auswertung:koinzidenz]{Justage der Koinzidenzapparatur} zeigte sich
ein Maximum der Zählrate bei einer kleinen Zeitverschiebung
und davon ausgehend ein weitestgehend linearer Abfall zu beiden Seiten.

Die \hyperref[sec:auswertung:mca]{Kalibrierung des \acs{MCA}} ergab einen perfekt linearen Zusammenhang zwischen Kanalnummer und Zeitdifferenz.
Die Abweichungen von der Regressionsgeraden liegen im Bereich der Rechengenauigkeit.

Die \hyperref[sec:auswertung:lebensdauer]{Bestimmung der mittleren Lebensdauer} durch einen Fit war in Anbetracht der Unsicherheiten zufriedenstellend.
Der negative Wert für $U_2$ ist auf die Messungenauigkeit sowie das ersatzweise Vorgehen zur Bestimmung von $U_1$ zurückzuführen.
Aufgrund der begrenzten Einstellmöglichkeiten des \ac{TAC} wurden keine Ereignisse für besonders kleine und große Partikel registriert.
Die hier bestimmte mittlere Lebensdauer $\tau$
weicht um \SI{-0.28}{\micro\second} beziehungsweise \SI{-12.9}{\percent} vom Literaturwert $\tau_\text{th} = \SI{2.20}{\micro\second}$ \cite{pdg} ab.


\subsection{Mögliche Fehlerquellen}

Die Abweichungen bei der Messung der Lebensdauer der Myonen können auf verschiedene Unsicherheiten bei der Messung zurückgeführt werden.

Zum einen wurde bei der Einstellung der Schwellspannung der Diskriminatoren diese nur über das Drehen einer Schraube variiert,
wobei keine Skala gegeben war und die Änderung nur über Handmaß geschah.
%
Wären zudem mehr Messungen der Impulse über einen längeren Zeitraum vorgenommen worden,
hätte die Schwellspannung weiter optimiert werden können,
sodass der Diskriminator effektiver Untergrundsignale hätte herausfiltern können.
%
Auch die Dauer der Impulse von \SI{10}{\nano\second} konnte auf dem Oszilloskop nur bedingt eingestellt werden.
%
Des Weiteren wurde bei der Messung der Impulsraten zur Bestimmung der bestmöglichen Verzögerungszeit an den Photomultipliern
manuell die Zeit sowie die Impulszähler gestoppt,
sodass es auch hier zu Unsicherheiten kommen kann.
%
Es konnten außerdem nur diskrete Werte eingestellt werden,
sodass die bestmögliche Verzögerungszeit tatsächlich bei einem leicht anderen Wert liegen könnte.
%
Eine weitere Unsicherheit ergab sich bei der Bestimmung der Untergrundrate,
da die Gesamtzahl der Start-Ereignisse $N_\text{Start}$ nicht vollständig gemessen werden konnte.
Aus diesem Grund wurde dieser Wert über die bei der Kalibrierung gemessenen Impulsraten und die Gesamtdauer der Messung abgeschätzt.
Die tatsächliche Untergrundrate kann dementsprechend von der hier angenommenen abweichen.
