\section{Diskussion}
\label{sec:diskussion}

\subsection{Abweichungen}

…


\subsection{Mögliche Fehlerquellen}

Die Abweichungen bei der Messung der Lebensdauer der Myonen können auf verschiedene Unsicherheiten bei der Messung zurückgeführt werden.

Zum Einen wurde bei der Einstellung der Schwellspannung der Diskriminatoren diese nur über das Drehen einer Schraube variiert,
wobei keine Skala gegeben war und die Änderung nur über Handmaß geschah.
Wären zudem mehr Messungen der Impulse über einen längeren Zeitraum vorgenommen worden,
hätte die Schwellspannung weiter optimiert werden können,
sodass der Diskriminator effektiver Untergrundsignale herausfiltern hätte können.
Auch die Dauer der Impulse von \SI{10}{\nano\second} konnte auf dem Oszilloskop nur bedingt eingestellt werden.
Des Weiteren wurde bei der Messung der Impulsraten zur Bestimmung der bestmöglichen Verzögerungszeit an den Photomultipliern manuell die Zeit,
sowie die Impulszähler gestoppt,
sodass es auch hier zu Unsicherheiten kommen kann.
Es konnten außerdem nur diskrete Werte eingestellt werden,
sodass die bestmögliche Verzögerungszeit tatsächlich bei einem leicht anderen Wert liegen könnte.
Eine weitere Unsicherheit ergab sich bei der Bestimmung der Untergrundrate,
da die Gesamtzahl der startenden Ereignisse $N_\text{Start}$ nicht vollständig gemessen werden konnte.
Aus diesem Grund wurde dieser Wert über die bei der Kalibrierung gemessenen Impulsraten und die Gesamtdauer der Messung abgeschätzt.
Die tatsächliche Untergrundrate kann dementsprechend von der hier angenommenen abweichen.

