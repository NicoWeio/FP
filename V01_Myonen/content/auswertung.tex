\section{Auswertung}
\label{sec:auswertung}

\subsection{Justage}
Um sicherzustellen,
dass fertigungsbedingte Unterschiede in der Reaktionszeit der \acp{PMT}
nicht zum ungewünschten Ausschluss von Ereignissen durch die Koinzidenz-Schaltung führen,
% dass die leicht verschieden auslösenden \acp{PMT}
% dass die Koinzidenz-Schaltung tatsächlich auslöst,
% wenn beide \ac{PMT} ein Signal liefern,
werden die Verzögerungsleitungen zwischen \acp{PMT} und Koinzidenz derart justiert,
dass die Ereignisrate am Ausgang maximal ist.
% Eine der Verzögerungsleitungen wird dabei auf Null belassen…
Für jede Einstellung wird über eine Zeit von $\SI{0}{\second}$ gemessen.

Die mit den Verzögerungsleitungen eingestellte Zeitverschiebung zwischen den \acp{PMT}
wird durch die vorzeichenbehaftete Variable $t_\text{diff}$ beschrieben.

In \autoref{fig:plt:1_koinzidenz} lässt sich ein deutliches Maximum der Ereignisrate
bei einer Zeitdifferenz von $t_\text{diff} = \SI{0}{\nano\second}$ erkennen.

\begin{table}
    \centering
    \caption{TODO.}
    \label{tab:1_koinzidenz}
    \expandableinput{build/tab/1_koinzidenz.tex}
\end{table}

\begin{figure}
    \centering
    \includegraphics[width=\textwidth]{build/plt/1_koinzidenz.pdf}
    \caption{Messwerte und Fit zur Lebensdauer…}
    \label{fig:plt:1_koinzidenz}
\end{figure}

\subsection{Kalibration}
\begin{table}
    \centering
    \caption{TODO.}
    \label{tab:2_mca}
    \expandableinput{build/tab/2_mca.tex}
\end{table}

\begin{figure}
    \centering
    \includegraphics[width=\textwidth]{build/plt/2_mca.pdf}
    \caption{Messwerte und Fit zur Lebensdauer…}
    \label{fig:plt:2_mca}
\end{figure}


\subsection{Theoretische Untergrundrate}


\subsection{Bestimmung der mittleren Lebensdauer}
\begin{figure}
    \centering
    \includegraphics[width=\textwidth]{build/plt/3_lebensdauer.pdf}
    \caption{Messwerte und Fit zur Lebensdauer…}
    \label{fig:plt:3_lebensdauer}
\end{figure}
