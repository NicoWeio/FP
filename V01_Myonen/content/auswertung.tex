\section{Auswertung}
\label{sec:auswertung}

\FloatBarrier
\subsection{Justage der Koinzidenzapparatur}
Um sicherzustellen,
dass fertigungsbedingte Unterschiede in der Reaktionszeit der \acp{PMT}
nicht zum ungewünschten Ausschluss von Ereignissen durch die Koinzidenz-Schaltung führen,
% dass die leicht verschieden auslösenden \acp{PMT}
% dass die Koinzidenz-Schaltung tatsächlich auslöst,
% wenn beide \ac{PMT} ein Signal liefern,
werden die Verzögerungsleitungen zwischen \acp{PMT} und Koinzidenz derart justiert,
dass die Ereignisrate am Ausgang maximal ist.
% Eine der Verzögerungsleitungen wird dabei auf Null belassen…
Für jede Einstellung wird über eine Zeit von $\SI{60}{\second}$ gemessen.
%
Die mit den Verzögerungsleitungen eingestellte Zeitverschiebung zwischen den \acp{PMT}
wird durch die vorzeichenbehaftete Variable $t_\text{diff}$ beschrieben.

Die in Impulsraten umgerechneten Messwerte sind in \autoref{tab:1_koinzidenz} aufgelistet.
In \autoref{fig:plt:1_koinzidenz} lässt sich das Maximum der Ereignisrate
bei einer Zeitdifferenz von $t_\text{diff} = \SI{-1}{\nano\second}$ erkennen.
Es beträgt \SI{19.8 \pm 1.0}{\per\second}.
Die zugehörige Halbwertsbreite beträgt \SI{10.1}{\nano\second}.

\begin{table}
    \centering
    \caption{Messwerte zur Abhängigkeit der Ereignisrate von der Zeitverschiebung zwischen \acp{PMT}.}
    \label{tab:1_koinzidenz}
    \expandableinput{build/tab/1_koinzidenz.tex}
\end{table}

\begin{figure}
    \centering
    \includegraphics[width=\textwidth]{build/plt/1_koinzidenz.pdf}
    \caption{Messwerte und Fit zur Abhängigkeit der Ereignisrate von der Zeitverschiebung zwischen \acp{PMT}.}
    \label{fig:plt:1_koinzidenz}
\end{figure}


\FloatBarrier
\subsection{Kalibration des \acs{MCA}}
Die vom \ac{TAC} ausgegebenen Amplituden werden im \ac{MCA} in \num{512} Kanälen histogrammiert.
In Vorbereitung auf die eigentliche Messung muss daher
der Zusammenhang zwischen Zeitdifferenz und Kanalnummer bestimmt werden.
Dafür wird für verschiedene am Doppelimpulsgenerator eingestellten Zeitdifferenzen
die entsprechende Kanalnummer notiert.
Hierbei zeigt sich ein perfekt linearer Zusammenhang.
Die Messwerte sind in \autoref{tab:2_mca} aufgeführt und in \autoref{fig:plt:2_mca} dargestellt.

Eine lineare Regressionsrechnung ergibt
für eine Ausgleichsgerade $t(c) = a \cdot c + b$
die folgenden Parameter:
\begin{align*}
    a &=\SI{0.0217 \pm 0}{\micro\second} \\
    b &=\SI{0.1391 \pm 0}{\micro\second} \; .
\end{align*}

\begin{table}
    \centering
    \caption{TODO.}
    \label{tab:2_mca}
    \expandableinput{build/tab/2_mca.tex}
\end{table}

\begin{figure}
    \centering
    \includegraphics[width=\textwidth]{build/plt/2_mca.pdf}
    \caption{Messwerte und Fit zur Lebensdauer…}
    \label{fig:plt:2_mca}
\end{figure}


\FloatBarrier
\subsection{Theoretische Untergrundrate}


\FloatBarrier
\subsection{Bestimmung der mittleren Lebensdauer}
Um schließlich aus den Daten des \ac{MCA}
– aufgelistet in \autoref{tab:3_lebensdauer} –
die mittlere Lebensdauer zu bestimmen,
wird eine Exponentialfunktion mit einer zusätzlichen Variable für den Untergrund $U_2$ an die Daten angepasst:
\begin{equation*}
    N(t) = N_0 \cdot \exp (-t / \tau) + U_2 \; .
\end{equation*}
Die leeren Kanäle an den Rändern des Spektrums (begründet durch die Einstellung des \ac{TAC}) werden dabei nicht berücksichtigt.
Auch die leeren Kanäle innerhalb des Spektrums werden nicht berücksichtigt,
damit der Fit nach der Unsicherheit je Kanal gewichtet werden kann.
Die gefundenen Parameter lauten:
\begin{align*}
    N_0 &= \num{81.2 \pm 1.6} \\
    \tau &= \SI{1.91 \pm 0.04}{\micro\second} \\
    U_2 &= \num{0.85 \pm 0.19} \; .
\end{align*}
% Die mittlere Lebensdauer $\tau$ ergibt sich daraus zu
% \begin{equation*}
%     \tau = \frac{1}{\lambda} = \SI{7.2 \pm 0.5}{\micro\second} \; .
% \end{equation*}

\begin{figure}
    \centering
    \includegraphics[width=\textwidth]{build/plt/3_lebensdauer.pdf}
    \caption{Messwerte und Fit zur Impulszahl je Kanal.}
    \label{fig:plt:3_lebensdauer}
\end{figure}
