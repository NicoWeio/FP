\section{Auswertung}
\label{sec:auswertung}

% NOTE: Titel übernommen aus der Durchführung
\subsection{Justierung der Koinzidenzapparatur} \label{sec:auswertung:koinzidenz}
Um sicherzustellen,
dass fertigungsbedingte Unterschiede in der Reaktionszeit der \acp{PMT}
nicht zum ungewünschten Ausschluss von Ereignissen durch die Koinzidenz-Schaltung führen,
% dass die leicht verschieden auslösenden \acp{PMT}
% dass die Koinzidenz-Schaltung tatsächlich auslöst,
% wenn beide \ac{PMT} ein Signal liefern,
werden die Verzögerungsleitungen zwischen \acp{PMT} und Koinzidenz derart justiert,
dass die Ereignisrate am Ausgang maximal ist.
% Eine der Verzögerungsleitungen wird dabei auf Null belassen…
Für jede Einstellung wird über eine Zeit von $\SI{20}{\second}$ gemessen.
%
Die mit den Verzögerungsleitungen eingestellte Zeitverschiebung zwischen den \acp{PMT}
wird durch die vorzeichenbehaftete Variable $t_\text{diff}$ beschrieben.

Die in Impulsraten umgerechneten Messwerte sind in \autoref{tab:1_koinzidenz} aufgelistet.
In \autoref{fig:plt:1_koinzidenz} lässt sich das Maximum der Ereignisrate
bei einer Zeitdifferenz von $t_\text{diff} = \SI{-1}{\nano\second}$ erkennen.
Es beträgt \SI{19.8 \pm 1.0}{\per\second}.
%
Zu beiden Seiten des Maximums fällt die Ereignisrate näherungsweise linear ab.
Dies ist konsistent mit der Überlegung,
dass eine Gleichverteilung der Ereignisse (die Gleichverteilung der Orte impliziert eine Gleichverteilung der Zeitdifferenzen zwischen beiden \acp{PMT})
mit einem Koinzidenzfenster gefaltet wird.
Ein Plateau,
welches nach diesem Modell möglich wäre,
wird jedoch nicht eindeutig beobachtet.

Mittels linearer Regression werden die Parameter der Geraden bestimmt,
die die Ereignisrate links und rechts des Maximums beschreiben.
Es ergeben sich
\begin{align*}
m_l &= \SI{1.80 \pm 0.09}{\per\nano\second\per\second} &
b_l &= \SI{20.23 \pm 0.55}{\per\second}
\\
m_r &= \SI{-1.68 \pm 0.08}{\per\nano\second\per\second} &
b_r &= \SI{17.95 \pm 0.43}{\per\second}
\end{align*}
und somit die Halbwertsbreite
\[
    \frac{\frac{I_{\max}}{2} - b_r}{m_r}
    -
    \frac{\frac{I_{\max}}{2} - b_l}{m_l}
    = \SI{10.5(8)}{\nano\second} \; .
\]

\begin{table}
    \centering
    \caption{Messwerte zur Abhängigkeit der Ereignisrate von der Zeitverschiebung zwischen den \acp{PMT}.}
    \label{tab:1_koinzidenz}
    \expandableinput{build/tab/1_koinzidenz.tex}
\end{table}

\begin{figure}
    \centering
    \includegraphics[width=\textwidth]{build/plt/1_koinzidenz.pdf}
    \caption{Messwerte und Fit zur Abhängigkeit der Ereignisrate von der Zeitverschiebung zwischen den \acp{PMT}.}
    \label{fig:plt:1_koinzidenz}
\end{figure}


\FloatBarrier
\subsection{Kalibrierung des \acs{MCA}} \label{sec:auswertung:mca}
Die vom \ac{TAC} ausgegebenen Amplituden werden im \ac{MCA} in \num{512} Kanälen histogrammiert.
In Vorbereitung auf die eigentliche Messung muss daher
der Zusammenhang zwischen Zeitdifferenz und Kanalnummer bestimmt werden.
Dafür wird für verschiedene (am Doppelimpulsgenerator eingestellte) Zeitdifferenzen
die entsprechende Kanalnummer notiert.
Hierbei zeigt sich ein perfekt linearer Zusammenhang.
Die Messwerte sind in \autoref{tab:2_mca} aufgeführt und in \autoref{fig:plt:2_mca} dargestellt.

Eine lineare Regressionsrechnung ergibt
für eine Ausgleichsgerade $t(c) = a \cdot c + b$
die folgenden Parameter:
\begin{align*}
    a &=\SI{0.0217 \pm 0}{\micro\second} \\
    b &=\SI{0.1391 \pm 0}{\micro\second} \; .
\end{align*}

\begin{table}
    \centering
    \caption{Messwerte zur Kalibrierung des \ac{MCA}.}
    \label{tab:2_mca}
    \expandableinput{build/tab/2_mca.tex}
\end{table}

\begin{figure}
    \centering
    \includegraphics[width=\textwidth]{build/plt/2_mca.pdf}
    \caption{Messwerte und Fit zur Kalibrierung des \ac{MCA}.}
    \label{fig:plt:2_mca}
\end{figure}


\FloatBarrier
\subsection{Bestimmung der mittleren Lebensdauer} \label{sec:auswertung:lebensdauer}
Im letzten Versuchsteil nimmt der \ac{MCA} über einen Zeitraum von $t_\text{Messung} = \SI{159914}{\second}$ ($= \SI{44.4}{\hour}$) Daten auf.
Diese sind in \autoref{tab:3_lebensdauer} vollständig angegeben.


Die Stopp-Ereigniszahl entspricht der Summe aller Kanäle des \ac{MCA}, % NOTE: Abgesehen von Overflow-Events…
wohingegen die Start-Ereigniszahl aufgrund eines Fehlers Dritter aus den vorherigen Messungen abgeschätzt werden muss.
Dazu wird auf das Maximum aus \autoref{sec:auswertung:koinzidenz} zurückgegriffen.
Es sind
\begin{align*}
    % ↓ wegen der Umrechnung nicht ganzzahlig
    N_\text{Start} &= \num{3166297.20 \pm 159914.00} \\
    \text{beziehungsweise} \\
    % NOTE: „Die Stopp Anzahl ist ja ein fester gemessener Wert.
    %   Wenn ihr den Wert der Startsignale vom Counter wüsstet, dann hätte der auch keinen Fehler.
    %   Die Gespeicherten Werte entsprechen ja euren Einträgen im Histogramm, da nochmal einen statistischen Fehler drauf zu packen ist nicht sinnvoll.“
    N_\text{Stopp} &= \num{7630} \; .
\end{align*}
% Suchzeit: 10 µs (siehe Durchführung)

Daraus wird der durch mehrere Myonen im gleichen Suchfenster hervorgerufene Untergrund über die gesamte Messzeit
gemäß \autoref{eqn:untergrund} zu
$U_{1, \text{ges}} = \num{627.05 \pm 63.34}$ insgesamt
beziehungsweise
$U_1 = \num{1.47 \pm 0.15}$ pro (nicht-leerem) Kanal
bestimmt.
Er wird von allen Kanälen mit Ereignissen (\num{428} von \num{512}) gleichmäßig abgezogen.

Um schließlich die mittlere Lebensdauer zu erhalten,
wird eine Exponentialfunktion mit einer zusätzlichen Variable für den übrigen (von $t$ unabhängigen) Untergrund $U_2$
an die bereits um $U_1$ bereinigten Daten angepasst:
\begin{equation*}
    N(t) = N_0 \cdot \exp (-t / \tau) + U_2 \; .
\end{equation*}
Die leeren Kanäle an den Rändern des Spektrums (begründet durch die Einstellung des \ac{TAC}) werden dabei nicht berücksichtigt.
Auch die leeren Kanäle innerhalb des Spektrums werden nicht berücksichtigt,
damit der Fit nach der Unsicherheit je Kanal gewichtet werden kann.
\autoref{fig:plt:3_lebensdauer_linear} und \autoref{fig:plt:3_lebensdauer_log} zeigen die Messwerte und die Fit-Funktion
in linearer beziehungsweise logarithmischer Darstellung.
Die gefundenen Parameter lauten:
\begin{align*}
    N_0 &= \num{87.3 \pm 1.9} \\
    \tau &= \SI{1.91 \pm 0.04}{\micro\second} \\
    U_2 &= \num{-0.61 \pm 0.19} \; .
\end{align*}
Die mittlere Lebensdauer der Myonen wurde damit zu $\tau = \SI{1.91 \pm 0.04}{\micro\second}$ bestimmt.
% % ↓ leichte Dopplung
% $U_2$ liefert eine Schätzung für den Untergrund (bezogen auf einen einzelnen Kanal),
% der nach Entfernung des theoretischen Untergrunds durch mehrere Myonen im gleichen Suchfenster verbleibt.

\begin{figure}
    \centering
    \includegraphics[width=\textwidth]{build/plt/3_lebensdauer_linear.pdf}
    \caption{Messwerte und Fit zur Impulszahl je Kanal.}
    \label{fig:plt:3_lebensdauer_linear}
\end{figure}

\begin{figure}
    \centering
    \includegraphics[width=\textwidth]{build/plt/3_lebensdauer_log.pdf}
    \caption{Messwerte und Fit zur Impulszahl je Kanal im logarithmischen Maßstab.}
    \label{fig:plt:3_lebensdauer_log}
\end{figure}
