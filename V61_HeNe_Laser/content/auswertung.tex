\section{Auswertung}
\label{sec:auswertung}

\subsection{Stabilitätsbedingung}
\label{sec:auswertung:stabilitaetsbedingung}
% Aufgabe 2 in der Versuchsanleitung

In \autoref{sec:stabilitaet} wurde anhand der \hyperref[eqn:stabilitaetsbedingung]{Stabilitätsbedingung} die maximale theoretische Resonatorlänge berechnet.
Um diese experimentell zu überprüfen,
wird die Intensität des Lasers für verschiedene Resonatorlängen und Spiegelkonfigurationen gemessen.
\autoref{fig:plt:stabilitaetsbedingung} stellt diese Messwerte dar; die zugrundeliegenden Daten sind in \autoref{tab:stabilitaetsbedingung} aufgeführt.
% In \autoref{fig:plt:stabilitaetsbedingung} sind diese Messwerte dargestellt.

Es ist zu beachten, dass diejenigen Resonatorlängen, bei denen kein Lasern einsetzte, nicht angegeben sind.
Andererseits waren aufgrund der begrenzten Länge der optischen Bank keine größeren Resonatorlängen als etwa \SI{212.8}{\centi\meter} möglich.

\begin{figure}
  \centering
   \includegraphics[width=\textwidth]{build/plt/2_stabilitaetsbedingung.pdf}
   \caption{Lichtintensität $I$ in Abhängigkeit der Resonatorlänge $L$ für verschiedene Spiegelkonfigurationen.}
   \label{fig:plt:stabilitaetsbedingung}
\end{figure}

\begin{table}
\centering
\caption{Messwerte zur Lichtintensität $I$ in Abhängigkeit der Resonatorlänge $L$ für verschiedene Spiegelkonfigurationen.}
\label{tab:stabilitaetsbedingung}
\begin{subtable}{.5\textwidth}
    \centering
    \caption{\enquote{konkav + konkav}}
    \expandableinput{build/tab/2_stabilitaetsbedingung_konkav_konkav.tex}
\end{subtable}%
\begin{subtable}{.5\textwidth}
    \centering
    \caption{\enquote{plan + konkav}}
    \expandableinput{build/tab/2_stabilitaetsbedingung_plan_konkav.tex}
\end{subtable}
\end{table}


% TODO: Mit Theorie-Abschnitt vereinigen
% Durch Lösen von \autoref{eqn:TODO} können die theoretisch größtmöglichen Resonatorlängen $L$ bestimmt werden.
% Hierbei ist zu beachten, dass bei Gleichheit nur noch ein metastabiler Zustand vorliegt.
% \begin{align*}
%     L_\text{max, kk, theo} &= r_1 + r_2 \\
%     L_\text{max, pk, theo} &= r_2 \\
% \end{align*}


\subsection{TEM-Moden}
\label{sec:auswertung:tem_moden}
% Aufgabe TODO in der Versuchsanleitung

\subsubsection{\TEM{00}-Mode}
Aus \autoref{eqn:tem} geht hervor,
dass die Lichtintensität der \TEM{00}-Mode (bei Betrachtung senkrecht zur Strahlachse) durch eine Gauß-Funktion beschrieben wird.
Um die gemessene Intensitätsverteilung sinnvoll mit der theoretischen zu vergleichen,
werden dabei zusätzliche Parameter für Hintergrund und Verschiebung der Strahlachse eingesetzt.
Daraus ergibt sich \autoref{eqn:tem00_fitfn},
deren Parameter mithilfe von \scipycurvefit an die Messwerte angepasst werden.
\autoref{fig:plt:tem_00} stellt Messwerte und Theoriekurve grafisch dar;
die zugrundeliegenden Daten sind in \autoref{tab:mess_tem_00} aufgeführt.

\begin{equation}
  I =
  I_\text{max} \cdot \exp \left( \frac{-(r - r_0)^2}{2ω^2} \right) + I_0
  \label{eqn:tem00_fitfn}
\end{equation}


Es ergeben sich die folgenden Parameter:
\begin{align*}
  I_\text{max} &= \SI{131.5 \pm 1.1}{\micro\watt}
  \tag{maximale Intensität}
  \\
  I_0 &= \SI{1.7 \pm 0.5}{\micro\watt}
  \tag{Hintergrund, beispielsweise durch Streulicht}
  \\
  r_0 &= \SI{0.082 \pm 0.018}{\milli\meter}
  \tag{Mittelpunkt des Laserstrahls} % bzw. Verschiebung von Strahlachse
  \\
  ω &= \SI{1.968 \pm 0.021}{\milli\meter} % Vorzeichen ist irrelevant!
  \tag{Breite der Gauß-Verteilung}
  \\
\end{align*}

\begin{figure}
  \centering
   \includegraphics[width=\textwidth]{build/plt/3_tem_00.pdf}
   \caption{Lichtintensität $I$ in Abhängigkeit der Distanz $r$ zur optischen Achse für die \TEM{00}-Mode.}
   \label{fig:plt:tem_00}
\end{figure}


\subsubsection{\TEM{01}-Mode}
Die Lichtintensität der \TEM{01}-Mode wird (bei Betrachtung senkrecht zur Strahlachse) durch \autoref{eqn:tem01_fitfn} beschrieben,
wenn wie zuvor zusätzliche Parameter für Hintergrund und Verschiebung der Strahlachse eingesetzt werden.
Das Anpassen der Parameter an die Messwerte erfolgt erneut mit \scipycurvefit.
\autoref{fig:plt:tem_01} stellt Messwerte und Theoriekurve grafisch dar;
die zugrundeliegenden Daten sind in \autoref{tab:mess_tem_01} aufgeführt.

\begin{equation}
  I =
  I_\text{max} \cdot
  \frac{4(r - r_0)^2}{ω^2} \cdot
  \exp \left( \frac{-(r - r_0)^2}{2ω^2} \right) + I_0
  \label{eqn:tem01_fitfn}
\end{equation}

Es ergeben sich die folgenden Parameter:
\begin{align*}
  I_\text{max} &= \SI{3.10 \pm 0.08}{\micro\watt}
  \tag{maximale Intensität}
  \\
  I_0 &= \SI{1.13 \pm 0.09}{\micro\watt}
  \tag{Hintergrund, beispielsweise durch Streulicht}
  \\
  r_0 &= \SI{0.304 \pm 0.024}{\milli\meter}
  \tag{Mittelpunkt des Laserstrahls} % bzw. Verschiebung von Strahlachse
  \\
  ω &= \SI{1.141 \pm 0.017}{\milli\meter} % Vorzeichen ist irrelevant!
  \tag{Breite der Gauß-Verteilung}
  \\
\end{align*}

\begin{figure}
  \centering
   \includegraphics[width=\textwidth]{build/plt/3_tem_01.pdf}
   \caption{Lichtintensität $I$ in Abhängigkeit der Distanz $r$ zur optischen Achse für die \TEM{01}-Mode.}
   \label{fig:plt:tem_01}
\end{figure}

% --- für TEM00 und TEM01 ↓
\begin{table}[H]
\centering
\caption{Messwerte zur Lichtintensität in Abhängigkeit der Distanz zur optischen Achse für verschiedene \TEM{}-Moden.}
\label{tab:mess_tem}
\begin{subtable}{.5\textwidth}
    \centering
    \caption{\TEM{00}}
    \label{tab:mess_tem_00}
    \expandableinput{build/tab/3_tem_00.tex}
\end{subtable}%
\begin{subtable}{.5\textwidth}
    \centering
    \caption{\TEM{01}}
    \label{tab:mess_tem_01}
    \expandableinput{build/tab/3_tem_01.tex}
\end{subtable}
\end{table}


\subsection{Polarisation}
\label{sec:auswertung:polarisation}
Wie in \autoref{sec:versuchsaufbau} beschrieben,
ist eine nahezu vollständige p-Polarisation des Laserlichts zu erwarten.
Um diese Erwartung zu überprüfen,
wird die Lichtintensität in Abhängigkeit von der Ausrichtung eines Polarisationsfilters gemessen.

Eine theoretische Beschreibung ist durch \autoref{eqn:polarisation_fitfn} gegeben,
%TODO: Ist „Nullwinkel“ überhaupt ein Wort?
wenn zusätzliche Parameter für Hintergrund und Verschiebung des Nullwinkels eingesetzt werden.
Das Anpassen der Parameter an die Messwerte erfolgt erneut mit \scipycurvefit.
\autoref{fig:plt:polarisation} stellt Messwerte und Theoriekurve grafisch dar;
die zugrundeliegenden Daten sind in \autoref{tab:mess_polarisation} aufgeführt.

Es ergeben sich die folgenden Parameter:
\begin{align*}
  I_\text{max} &= \SI{860 \pm 1}{\micro\watt}
  \tag{maximale Intensität}
  \\
  I_0 &= \SI{-6 \pm 8}{\micro\watt}
  \tag{Hintergrund, beispielsweise durch Streulicht}
  \\
  \alpha_0 &= \SI{0.5 \pm 0.4}{\milli\meter}
  \tag{Nullwinkel}
  \\
\end{align*}

\begin{equation}
  I =
  I_\text{max} \cdot \sin(\alpha + \alpha_0)^2 + I_0
  \label{eqn:polarisation_fitfn}
\end{equation}



\begin{figure}
  \centering
   \includegraphics[width=\textwidth]{build/plt/4_polarisation.pdf}
   \caption{Lichtintensität $I$ in Abhängigkeit der Winkeleinstellung $\alpha$ des Polarisationsfilters.}
   \label{fig:plt:polarisation}
\end{figure}

\begin{table}
  \centering
  \caption{Messwerte zur Lichtintensität $I$ in Abhängigkeit des Polarisationswinkels $\alpha$.}
  \label{tab:mess_polarisation}
  \expandableinput{build/tab/4_polarisation.tex}
\end{table}


\subsection{Multimodenbetrieb und Frequenzspektrum des Lasers}
\label{sec:auswertung:frequenzspektrum}
% \subsection{Longitudinale Moden}
% 5. Multimodenbetrieb und Frequenzspektrum des Lasers
Für $L = \SI{162.9}{\centi\meter}$ wird mithilfe eines Oszilloskops das Frequenzspektrum des Lasers im Multimodenbetrieb ausgemessen,
wie in \autoref{fig:screenshot_oszilloskop} gezeigt.
Die ermittelten Peaks sind in \autoref{tab:frequenzspektrum} angegeben.
Die Distanzen $\symup{\Delta}f$ selbiger betragen
\SI{277}{\mega\hertz},
\SI{276}{\mega\hertz} und
\SI{277}{\mega\hertz}.

Der theoretische Modenabstand lässt sich anhand von \autoref{eqn:modenabstand} ermitteln,
wobei $c$ die Lichtgeschwindigkeit ist:
\begin{equation}
  \symup{\Delta}f_\text{theo} = \frac{c}{2L} \ .
  \label{eqn:modenabstand}
\end{equation}
Mit der oben angegebenen Resonatorlänge ist
\[
  \symup{\Delta}f_\text{theo} = \SI{920.2}{\mega\hertz} \ .
\]
Die beobachteten Moden sollen Vielfache von $\symup{\Delta}f_\text{theo}$ sein,
es soll also
\[
  f = n \cdot \symup{\Delta}f_\text{theo} = n\frac{c}{2L}
\]
gelten.


\begin{figure}
  \centering
   \includegraphics[width=0.75\textwidth]{content/img/5_frequenzspektrum_oszilloskop_inverted.jpg}
   \caption{Bildschirmaufnahme des Oszilloskops.}
   \label{fig:screenshot_oszilloskop}
\end{figure}

\begin{table}
  \centering
  \caption{
    Messwerte zu den Peaks im Frequenzspektrum.
    $f$ bezeichnet die Frequenz, $I$ die Intensität.
    Es wird dasjenige $n$ angegeben, welches $|f_\text{theo} - f|$ minimiert;
    dabei ist $f_\text{theo} = n \cdot \symup{\Delta}f_\text{theo}$.
  }
  % Intensität hier in dBm=Decibelmilliwatt
  \def\belmilliwatt{Bm}
  \label{tab:frequenzspektrum}
  \expandableinput{build/tab/5_frequenzspektrum.tex}
\end{table}



\subsection{Wellenlänge}
\label{sec:auswertung:wellenlaenge}
Um die Wellenlänge des Laserlichts zu bestimmen,
werden dessen Beugungsmaxima an verschiedenen optischen Gittern vermessen.

Durch Umformen von
\begin{equation*}
  k \lambda = d \sin(\alpha_k)
\end{equation*}
ergibt sich die praktisch anwendbare Bestimmungsgleichung
\begin{equation}
  \lambda = \frac{d a_k}{k \sqrt{e^2 + a_k^2}} \ ,
  \label{eqn:wellenlaenge}
\end{equation}
wobei $k$ die Ordnung des Beugungsmaximums,
$d$ der Spaltabstand (Gitterkonstante),
$e$ der Abstand zwischen Gitter und Schirm, % Oxford-Komma: Persönliche Präferenz, eigentlich nicht richtig.
und $a_k$ der Abstand des Beugungsmaximums zum Hauptmaximum (und somit zur optischen Achse)
ist.

Die Messwerte, anhand derer die Wellenlänge nun bestimmt wird, sind in \autoref{tab:mess_wellenlaenge} zu finden.
Dort sind auch die nach \autoref{eqn:wellenlaenge} bestimmten Wellenlängen für jede Messreihe angegeben.
% TODO ↓
Im Mittel über die vier Messreihen ergibt sich $\bar\lambda = \SI{426.85 \pm 7.44}{\micro\meter}$.

% Die Ordnung $k$ ist natürlich nicht wirklich vorzeichenbehaftet.
% TODO: In der Tabelle Vorzeichen streichen?

\begin{table}
  \centering
  \caption{
    Abstände der Interferenzmaxima der Ordnung $k$ vom Hauptmaximum für verschiedene optische Gitter.
    (Das Vorzeichen von $k$ ist nicht wirklich vorzeichenbehaftet.)
    Zusätzlich sind der Spaltabstand (inverse Gitterkonstante) $\sfrac{1}{d}$, der Abstand zum Schirm $e$ und die jeweils berechnete Wellenlänge $\lambda$ angegeben.
  }
  \label{tab:mess_wellenlaenge}
  \begin{tabular}{S S S S S}
  \toprule
  & \multicolumn{4}{c}{$\sfrac{1}{d} \mathbin{/} \si{\per\milli\meter}$} \\
  \cmidrule(lr){2-5}
  & 80 & 100 & 600 & 1200 \\
  \midrule

  & \multicolumn{4}{c}{$e \mathbin{/} \si{\centi\meter}$} \\
  \cmidrule(lr){2-5}
  & 90.5 & 90.5 & 63.0 & 33.0 \\
  \midrule

  {$k$} &
  \multicolumn{4}{c}{$a_k \mathbin{/} \si{\centi\meter}$} \\
  \midrule

  \expandableinput{build/tab/6_wellenlaenge.tex}
  \midrule

  & \multicolumn{4}{c}{$\lambda \mathbin{/} \si{\nano\meter}$} \\
  \cmidrule(lr){2-5}
  & 460.05 \pm 5.46 & 643.74 \pm 7.53 & 275.09 \pm 23.27 & 328.51 \pm 16.07 \\
  \bottomrule
  \end{tabular}
\end{table}
