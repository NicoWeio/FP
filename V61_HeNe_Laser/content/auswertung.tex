\section{Auswertung}
\label{sec:auswertung}

\subsection{Stabilitätsbedingung}
% Aufgabe 2 in der Versuchsanleitung

In \autoref{sec:stabilitaet} wurde anhand der \hyperref[eqn:stabilitaetsbedingung]{Stabilitätsbedingung} die maximale theoretische Resonatorlänge berechnet.
Um diese experimentell zu überprüfen,
wird die Intensität des Lasers für verschiedene Resonatorlängen und Spiegelkonfigurationen gemessen.
In \autoref{fig:plt:stabilitaetsbedingung} sind diese Messwerte dargestellt.
% TODO: Tabelle

Dabei ist zu beachten, dass diejenigen Resonatorlängen, bei denen kein Lasern einsetzte, nicht angegeben sind.
Andererseits waren aufgrund der begrenzten Länge der optischen Bank keine größeren Resonatorlängen als etwa \SI{212.8}{\centi\meter} möglich.

\begin{figure}
  \centering
   \includegraphics[width=\textwidth]{build/plt/2_stabilitaetsbedingung.pdf}
   \caption{Intensität $I$ in Abhängigkeit der Resonatorlänge $L$ für verschiedene Spiegelkonfigurationen.}
   \label{fig:plt:stabilitaetsbedingung}
\end{figure}

% TODO: Mit Theorie-Abschnitt vereinigen
Durch Lösen von \autoref{eqn:TODO} können die theoretisch größtmöglichen Resonatorlängen $L$ bestimmt werden.
Hierbei ist zu beachten, dass bei Gleichheit nur noch ein metastabiler Zustand vorliegt.
\begin{align*}
    L_\text{max, kk, theo} &= r_1 + r_2 \\
    L_\text{max, pk, theo} &= r_2 \\
\end{align*}


\subsection{TEM-Moden}
% Aufgabe 2 in der Versuchsanleitung
\lipsum[1]

\begin{figure}
  \centering
   \includegraphics[width=\textwidth]{build/plt/3_tem_00.pdf}
   \caption{Lichtintensität in Abhängigkeit der Distanz zur optischen Achse für die \TEM{00}-Mode.}
   \label{fig:plt:tem_00}
\end{figure}

\begin{figure}
  \centering
   \includegraphics[width=\textwidth]{build/plt/3_tem_01.pdf}
   \caption{Lichtintensität in Abhängigkeit der Distanz zur optischen Achse für die \TEM{01}-Mode.}
   \label{fig:plt:tem_01}
\end{figure}


\subsection{Polarisation}
\lipsum[1]

\begin{figure}
  \centering
   \includegraphics[width=\textwidth]{build/plt/4_polarisation.pdf}
   \caption{Lichtintensität in Abhängigkeit der Winkeleinstellung $\alpha$ des Polarisationsfilters.}
   \label{fig:plt:polarisation}
\end{figure}


\subsection{Multimodenbetrieb und Frequenzspektrum des Lasers}
% \subsection{Longitudinale Moden}
% 5. Multimodenbetrieb und Frequenzspektrum des Lasers


\subsection{Wellenlänge}
\lipsum[1]

