\section{Auswertung}
\label{sec:auswertung}

\section{Stabilitätsbedingung}
% Aufgabe 2 in der Versuchsanleitung
\lipsum[1]

\begin{figure}
  \centering
   \includegraphics[width=\textwidth]{build/plt/2_stabilitaetsbedingung_theorie.pdf}
   \caption{Stabilitätsparameter $g_1 \cdot g_2$ in Abhängigkeit der Resonatorlänge $L$ für verschiedene Spiegelkonfigurationen.}
   \label{fig:plt:stabilitaetsbedingung_theorie}
\end{figure}

Durch Lösen von \autoref{eqn:TODO} können die theoretisch größtmöglichen Resonatorlängen $L$ bestimmt werden.
Hierbei ist zu beachten, dass bei Gleichheit nur noch ein metastabiler Zustand vorliegt.
\begin{align*}
    L_\text{max, kk, theo} &= r_1 + r_2 \\
    L_\text{max, pk, theo} &= r_2 \\
\end{align*}


\section{TEM-Moden}
% Aufgabe 2 in der Versuchsanleitung
\lipsum[1]

\begin{figure}
  \centering
   \includegraphics[width=\textwidth]{build/plt/3_tem_00.pdf}
%    \caption{Lichtintensität in Abhängigkeit der Distanz zur optischen Achse für die TEM₀₀-Mode.}
   \caption{Lichtintensität in Abhängigkeit der Distanz zur optischen Achse für die $\text{TEM}_{00}$-Mode.}
   \label{fig:plt:tem_00}
\end{figure}

\begin{figure}
  \centering
   \includegraphics[width=\textwidth]{build/plt/3_tem_01.pdf}
%    \caption{Lichtintensität in Abhängigkeit der Distanz zur optischen Achse für die TEM₀₁-Mode.}
   \caption{Lichtintensität in Abhängigkeit der Distanz zur optischen Achse für die $\text{TEM}_{01}$-Mode.}
   \label{fig:plt:tem_01}
\end{figure}


\section{Polarisation}
\lipsum[1]


\section{Multimodenbetrieb und Frequenzspektrum des Lasers}
% \section{Longitudinale Moden}
% 5. Multimodenbetrieb und Frequenzspektrum des Lasers


\section{Wellenlänge}
\lipsum[1]

