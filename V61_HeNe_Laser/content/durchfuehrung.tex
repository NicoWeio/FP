\section{Durchführung}
\label{sec:durchfuehrung}

    Im folgenden Abschnitt werden der Aufbau und die Durchführung des Versuches beschrieben.

\subsection{Versuchsaufbau}
\label{sec:versuchsaufbau}

    Der Grundaufbau besteht aus einer Metallschiene,
    auf der auf der einen Seite ein Justierlaser mit grünem Licht befestigt ist.
    In der Mitte befindet sich der ein Laserrohr,
    welches von zwei Spiegeln umgeben ist,
    die den Resonator bilden.
    Auf der anderen Seite der Schiene ist eine Photodiode befestigt,
    welche die Intensität des Laserstrahls misst.
    %\begin{figure}
    %   \centering
    %   \include[text=\textwidth]{}
    %   \caption{Aufbau mit Laserrohr und Resonator.}
    %   \label{fig:aufbau}
    %\end{figure}
    Die \autoref{fig:aufbau} zeigt den Aufbau.
    Zusätzlich stehen Blenden, optische Gitter, eine Linse und ein Polarisationsfilter zur Verfügung.

\subsection{Einjustierung des He-Ne-Lasers}

    Für die Justierung wird zunächst nur der Justierlaser betrachet.
    Vor dem Justierlaser befindet sich eine Blende,
    auf der ein Fadenkreuz aufgezeichnet ist.
    Mithilfe der Schrauben an den Halterungen der Resonatorspiegel werden diese nun horizontal und vertikal so ausgerichtet,
    dass die von den Spiegeln reflektierten Punkte mittig auf das Zentrum des Fadenkreuzes treffen.
    Ist die richtige Einstellung gefunden,
    fängt der He-Ne-Laser an zu strahlen und es ist ein roter Laserstrahl erkennbar.
    Der Justierlaser kann dann abgeschaltet werden.\\
    \\
    Für die folgenden Messungen werden konkave Resonatorspiegel mit einem Radius von $\SI{1400}{\milli\meter}$ verwendet.

\subsection{Ausmessung der Wellenlänge des He-Ne-Lasers}

    Nach Einjustierung des Lasers wird nun ein optisches Gitter zwischen den rechten Resonatorspiegel und die Photodiode eingefügt.
    Die Photodiode selbst ist hier nicht notwenig und wird deshalb entfernt.
    Die Aufspaltung des Strahls durch das Gitter ist auf einem Schirm zu erkennen.
    Mithilfe eines Maßbandes werden nun für vier verschiedene Gitter die Abstände zwischen den Beugungsmaxima gemessen,
    sowie der Abstand zwischen Gitter und Schirm bestimmt.

\subsection{Polarisation}

    Für diese Messung wird die Photodiode wieder auf der Schiene befestigt,
    sowie ein Polarisationsfilter zwischen Resonatorspiegel und Photodiode eingebaut.
    Nun wird der Polarisationsfilter in Schritten von $\SI{10}{\degree}$ verstellt und die jeweilige Intensität mithilfe der Photodiode gemessen.
    Es werden volle $\SI{360}{\degree}$ durchlaufen.

\subsection{Stabilitätsbedingung und Messung der Moden}

    In diesem Versuchsteil sollen die maximal möglichen Resonatorlängen,
    die in \autoref{sec:stabilitaet} berechnet wurden,
    überprüft werden.
    Dazu wird der Abstand der Resonatorspiegel langsam abwechselnd erhöht und dabei die Intensität gemessen.
    Der Abstand selbst wird wieder mit einem Maßband vermessen.
    Nachdem ein Spiegel verschoben wurde,
    muss neu justiert werden.
    Dazu wird an dem Spiegel,
    der zuletzt bewegt wurde,
    entweder die horizontale oder die vertikale Schraube so verstellt,
    dass die Intensität maximal wird.
    Dann wird an dem anderen Spiegel die entsprechende Schraube verstellt.
    Anschließend wird die zweite Schraube des ersten Spiegels eingestellt,
    dann die entsprechende Schraube am anderen Spiegel.
    Die Schrauben werden immer so verstellt,
    dass die gemessene Intensität des Strahls maximal wird.
    Anschließend wird der Abstand der Spiegel wieder vergößert.
    Wenn im Laufe dessen der Strahl des He-Ne-Lasers verloren geht,
    kann mithilfe des Justierlasers nachjustiert werden,
    bis der rote Strahl wieder auftritt.\\
    Wenn eine Laser-Intensität von $\SI{5}{\milli\watt}$ erreicht ist,
    können die Moden ausgemessen werden.
    In diesem Fall ergab sich bei einer Resonatorlänge von $\SI{162.9}{\centi\meter}$ eine Intensität von $\SI{5.1}{\milli\watt}$.

\subsubsection{Die longitudinale Mode}

    Zur Messung der longitudinalen Mode wird eine mit einem Oszilloskop verbundene Photozelle verwendet.
    Diese muss so ausgerichtet werden,
    dass der Strahl genau auf das Eingangsfenster der Diode fällt.
    Auf dem Oszilloskop sind nun Peaks zu sehen,
    deren x- und y- Position mithilfe des Cursers vermessen werden.

\subsubsection{Die transversalen Moden}

    Nun sollen die transversalen \TEM{00}- und \TEM{01}-Moden ausgemessen werden.
    Dazu wird zwischen Resonatorspiegel und die Photodiode eine Vergrößerungslinse und eine Blende eingebaut,
    welche vor und zurück um $\SI{10}{\milli\meter}$ verschoben werden kann.\\
    Die \TEM{00}-Mode kennzeichnet sich durch einen runden Lichtfleck auf dem Schirm aus.
    Unter Verschiebung der Blende kann nun die Intensitätsverteilung dieser Mode vermessen werden.\\
    Für die \TEM{01}-Mode musste die Resonatorlänge auf $\SI{178.8}{\centi\meter}$ mit einer Intensität von $\SI{5.6}{\milli\watt}$ erhöht werden.
    Um die Mode herauszufiltern,
    wird ein gespannter Goldfaden zwischen Laserrohr und Resonatorspiegel befestigt,
    welcher so in den Laserstrahl gelenkt werden muss,
    dass sich das in \autoref{sec:moden} dargestellte Bild ergibt.
    Auch hier wird die Intensitätsverteilung mit Verschieben der Blende vermessen.\\
    \\
    Nach dem Vermessen der Moden wird die Überprüfung der Stabilitätsbedingung bis zum maximal einstellbaren Resonatorlänge fortgeführt.\\
    Abschließend wird die Messung der Stabilitätsbedingung für einen weiteren Resonator wiederholt,
    indem einer der konkaven Spiegel durch einen planen Spiegel ausgetauscht wird.
