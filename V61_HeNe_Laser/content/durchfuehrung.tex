\section{Durchführung}
\label{sec:durchfuehrung}

    Im folgenden Abschnitt werden der Aufbau und die Durchführung des Versuches beschrieben.

\subsection{Versuchsaufbau}
\label{sec:versuchsaufbau}

    Der Grundaufbau besteht aus einer optischen Bank,
    auf der einen Seite ein Justierlaser mit grünem Licht befestigt ist.
    In der Mitte befindet sich das mit einem Helium-Neon-Gasgemisch gefüllte Laserrohr,
    welches von zwei Spiegeln umgeben ist,
    die den Resonator bilden.
    An beiden Enden des Laserrohrs sind sogenannte Brewster-Fenster angebracht,
    die die Polarisation des Laserstrahls bestimmen.
    Das entstehende Laserlicht ist zu Beginn senkrecht und parallel polarisiert.
    Wenn das Licht im Brewster-Winkel $\alpha_\text{p}$ auf die Grenzfläche der Fenster trifft,
    ist der reflektierte Strahl vollständig s-polarisiert und der gebrochene Strahl größtenteils p-polarisiert.
    Das führt dazu,
    dass der beobachtete Laserstrahl größtenteils p-polarisiert ist,
    wobei die p-polarisierte Mode bei der Messung bevorzugt wird.\\
    Auf der anderen Seite der optischen Bank ist eine Photodiode befestigt,
    welche die Intensität des Laserstrahls misst.
    \autoref{fig:aufbau} zeigt den Aufbau.
    Zusätzlich stehen Blenden, optische Gitter, eine Linse und ein Polarisationsfilter zur Verfügung.

    \begin{figure}
      \centering
      \includegraphics[width=\textwidth]{content/img/Abb_1.pdf}
      \caption{Aufbau mit Laserrohr und Resonator. \cite{versuchsanleitung}}
      \label{fig:aufbau}
    \end{figure}


\subsection{Einjustierung des He-Ne-Lasers}

    Für die Justierung wird zunächst nur der Justierlaser betrachtet.
    Vor dem Justierlaser befindet sich eine Lochblende,
    auf der ein Fadenkreuz aufgezeichnet ist.
    Mithilfe der Schrauben an den Halterungen der Resonatorspiegel werden diese nun horizontal und vertikal so ausgerichtet,
    dass die von den Spiegeln reflektierten Punkte mittig auf das Zentrum des Fadenkreuzes treffen.
    % Man sieht vier Punkte; je Spiegel kann Licht an der Grenzfläche Luft—Spiegel oder Spiegel—Luft reflektiert werden.
    Ist eine ausreichend gute Einstellung gefunden,
    beginnt der He-Ne-Laser zu strahlen
    und ein roter Laserstrahl ist erkennbar.
    Der Justierlaser kann dann abgeschaltet werden.


    Für die folgenden Messungen werden konkave Resonatorspiegel mit einem Radius von \SI{1400}{\milli\meter} verwendet.


\subsection{Ausmessung der Wellenlänge des He-Ne-Lasers}

    Nach Einjustierung des Lasers wird nun ein optisches Gitter am rechten Ende des Aufbaus eingesetzt.
    Die Photodiode wird entfernt, da keine Intensitätsmessung vorgenommen werden soll.
    Die Aufspaltung des Strahls durch das Gitter ist auf einem Schirm zu erkennen.
    Mithilfe eines Maßbandes werden nun für vier verschiedene Gitter die Abstände zwischen den Beugungsmaxima gemessen
    und der Abstand zwischen Gitter und Schirm notiert.


\subsection{Polarisation}

    Für diese Messung wird die Photodiode wieder auf der optischen Bank befestigt,
    sowie ein Polarisationsfilter zwischen Resonatorspiegel und Photodiode eingebaut.
    Nun wird der Polarisationsfilter in Schritten von \SI{10}{\degree} verstellt und die jeweilige Intensität mithilfe der Photodiode gemessen.
    Es werden volle \SI{360}{\degree} durchlaufen.


\subsection{Stabilitätsbedingung und Messung der Moden}

    In diesem Versuchsteil sollen die maximal möglichen Resonatorlängen,
    die in \autoref{sec:stabilitaet} berechnet wurden,
    überprüft werden.
    Dazu wird der Abstand der beiden Resonatorspiegel im Wechsel langsam erhöht und dabei die Intensität gemessen.
    Der Abstand selbst wird wieder mit einem Maßband vermessen.
    Nachdem ein Spiegel verschoben wurde,
    muss neu justiert werden.
    Dazu wird an dem Spiegel,
    der zuletzt bewegt wurde,
    entweder die horizontale oder die vertikale Schraube so verstellt,
    dass die Intensität maximal wird.
    Dann wird an dem anderen Spiegel die entsprechende Schraube verstellt.
    Anschließend wird die zweite Schraube des ersten Spiegels eingestellt,
    dann die entsprechende Schraube am anderen Spiegel.
    Die Schrauben werden immer so verstellt,
    dass die gemessene Intensität des Strahls maximal wird.
    Anschließend wird der Abstand der Spiegel wieder vergrößert.
    Wenn im währenddessen der Strahl des He-Ne-Lasers verloren geht,
    kann mithilfe des Justierlasers nachjustiert werden,
    bis der rote Strahl wieder auftritt.

    Wenn eine Laser-Intensität von \SI{5}{\milli\watt} erreicht ist,
    können die Moden ausgemessen werden.
    In diesem Fall ergab sich bei einer Resonatorlänge von \SI{162.9}{\centi\meter} eine Intensität von \SI{5.1}{\milli\watt}.


\subsubsection{Die longitudinale Mode}

    Zur Messung der longitudinalen Mode wird eine mit einem Oszilloskop verbundene Photozelle verwendet.
    Diese muss so ausgerichtet werden,
    dass der Strahl genau auf das Eingangsfenster der Diode fällt.
    Auf dem Oszilloskop wird das Frequenzspektrum dargestellt.
    Die $x$- und $y$-Positionen der dabei sichtbaren Peaks
    werden mithilfe des Cursors vermessen werden.


\subsubsection{Die transversalen Moden}

    Nun sollen die transversalen Moden \TEM{00} und \TEM{01} ausgemessen werden.
    Dazu werden zwischen Resonatorspiegel und Photodiode eine Vergrößerungslinse und eine Blende eingebaut,
    welche um \SI{10}{\milli\meter} senkrecht zur Strahlrichtung verschoben werden kann.

    Die \TEM{00}-Mode ist durch einen runden Lichtfleck auf dem Schirm charakterisiert.
    Unter Verschiebung der Blende kann nun die Intensitätsverteilung dieser Mode vermessen werden.

    Um die \TEM{01}-Mode zu vermessen,
    wird die Resonatorlänge auf \SI{178.8}{\centi\meter} mit einer Intensität von \SI{5.6}{\milli\watt} erhöht.
    Es wird ein gespannter Goldfaden zwischen Laserrohr und Resonatorspiegel befestigt,
    welcher so in den Laserstrahl gelenkt werden muss,
    dass sich das in \autoref{fig:theorie:tem_00} dargestellte Bild ergibt.
    Auch hier wird die Intensitätsverteilung durch Verschieben der Blende vermessen.


    % ↓ Gehört zu „Stabilitätsbedingung und Messung der Moden“.
    %   Wird hier belassen, damit klar ist, dass alle anderen Teile mit dem konkav-konkav-Setup durchgeführt wurden.
    Nach dem Vermessen der Moden wird die Überprüfung der Stabilitätsbedingung bis zur maximal einstellbaren Resonatorlänge fortgeführt.

    Abschließend wird die Messung der Stabilitätsbedingung für einen weiteren Resonator wiederholt,
    indem einer der konkaven Spiegel durch einen planen Spiegel ausgetauscht wird.
