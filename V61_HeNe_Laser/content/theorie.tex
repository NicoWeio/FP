\section{Theorie}
\label{sec:theorie}

    Im folgenden Abschnitt werden die theoretischen Grundlagen eines Lasers (light amplification by stimulated emission of radiation) beschrieben.

\subsection{Bestandteile und Funktionsweise eines Lasers}

    Ein Laser besteht im Wesentlichen aus drei Bestandteilen,
    dem aktiven Medium, einer Energiepumpe und einem Resonator.
    Der Resonator soll in \autoref{sec:stabilitaet} näher beschrieben werden.
    In einem Laserrohr sind aktives Medium und Energiepumpe zusammengefasst,
    welche meist durch zwei Gase realisiert sind.
    Mithilfe der Energiepumpe sollen die Atome des aktiven Medium durch Stöße angeregt werden,
    sodass sich eine sogenannte \textit{Besetzungsinversion} einstellt.
    Damit ist gemeint,
    dass sich im aktiven Medium mehr Elektronen auf einem höheren Energieniveau als auf einem niedrigeren befinden.
    Mithilfe von Spontaner oder Stimulierter Emission,
    welche in \autoref{fig:absorption_emission} dargestellt sind,
    fallen die Elektronen wieder auf das niedrigere Energieniveau zurück und die entsprechende Energiedifferenz wird in Form eines Photons mit der entsprechenden Wellenlänge frei.
    Je größer die Besetzungsinversion ist,
    also je mehr Elektronen sich auf einem höheren Niveau befinden,
    desto größer ist die Verstärkung des Laserlichts,
    da einfach mehr Elektronen in den niedrigeren Zustand zurückkehren.
    %\begin{figure}[H]
    %    \centering
    %    \includegraphics[width=\textwidth]{}
    %    \caption{Schematische Darstellung von Absorption, Spontaner und Stimulierter Emission in einem zwei-Niveau-System.}
    %    \label{fig:absorption_emission}
    %\end{figure}
    Aufgrund der Maxwell-verteilten Besetzung der Elektronen in einem zwei-Niveau-System kann in diesem System keine Besetzungsinversion erreicht werden,
    für einen Laser sind also mehr als zwei Niveaus erforderlich.\\
    \\
    Im Falle des He-Ne-Lasers wird das aktive Medium durch Neon realisiert,
    die Energiepumpe durch Helium-Atome,
    welche zu Beginn durch Stöße mit Elektronen angeregt werden und die überschüssige Energie bei einem Stoß 2. Art an die Neon-Atome übertragen.
    Der He-Ne-Laser ist ein drei-Niveau-Laser,
    es werden die oberen $2s$- und $3s$-Niveaus im Neon besetzt.
    Aufgrund der Auswahlregel $\symup{\Delta} l = \pm 1$ können nur Übergänge in niedrigere $p$-Niveaus stattfinden.
    Der hier beobachtete Übergang $3s_2 \to 2p_4$ entspricht einer Roten Linie mit der Wellenlänge $\lambda = \SI{633}{\nano\meter}$.

\subsection{Stabilität der Resonatoren}
\label{sec:stabilitaet}

    Der Resonator besteht aus zwei Spiegeln und umgibt das Laserrohr,
    indem sich aktives Medium und Energiepumpe befinden.
    Er dient dazu,
    das emittierte Licht zu reflektieren,
    sodass sich zwischen den Resonatorspiegeln eine stehende Welle ausbildet.
    Der Resonator soll aufgrunddessen einen selbstschwingenden Oszillator bilden,
    da das Licht der stehenden Welle wieder auf die Neon-Atome trifft und so selbst wieder Atome anregt.
    Dieses Phänomen wird Absorption genannt und ist ebenfalls in \autoref{fig:absorption_emission} dargestellt.\\
    Damit der Resonator eine stehende Welle ausbilden kann,
    also stabil ist,
    muss die sogenannte \textit{Stabilitätsbedingung}
    \begin{equation}
        0 < g_i \cdot g_j < 1
        \label{eqn:stabilitaetsbedingung}
    \end{equation}
    erfüllt sein.
    Die Parameter $g_i, g_j$ werden mithilfe von
    \begin{equation}
        g = 1 - \frac{L}{r_i}
        \label{eqn:stabilitaetsparameter}
    \end{equation}
    berechnet,
    wobei $L$ die Resonatorlänge und $r_i$ der Radius des verwendeten Spiegels ist.
    Für zwei plane Spiegel mit dem Radius $r = \infty$ wäre das Produkt $g_ig_j = 1$ und der Resonator damit metastabil.
    Auch in diesem Fall kann der Laser trotzdem funktionieren.
    Für die hier verwendeten Resonatoren,
    bestehend aus zwei konkaven Spiegeln mit Radius $r = \SI{1400}{\milli\meter}$ im ersten Fall und einem planen und einem konkaven Spiegel mit Radius $r = \SI{1400}{\milli\meter}$ im zweiten Fall,
    ergibt sich nach \autoref{eqn:stabilitaetsbedingung} im ersten Fall eine maximale Länge von $L_\text{konkav,konkav} = \SI{2800}{\milli\meter}$ und im zweiten Fall $L_\text{plan,konkav} = \SI{1400}{\milli\meter}$.
    Der Verlauf der Stabilitätsbedingung $g_i \cdot g_j$ als Funktion der Resonatorlänge $L$ ist in \autoref{fig:resonatorlaenge_plot} für verschiedene Kombinationen von Spiegeln dargestellt.
    %\begin{figure}[H]
    %    \centering
    %    \include[width=\textwidth]{}
    %    \caption{Verlauf der Stabilität von verschiedenen Resonatoren als Funktion der Resonatorlänge.
    %            Für die Längen, in denen $0 < g_i g_j < 1$ gilt, ist der Resonator stabil.}
    %    \label{fig:resonatorlaenge_plot}
    %\end{figure}

\subsection{Transversale und longitudinale Moden in einem Laser}
\label{sec:moden}

    In einem offenen Resonator bildet sich eine stehende Welle aus,
    die aus longitudinalen und transversalen Moden besteht.
    Longitudinale Moden sind dabei längs der Ausbreitungsrichtung,
    die transversalen Moden senkrecht zur Ausbreitungsrichtung.
    Die transversalen Moden werden auch $TEM_{mn}$-Moden genannt,
    "Transversal electromagnetic".
    Der Index $mn$ bezeichnet die Anzahl der Knoten in der hier xy-Ebene.\\
    In diesem Versuch sollen die $TEM_{00}$- und die $TEM_{01}$-Mode untersucht werden.
    Die $TEM_{00}$-Mode wird auch als \textit{Fundamentalmode} bezeichnet.
    Ihre Intensitätsverteilung hat die Form einer Gauß-Kurve,
    wie in \autoref{fig:tem00} erkennbar.
    %TODO: Ich dachte hier an die Abb. aus dem Demtröder, S.170, Abb 5.8
    %Könnte dann auch eine Abbildung werden.
    %\begin{figure}
    %    \centering
    %    \include[width=\textwidth]{}
    %    \caption{Intensitätsverteilung der $TEM_{00}$-Mode.\cite{}}
    %    \label{fig:tem00}
    %\end{figure}
    Die Intensitätsverteilung der $TEM_{01}$-Mode ist in \autoref{fig:tem01} dargestellt.
    %\begin{figure}
    %    \centering
    %    \include[width=\textwidth]{}
    %    \caption{Intensitätsverteilung der $TEM_{01}$-Mode.\cite{}}
    %    \label{fig:tem01}
    %\end{figure}
    Die unterschiedlichen Moden können mithilfe einer Modenblende herausgefiltert werden.

\subsection{Polarisation des Lasers}

    Das entstehende Laserlicht ist zu Beginn senkrecht und parallel polarisiert.
    Mithilfe sogenannter Brewster-Fenster wird nun ein Teil herausgefiltert.
    Wenn das Licht im Brewster-Winkel $\alpha_\text{p}$ auf die Grenzfläche der Fenster trifft,
    ist der reflektierte Strahl vollständig s-polarisiert und der gebrochene Strahl größtenteils p-polarisiert.
    Das führt dazu,
    dass der beobachtete Laserstrahl größtenteils p-polarisiert ist,
    wobei die p-polarisierte Mode bei der Messung bevorzugt wird.