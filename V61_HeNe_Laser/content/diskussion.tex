\section{Diskussion}
\label{sec:diskussion}

\subsection{Abweichungen}

…


\subsection{Mögliche Fehlerquellen}
% Hab erstmal alles aufgeschrieben, was mir so eingefallen ist.

    Bei der Ausmessung der Wellenlänge durch Messung der Abstände der Beugungsmaxima wurde zur Messung ein einfaches Maßband verwendet.
    Da die Beugungsmaxima auf einen großen Schirm gestrahlt wurde und der Raum vollständig abgedunkelt war,
    ist es gut möglich,
    dass beim Ablesen der Abstände Fehler aufgetreten sind.
    Die Maxima hatten zudem eine gewisse Ausdehnung,
    welche beim Abstand nicht berücksichtigt wurde.
    Gerade bei den optischen Gittern mit vielen Linien pro $\si{\milli\meter}$ waren nur drei Maxima zu erkennen,
    da das Gitter auf der Schiene nicht weiter verschoben werden konnte.
    Für diese Fälle stehen demnach weniger Messwerte zur Verfügung.\\
    Bei der Vermessung der Moden konnte die verwendete Blende nur $\SI{10}{\milli\meter}$ in beide Richtungen verschoben werden,
    sodass möglicherweise nicht der ganze Modenverteilen ausgemessen werden konnte.
    Zudem gelang es erst bei höheren Intensitäten die $TEM_{01}$-Mode herauszufiltern und in eine ungefähre Position zu verschieben,
    damit sie ausgemessen werden konnte.
    Es hätte bestimmt eine bessere Position geben können,
    um die Mode vollständig aufnehmen zu können.
    Bei der longitudinalen Mode konnte das Oszilloskop nicht gut verstellt werden,
    sodass möglicherweise nicht angezeigte Peaks nicht mit aufgenommen wurden.\\
    Für die Überprüfung der Stabilitätsbedingung wurde der Abstand zwischen den Resonatorspiegeln ebenfalls nur mit einem Maßband vermessen,
    es können also auch hier wieder Messfehler aufgetreten sein.
    Zudem wurde das Maßband durch das zwischenstehende Laserrohr leicht nach oben gebogen,
    sodass sich keine Gerade ergab.
    Der in \autoref{sec:stabilitaet} berechnete,
    theoretische Wert für den Resonator aus zwei konkaven Spiegeln konnte aufgrund der nicht ausreichenden Länge der Metallschiene nicht erreicht werden.\\
    Insgesamt ist noch zu sagen,
    dass die Spiegel zwischendurch gereinigt werden mussten,
    da aufgrund von Staubpartikeln in der Luft kein Laserstrahl entstehen konnte.
    Diese Staubpartikel können auch bei funktionierendem Laser die Intensität beeinflusst haben.
    Das Verstellen der Spiegel für eine maximale Intensität geschah nach Fingergefühl und Augenmaß,
    es könnte also eine bessere Spiegelkonfiguration zur Messung der Intensitäten gegeben haben.

% TODO: Dies waren meine Ideen. Sie sind ggf. noch in den Text zu integrieren.
\begin{itemize}
    \item Manuelles Nachführen der Photodiode
    \item Ablesen der Beugungsmaxima von der Seite
    \item Streu-/Restlicht im Raum
    \item Starke Schwankungen der gemessenen Intensität
    \item Stabilitätsbedingung: Begrenzte Zeit, um Spiegel jeweils einzujustieren
\end{itemize}
