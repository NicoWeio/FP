\section{Diskussion}
\label{sec:diskussion}

\subsection{Abweichungen}
    Die \hyperref[sec:auswertung:stabilitaetsbedingung]{Stabilitätsbedingung} konnte nur insofern verifiziert werden,
    als dass das theoretische Maximum der Resonatorlänge
    für die Konfiguration \enquote{plan + konkav} erreicht wurde
    und für die Konfiguration \enquote{konkav + konkav} aufgrund der begrenzten Länge der optischen Bank nicht erreicht werden konnte.\\
    Es lässt sich nicht ausschließen,
    dass der Laser in der Konfiguration \enquote{plan + konkav} auch für größere Resonatorlängen funktioniert hätte,
    da das Nichtvorhandensein eines Laserstrahls ebenso die Folge von unzureichender Justage hätte sein können.

    Die gewünschten \hyperref[sec:auswertung:tem_moden]{TEM-Moden} wurden erfolgreich vermessen.
    Das theoretische Modell der Intensitätsverteilung passt
    – unter Berücksichtigung der Störeinflüsse –
    gut zu den jeweiligen Messungen,
    wie in den Abbildungen \ref{fig:plt:tem_00} und \ref{fig:plt:tem_01} zu sehen ist.
    Die Unsicherheiten in den Fit-Parametern sind \autoref{sec:auswertung:tem_moden} zu entnehmen.


    % Polarisation


    % Multimodenbetrieb

    % Wellenlänge
    Die \hyperref[sec:auswertung:wellenlaenge]{Wellenlänge} des Lasers konnte mit verschiedenen Gittern unterschiedlich gut gemessen werden.
    Der zugehörige Theoriewert ist $\lambda = \SI{632.816}{\nano\meter}$.
    % Sowohl für das gröbste ($\sfrac{1}{d} = \SI[per-mode=reciprocal]{80}{\per\milli\meter}$)
    % als auch für das feinste ($\sfrac{1}{d} = \SI[per-mode=reciprocal]{1200}{\per\milli\meter}$; nur zwei Messwerte) Gitter
    % sind die Abweichungen von diesem immens,
    Nur für ein Gitter ($\sfrac{1}{d} = \SI[per-mode=reciprocal]{100}{\per\milli\meter}$)
    ist die relative Abweichung zufriedenstellend,
    wie auch \autoref{tab:wellenlaenge_relerr} zu entnehmen ist.

    Der Mittelwert über die vier berechneten Wellenlängen,
    \SI{426.85 \pm 7.44}{\nano\meter},
    weicht immer noch um \SI{32.55}{\percent} vom theoretischen Wert ab.

    \begin{table}[H]
    \centering
    \caption{Relative Abweichungen der gemessenen Wellenlängen je Gitter.}
    \label{tab:wellenlaenge_relerr}
    \begin{tabular}{S S}
    \toprule
    $\sfrac{1}{d} \mathbin{/} \si{\per\milli\meter}$ &
    $\text{rel. Abweichung} \mathbin{/} \si{\percent}$ \\
    \midrule
      80 & 27.30 \\ % \pm 0.86
     100 &  1.73 \\ % \pm 1.19
     600 & 56.53 \\ % \pm 3.68
    1200 & 48.09 \\ % \pm 2.54
    \bottomrule
    \end{tabular}
    \end{table}

    …
    % TODO: Zu den Fehlerquellen ↓
    Wenn der Schirm nicht genau senkrecht zur Strahlachse ausgerichtet ist,
    erhöht sich der gemessene Abstand der Intensitätsmaxima und somit gemäß \autoref{eqn:wellenlaenge} auch die Wellenlänge.
    Dieser Effekt konnte durch Mittelung nur teilweise kompensiert werden,
    da links und rechts vom Hauptmaximum unterschiedlich viele Nebenmaxima vermessen werden konnten.
    Allerdings kann dieser Effekt nicht hauptsächlich für die Abweichungen verantwortlich sein,
    da die gemessenen Wellenlängen (mit Ausnahme des Gitters mit $\sfrac{1}{d} = \SI[per-mode=reciprocal]{100}{\per\milli\meter}$)
    \textit{kleiner} als die theoretische Wellenlänge sind.


\subsection{Mögliche Fehlerquellen}
    Bei der \hyperref[sec:auswertung:wellenlaenge]{Ausmessung der Wellenlänge} durch Messung der Abstände der Beugungsmaxima wurde zur Messung ein einfaches Maßband verwendet.
    Da die Beugungsmaxima auf einen großen Schirm gestrahlt wurden und der Raum vollständig abgedunkelt war,
    ist es gut möglich,
    dass beim Ablesen der Abstände Fehler aufgetreten sind.
    % TODO: Vermindert das nicht vielmehr die Fehler? ↑
    % Gemeint war das Ablesen *des Maßbands*
    Die Maxima hatten zudem eine gewisse Ausdehnung,
    welche beim Abstand nicht berücksichtigt wurde.
    Gerade bei den optischen Gittern mit vielen Linien pro \si{\milli\meter} waren nur drei Maxima zu erkennen,
    da das Gitter auf der Schiene nicht weiter verschoben werden konnte.
    Für diese Fälle stehen demnach weniger Messwerte zur Verfügung,
    was eine größere Unsicherheit im Mittelwert bedeutet.


    Bei der \hyperref[sec:auswertung:tem_moden]{Vermessung der Moden} konnte die verwendete Blende nur \SI{10}{\milli\meter} in beide Richtungen verschoben werden,
    sodass möglicherweise nicht die ganze Modenverteilung ausgemessen werden konnte.
    Zudem gelang es erst bei höheren Intensitäten
    die \TEM{01}-Mode zu stabilisieren und in eine ungefähre Position zu verschieben,
    damit sie ausgemessen werden konnte.
    Es hätte bestimmt eine bessere Position geben können,
    um die Mode vollständig aufnehmen zu können.\\
    Bei der longitudinalen Mode konnte das Oszilloskop nicht gut verstellt werden,
    sodass möglicherweise nicht angezeigte Peaks nicht mit aufgenommen wurden.


    Für die \hyperref[sec:auswertung:stabilitaetsbedingung]{Überprüfung der Stabilitätsbedingung}
    wurde der Abstand zwischen den Resonatorspiegeln ebenfalls nur mit einem Maßband vermessen;
    es können also auch hier wieder Messfehler aufgetreten sein.
    Zudem wurde das Maßband durch das zwischenstehende Laserrohr leicht nach oben gebogen,
    sodass sich keine Gerade ergab.
    Der in \autoref{sec:stabilitaet} berechnete theoretische Wert für den Resonator aus zwei konkaven Spiegeln
    konnte aufgrund der nicht ausreichenden Länge der Metallschiene nicht erreicht werden.


    Schließlich ist anzumerken,
    dass die Spiegel zwischendurch gereinigt werden mussten,
    da aufgrund von während der Durchführung dort abgelagerten Staubpartikeln kein Laserstrahl entstehen konnte.
    Diese Staubpartikel können auch bei funktionierendem Laser die Intensität beeinflusst haben.
    Das Verstellen der Spiegel für eine maximale Intensität geschah nach Fingergefühl und Augenmaß,
    es könnte also eine bessere Spiegelkonfiguration zur Messung der Intensitäten gegeben haben.

% TODO: Dies waren meine Ideen. Sie sind ggf. noch in den Text zu integrieren.
\begin{itemize}
    \item Manuelles Nachführen der Photodiode
    \item Ablesen der Beugungsmaxima von der Seite
    \item Streu-/Restlicht im Raum
    \item Starke Schwankungen der gemessenen Intensität
    \item Stabilitätsbedingung: Begrenzte Zeit, um Spiegel jeweils einzujustieren
\end{itemize}
