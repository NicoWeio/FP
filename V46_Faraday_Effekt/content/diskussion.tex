\section{Diskussion}
\label{sec:diskussion}

Im Folgenden werden die in \autoref{sec:auswertung} berechneten Werte für die effektive Masse der freien Elektronen diskutiert.

\subsection{Abweichungen}

Die relative Abweichung der berechneten Werte wird mit der Gleichung
\begin{equation*}
    \sigma = \frac{\lvert m^* - m^*_\text{Theorie} \rvert}{m^*_\text{Theorie}} \cdot \SI{100}{\percent}
\end{equation*}
bestimmt,
wobei der Literaturwert der effektiven Masse von GaAs durch $m^*_\text{Theorie} = 0.067 \cdot m_\text{e}$ \cite{theoriewert} gegeben ist.
Für die Probe mit der Dotierung $N=\SI[per-mode=reciprocal]{2.8e18}{\per\cubic\centi\meter}$ ergibt sich eine relative Abweichung von \SI{14.5(2.9)}{\percent} des berechneten Wertes $m^*_{N=\SI[per-mode=reciprocal]{2.8e18}{\per\cubic\centi\meter}} = (0.0767 \pm 0.0020) \cdot m_\text{e}$ zum Literaturwert.
Die berechnete effektive Masse $m^*_{N=\SI[per-mode=reciprocal]{1.2e18}{\per\cubic\centi\meter}} = (0.0738 \pm 0.0015) \cdot m_\text{e}$ der anderen Probe weicht um \SI{10.1(2.2)}{\percent} vom Literaturwert ab.
Aufgrund der geringen Abweichung kann geschlossen werden,
dass die Bestimmung der effektiven Masse mit dem hier verwendeten Messverfahren erfolgreich war.


\subsection{Mögliche Fehlerquellen}

Mögliche Ursachen für die genannten Abweichungen können vor allem durch das manuelle Einstellen und Ablesen des Winkels für die minimale Reflexion gegeben sein,
wobei das Ablesen durch Fluktuationen auf dem Oszilloskop erschwert wurde.
Außerdem konnte bei einigen Interferenzfiltern kein eindeutiges Minimum gefunden werden,
da sich der Wert auf dem Oszilloskop auch bei größeren Änderungen des Winkels nicht änderte.
Zudem können Abweichungen durch die nicht perfekte Abschirmung der Photodetektoren vom Umgebungslicht entstanden sein,
was eine mögliche Ursache für das erschwerte Ablesen des Minimums darstellen könnte.
Auch durch eine bessere Justage könnten die Ergebnisse verbessert werden.
