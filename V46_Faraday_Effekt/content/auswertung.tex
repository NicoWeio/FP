\section{Auswertung}
\label{sec:auswertung}

Im folgenden Abschnitt werden die Messwerte ausgewertet,
um schließlich die effektive Masse der Elektronen in GaAs zu bestimmen.
Es werden zwei n-dotierte GaAs-Proben,
sowie eine hochreine Probe untersucht,
deren Länge $L$ und Dotierung $N$ in \autoref{tab:probenwerte} aufgeführt sind.
\begin{table}
    \centering
    \caption{Länge $L$ und Dotierung $N$ der hier untersuchten GaAs-Proben.}
    \label{tab:probenwerte}
    \begin{tabular}{S S S}
        \toprule
         & $L/\si{\milli\meter}$ & $N/\si{\per\cubic\centi\meter}$ \\
        \midrule
        \text{Probe 1} & 1.296 & 2.8e18 \\
        \text{Probe 2} & 1.360 & 1.2e18 \\
        \text{hochrein} & 5.11 & \\
        \bottomrule
    \end{tabular}
\end{table}

\subsection{Bestimmung der maximalen Magnetfeldstärke}

Zur Bestimmung der Magnetfeldstärke,
welche am Ort der Probe vorliegt,
wird das Magnetfeld wie in \autoref{} beschrieben,
vermessen.
Die Probe befindet sich dabei am Ort der maximalen Magnetfeldstärke,
welche in diesem Fall durch $B = \SI{405}{\milli\tesla}$ gegeben ist.
Die entsprechenden Messdaten sind in \autoref{fig:magnetfeldstaerke} dargestellt und in \autoref{tab:messwerte_magnet} aufgelistet.
\begin{figure}
    \centering
    \includegraphics[width=\textwidth]{build/plt/magnet.pdf}
    \caption{Messdaten der Magnetfeldstärke abhängig vom Ort der Hallsonde.
    Die maximale Feldstärke ist durch eine eingezeichnete Konstante makiert.}
    \label{fig:magnetfeldstaerke}
\end{figure}

\begin{table}
    \centering
    \caption{Messwerte der Magnetfeldstärke abhängig vom Ort der Hallsonde.}
    \label{tab:messwerte_magnet}
    \begin{tabular}{S S}
        \toprule
        $z/\si{\milli\meter}$ & $B/\si{\milli\tesla}$ \\
        \midrule
         0 & 281 \\
         2 & 335 \\
         4 & 373 \\
         6 & 394 \\
         8 & 404 \\
        10 & 405 \\
        12 & 398 \\
        14 & 381 \\
        16 & 348 \\
        18 & 293 \\
        20 & 214 \\
        \bottomrule
    \end{tabular}
\end{table}

\subsection{Berechnung der effektiven Masse aus der Faraday-Rotation}

Die Faraday-Rotation wird für zwei verschieden n-dotierte und eine reine GaAs-Probe wie in \autoref{} beschrieben,
gemessen.
Der Rotationswinkel lässt sich dabei mithilfe der Gleichung
\begin{equation}
    \theta = \frac{1}{2} \cdot \lvert \theta_1 - \theta_2 \rvert
\end{equation}
bestimmen,
wobei $\theta_1$ und $\theta_2$ den Winkel bei der jeweiligen Polung des Magnetfelds beschreiben.
Die jeweiligen Messwerte sind in Abhängigkeit der Wellenlänge $\lambda$ in \autoref{tab:messwerte_probe1},
\autoref{tab:messwerte_probe2} und \autoref{tab:messwerte_probe3} aufgeführt,
wobei der Rotationswinkel schließlich auf die Länge der jeweiligen Probe normiert wurde.
\autoref{fig:messwerte_rotation} zeigt den graphischen Verlauf des Rotationwinkels $\theta$ in Abhängigkeit von $\lambda^2$.

\begin{table}
    \centering
    \caption{Messwerte der Faraday-Rotation für die n-dotierte GaAs-Probe mit $N=\SI[per-mode=reciprocal]{2.8e18}{\per\cubic\centi\meter}$.}
    \label{tab:messwerte_probe1}
    \begin{tabular}{S S S S S}
        \toprule
        $\lambda/\si{\micro\meter}$ & $\theta_{1}/\si{\degree}$ & $\theta_{2}/\si{\degree}$ & $\theta/\si{\radian}$ & $\frac{\theta}{L_1}/\si{\radian\per\meter}$ \\
        \midrule
        1.060 &  75.25 &  86.00 & 0.09 &  72.39 \\
        1.290 &  77.00 &  85.08 & 0.07 &  54.43 \\
        1.450 &  78.58 &  87.33 & 0.08 &  58.92 \\
        1.720 &  76.08 &  84.83 & 0.08 &  58.92 \\
        1.960 &  72.25 &  82.17 & 0.09 &  66.77 \\
        2.156 &  73.25 &  84.00 & 0.09 &  72.39 \\
        2.340 &  45.08 &  56.50 & 0.10 &  76.87 \\
        2.510 &  24.33 &  41.00 & 0.15 & 112.23 \\
        2.650 & 338.00 & 352.83 & 0.13 &  99.88 \\
        \bottomrule
    \end{tabular}
\end{table}

\begin{table}
    \centering
    \caption{Messwerte der Faraday-Rotation für die n-dotierte GaAs-Probe mit $N=\SI[per-mode=reciprocal]{1.2e18}{\per\cubic\centi\meter}$.}
    \label{tab:messwerte_probe2}
    \begin{tabular}{S S S S S}
        \toprule
        $\lambda/\si{\micro\meter}$ & $\theta_{1}/\si{\degree}$ & $\theta_{2}/\si{\degree}$ & $\theta/\si{\radian}$ & $\frac{\theta}{L_1}/\si{\radian\per\meter}$ \\
        \midrule
        1.060 &  79.00 &  86.75 & 0.07 & 49.73 \\
        1.290 &  78.75 &  85.08 & 0.06 & 40.64 \\
        1.450 &  81.25 &  86.92 & 0.05 & 36.36 \\
        1.720 &  79.67 &  85.42 & 0.05 & 36.90 \\
        1.960 &  76.00 &  81.17  & 0.05 & 33.15 \\
        2.156 &  76.50 &  82.08 & 0.05 & 35.83 \\
        2.340 &  48.08 &  54.67 & 0.06 & 42.24 \\
        2.510 &  30.00 &  37.58 & 0.07 & 48.66 \\
        2.650 & 339.33 & 348.25 & 0.08 & 57.22 \\
        \bottomrule
    \end{tabular}
\end{table}

\begin{table}
    \centering
    \caption{Messwerte der Faraday-Rotation für die hochreine GaAs-Probe.}
    \label{tab:messwerte_probe3}
    \begin{tabular}{S S S S S}
        \toprule
        $\lambda/\si{\micro\meter}$ & $\theta_{1}/\si{\degree}$ & $\theta_{2}/\si{\degree}$ & $\theta/\si{\radian}$ & $\frac{\theta}{L_1}/\si{\radian\per\meter}$ \\
        \midrule
        1.060 &  75.17 &  99.25 & 0.21 & 41.13 \\
        1.290 &  89.00 &  96.25 & 0.06 & 12.38 \\
        1.450 &  83.08 &  97.00 & 0.12 & 23.77 \\
        1.720 &  83.67 &  92.00 & 0.07 & 14.23 \\
        1.960 &  81.00 &  87.67 & 0.06 & 11.39 \\
        2.156 &  82.42 &  88.08 & 0.05 &  9.68 \\
        2.340 &  53.58 &  58.75 & 0.05 &  8.82 \\
        2.510 &  33.17 &  36.25 & 0.03 &  5.27 \\
        2.650 & 337.33 & 333.50 & 0.03 &  6.55 \\
        \bottomrule
    \end{tabular}
\end{table}

\begin{figure}
    \centering
    \includegraphics[width=\textwidth]{build/plt/rotation.pdf}
    \caption{Messwerte der Rotationswinkel $\theta$ für verschieden dotierte und eine hochreine GaAs-Probe in Abhängigkeit des Quadrats der Wellenlänge.}
    \label{fig:messwerte_rotation}
\end{figure}

Um nun die effektive Masse der Elektronen in GaAs zu bestimmen,
werden jeweils die Messwerte der hochreinen Probe von den Messwerten der dotierten Proben subtrahiert,
um nur die Effekte der freien Elektronen im Halbleiter zu berücksichtigen.
Dies ist in \autoref{} dargestellt.
Zudem wird mithilfe der Gleichung
\begin{equation}
    \frac{\theta}{L} (\lambda^2) = \frac{e^2}{8 \symup{\pi}^2 \epsilon_0 c^2} \frac{1}{{m^{*}}^2} \frac{N B}{n} \cdot \lambda^2
\end{equation}
eine lineare Regression der Verteilung beider Proben durchgeführt.
Dabei ist $e$ die Elementarladung,
$\epsilon_0$ die elektrische Feldkonstante und $n$ der Brechungsindex von GaAs,
welcher durch $n = $ \cite{} gegeben ist.
%Quelle n: https://www.sciencedirect.com/science/article/pii/B9780125444156500182?ref=cra_js_challenge&fr=RR-1
