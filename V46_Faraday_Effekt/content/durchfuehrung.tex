\section{Durchführung}
\label{sec:durchfuehrung}

Zur Messung der Faraday-Rotation wird ein Zwei-Strahl-Verfahren verwendet,
um Störfaktoren wie das Rauschen der Photodetektoren zu minimieren.
Zunächst wird auf den Versuchsaufbau eingegangen,
um anschließend den Messprozess zu beschreiben.


\subsection{Aufbau}
\label{sec:durchfuehrung:aufbau}
\begin{figure}
    \centering
    \includegraphics[width=\textwidth]{content/img/Abb_1.pdf}
    \caption{Schematische Darstellung des Versuchsaufbaus \cite{versuchsanleitung}.}
    \label{fig:aufbau}
\end{figure}

% - Lichtquelle
% - Kondensor
% - Lichtzerhacker
% - Polarisator mit Goniometer
% - Elektromagnet
%     - Konstantstromquelle (Polwender…)
%     - Probe
% - Interferenzfilter
% - Analysator
% - 2x Photodetektor
%     - Zeitkonstante …?
% - Differenzverstärker
% - Selektivverstärker (@Lichtzerhacker)
% - Oszilloskop

Der Versuchsaufbau ist in \autoref{fig:aufbau} gezeigt.
Als \textbf{Lichtquelle} dient eine Halogen-Lampe,
    welche genügend Strahlung im Infrarotbereich emittiert.
Eine \textbf{Kondensorlinse} kollimiert den Lichtstrahl,
    welcher durch den anschließenden \textbf{Lichtzerhacker} zeitlich moduliert wird.
Ein \textbf{Polarisator},
    dessen Winkeleinstellung $\theta$ an einem Goniometer abgelesen werden kann,
gibt die Polarisationsrichtung vor,
    mit der das Licht auf die \textbf{Probe} trifft.
Diese ist im Magnetfeld eines \textbf{Elektromagneten} mit zwei Spulen platziert,
    welcher durch eine \textbf{Konstantstromquelle} mit Strom versorgt wird.
Das Umpolen der Spulen erfolgt (entgegen der Abbildung) durch Umstecken der Kabel.
Mit einem \textbf{Interferenzfilter} wird eine bestimmte Wellenlänge im Infrarotbereich selektiert.
Ein Glan-Thompson-Prisma dient als \textbf{Analysator},
    der das monochromatische Infrarotlicht in zwei Polarisationsrichtungen aufteilt.
Die beiden Strahlen werden mithilfe zweier \textbf{Photowiderstände} aufgenommen, % mit Sammellinse; mit regelbarer Zeitkonstante
    welche durch einen \textbf{Differenzverstärker} miteinander verknüpft sind.
Der \textbf{Selektivverstärker} ist mit dem Lichtzerhacker verbunden,
    um das modulierte Signal mit hoher Güte zu verstärken und somit zu filtern.
Das verstärkte Signal wird schließlich auf einem \textbf{Oszilloskop} abgebildet.



\subsection{Messung}
\label{sec:durchfuehrung:messung}
% █ Justierung
Vor den eigentlichen Messungen wird der Versuchsaufbau justiert.
Dabei wird zunächst die Funktion des Analysators überprüft,
    dessen durchgehender Lichtstrahl bei geeigneter Stellung des Polarisators verschwinden sollte.
Anschließend werden die Photodetektoren ausgerichtet und die Verstärker eingestellt.
% NOTE: Den Elektromagneten haben wir erst im Anschluss ausgemessen.

% █ eigentliche Messung
Für jede Probe (siehe \autoref{tab:probenwerte}) und jeden Interferenzfilter werden zwei Messungen entsprechend der beiden Polungen des Elektromagneten durchgeführt.
Dabei wird die Probe im Luftspalt des Elektromagneten platziert und dahinter einer der Interferenzfilter eingesetzt.
Nun kann die Polarisationsrichtung des auf die Probe fallenden Lichts mit dem Goniometer so eingestellt werden,
dass das auf dem Oszilloskop abgebildete Signal eine minimale Amplitude (im Idealfall Null) aufweist.
Dies ist der Fall, wenn die am Goniometer eingestellte Polarisation die Faraday-Rotation gerade ausgleicht.
Es werden die Wellenlänge $\lambda$ des verwendeten Interferenzfilters und die Einstellung $\theta$ des Goniometers notiert.
Dieser Vorgang wird für alle Interferenzfilter und beide Polungen des Elektromagneten sowie schließlich für alle Proben wiederholt.

% NOTE: Doppelung mit der Auswertung
% Indem auch eine hochreine Probe ohne Dotierung vermessen wird,
% kann die durch im Valenzband gebundene Elektronen verursachte Faraday-Rotation bestimmt und von den anderen Messungen abgezogen werden.

% █ Ausmessen des Elektromagneten
Um die magnetische Flussdichte $B$ im Elektromagneten zu bestimmen,
    der die Probe ausgesetzt ist,
wird zuletzt der Luftspalt im Zentrum mit einer Hall-Sonde entlang der Strahlachse abgefahren.
Das Maximum der so erhaltenen Kurve $B(z)$ dient als Referenzwert bei der Berechnung der effektiven Masse $m^*$.
